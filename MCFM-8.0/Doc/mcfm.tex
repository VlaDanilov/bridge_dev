\documentclass[12pt]{article}
\usepackage{hyperref}

\begin{document}
\def\GeV{\mbox{GeV}}
\def\cteqten{\mbox{1007.2241 [hep-ph]}}
\def\cteqsixsixm{\mbox{0802.0007 [hep-ph]}}
\def\cteqsixonem{\mbox{hep-ph/0303013}}
\def\cteqsix{\mbox{hep-ph/0201195}}
\def\cteqfive{\mbox{hep-ph/9903282}}
\def\cteqfour{\mbox{hep-ph/9606399}}
\def\cteqthree{\mbox{MSU-HEP/41024}}
\def\mrstff{\mbox{hep-ph/0603143}}
\def\mrstohtwo{\mbox{hep-ph/0211080}}
\def\mrstohtwofirst{\mbox{hep-ph/0201127}}
\def\mrstohone{\mbox{hep-ph/0110215}}
\def\mrsninenine{\mbox{hep-ph/9907231}}
\def\mrsnineeight{\mbox{hep-ph/9803445}}
\def\mrsninesix{\mbox{PLB387 (1996) 419}}
\def\mrsninefive{\mbox{PLB354 (1995) 155}}
\def\hmrs{\mbox{Durham DTP-90-04}}
\def\mstwoheight{\mbox{0901.0002 [hep-ph]}}
\def\MCFM{{\tt MCFM }}
\def\pow{{\lower.12ex\hbox{\texttt{\char`\^}}}}

\thispagestyle{empty}
\vspace*{3cm}
\begin{center}
{\Huge MCFM v8.0} \\
\vspace*{0.5cm}
\Large{A Monte Carlo for FeMtobarn} \\
\Large{processes at Hadron Colliders} \\
\vspace*{1.5cm}
{\huge Users Guide} \\
\vspace*{4cm}
{\it Authors:} \\
\vspace*{0.2cm}
John M. Campbell ({\tt johnmc@fnal.gov}) \\
R. Keith Ellis ({\tt ellis@fnal.gov}) \\
Walter Giele ({\tt giele@fnal.gov}) \\
Ciaran Williams ({\tt ciaranwi@buffalo.edu}) \\
\vspace*{1.5cm}
{\it \small Updated: May 25, 2016}
\end{center}

\newpage

\tableofcontents

\newpage

\section{Overview}

\MCFM is a parton-level Monte Carlo program that gives NLO predictions
for a range of processes at hadron colliders. The program has been
developed over a number of years and results have been presented in
a number of published papers.  The papers describing the original
code and the most significant developments in the NLO implementation are:
\begin{itemize}
\item J.~M.~Campbell and R.~K.~Ellis, \\
  {\it ``An update on vector boson pair production at hadron colliders,''} \\
  Phys.\ Rev.\ D {\bf 60}, 113006 (1999)
  [arXiv:hep-ph/9905386].
\item J.~M.~Campbell, R.~K.~Ellis and C.~Williams, \\
  {\it ``Vector boson pair production at the LHC,''} \\
  JHEP {\bf 1107}, 018 (2011)
  [arXiv:1105.0020 [hep-ph]]. 
\item J.~M.~Campbell, R.~K.~Ellis and W.~Giele, \\
  {\it ``A Multi-Threaded Version of MCFM''}, \\
    EPJ {\bf C75}, 246 (2015)
    [arXiv:1503.06182 [hep-ph]].
\end{itemize}
 
As of v8.0 MCFM can also compute selected color-singlet processes through NNLO in QCD
perturbation theory.  The processes available at this precision, as well as
benchmark numbers, are detailed in Section~\ref{sec:NNLO}.  When using MCFM 8.0
for NNLO calculations please refer to:
\begin{itemize}
\item 
  R.~Boughezal, J.~M.~Campbell, R.~K.~Ellis, \\
   C.~Focke, W.~Giele, X.~Liu,~F. Petriello and  C.~Williams, \\
  {\it ``Color singlet production at NNLO in MCFM''},
  arXiv:1605.08011.
\end{itemize}

Other relevant references, corresponding to publications associated with the
implementation of specific processes at NLO and NNLO, are listed
in Appendix~\ref{MCFMrefs}.  Appendices~\ref{changelog8.0}--\ref{changelog} contain
a record of changes to the code since v6.0.

\section{Installation}

The tar'ed and gzip'ed package may be downloaded from
the \MCFM home-page at {\tt http://mcfm.fnal.gov}.
After extracting, the source can be initialized by running the
{\tt ./Install} command and then compiled with {\tt make}. The
{\tt ./Install} script may be edited prior to running, to include
the locations of the CERNLIB and LHAPDF libraries, if desired.
The code has been developed and tested under Redhat Linux and Mac OSX,
using the compiler {\tt gfortran}. Please report
any compilation problems under other operating systems to the authors.
Note that, as of version 6.0, the code requires a Fortran90 compiler.

The directory structure of the installation is as follows:
\begin{itemize}
\item {\tt Doc}. The source for this document.
\item {\tt Bin}. The directory containing the executable {\tt mcfm\_omp},
and various essential files -- notably the options file {\tt input.DAT}.
\item {\tt Bin/Pdfdata}. The directory containing the PDF data-files.
\item {\tt obj\_omp}. The object files produced by the compiler. 
\item {\tt src}. The Fortran source files in various subdirectories.
\item {\tt QCDLoop}. The source files to version 1.95 of the Fortran
library QCDLoop~\cite{Ellis:2007qk}. The location of these libraries
is set in the {\tt makefile} (by {\tt QLDIR} and {\tt FFDIR}) and may be
changed to reflect existing installations if desired. 
\end{itemize}
The files which it is most likely that the user will need to modify
are located in {\tt src/User}. It is convenient, if one wants to 
modify one of these files, (or any other file in the subdirectories of the 
{\tt src} directory),
to copy it first to the directory where the user has installed MCFM.
The makefile will use this file in preference to the identically named
file in the sub-directories of {\tt src}.
 
\subsection{OpenMP and MPI}
As of version 7.0, \MCFM has been modified 
using OpenMP (Open multi-processing)  to implement multi-threading. 
By using OpenMP, \MCFM compiled with the appropriate flags
will execute on any processor, 
automatically adjusting to the number of available threads.
The integration routine Vegas now distributes the event evaluation over the threads, 
and combines all events at the end of every iteration to
optimize the numerical integration.
Special care has been taken that the results of the
Monte Carlo integration are independent of the 
number of threads used.

In version 8.0 the OpenMP version of MCFM is compiled by default.  To revert
to a single core version, the user should first ensure
that all the libraries are compiled appropriately.  This is achieved
by running the command {\tt ./Install\_noomp}, in direct analogy with the
normal {\tt Install} script described above.  After this is complete,
the {\tt makefile} must be edited and the value of the variable
{\tt USEOMP} changed to {\tt NO}.  This is specified near the top
of the file and care must be taken to ensure that no trailing characters
are included.  The code can then be compiled using {\tt make},
with the resulting executable {\tt mcfm} placed in the {\tt Bin}
directory.  Note that the associated object files are stored in an
additional subdirectory, {\tt obj}.  The use of this version
of the code is otherwise identical to the ordinary one, apart from
the replacement of {\tt ./mcfm\_omp} by {\tt ./mcfm}.

Two environment variables are useful when using the OMP version of MCFM.
The first, {\tt OMP\_STACKSIZE} may need to be set in order for the program
to run correctly.  On some systems, depending on the OMP implementation,
the program will crash when calculating some of the more complicated processes,
for example $W+2$~jet production at NLO.  Setting this
variable to {\tt 16000}, for instance in the Bash shell by using the
command {\tt export OMP\_STACKSIZE=16000}, has been found to be sufficient
for all processes.  The second useful variable is {\tt OMP\_NUM\_THREADS}
which may be used to directly control the number of threads used during
OMP execution (the default is the maximum number of threads available
on the system).

In version 8.0 it is also possible to run MCFM on a cluster using MPI
(Message Passing Interface).  To run in this  mode, edit the flag {\tt USEMPI}
in the makefile to {\tt YES} and specify compilers in the makefile
that are MPI-capable.  The appropriate lines to edit may be found by
searching for the default compilers in the distributed version,
which are located in {\tt /usr/local/openmpi}.  Note that, in MPI mode, the code
also automatically uses the OpenMP improvements.

\subsection{Installation as a library}
It is also possible to compile all of \MCFM as a library.
This can be of use if the user desires to use routines of \MCFM 
in association with another program. This is achieved by the make command
{\tt make mcfmlib}.  The \MCFM executable can also be compiled using
the library with {\tt make mcfmalt}.
 
\section{Input parameters}
\label{Input_parameters}
\MCFM now allows the user to choose between a number of schemes
for defining the electroweak couplings. These choices are summarized
in Table~\ref{ewscheme}. The scheme is selected by modifying the
value of {\tt ewscheme} in {\tt src/User/mdata.f} prior to compilation, 
which also contains
the values of all input parameters (see also Table~\ref{default}).

\begin{table}
\begin{center}
\begin{tabular}{|c|c|c|c|c|c|c|} \hline
 Parameter & Name & Input Value
 & \multicolumn{4}{c|}{Output Value determined by \tt ewscheme} \\
\cline{4-7}
& ({\tt \_inp}) & & {\tt -1} & {\tt 0} & {\tt 1} & {\tt 2} \\ \hline
$G_F$            & {\tt Gf}      & 1.16639$\times$10$^{-5}$ 
 & input & calculated & input & input \\
$\alpha(M_Z)$    & {\tt aemmz}   & 1/128.89                 
 & input & input & calculated & input \\
$\sin^2 \theta_w$& {\tt xw}      & 0.2223               
 & calculated & input & calculated & input \\
$M_W$            & {\tt wmass}   & 80.385 GeV                
 & input & calculated & input & calculated \\
$M_Z$            & {\tt zmass}   & 91.1876 GeV               
 & input & input & input & calculated \\
$m_t$            & {\tt mt}      & {\tt input.DAT}                  
 & calculated & input & input & input \\
\hline
\end{tabular}
\caption{Different options for the scheme used to fix the electroweak
parameters of the Standard Model and the corresponding default input
values. $M_W$ and $M_Z$ are taken from ref.~\cite{Amsler:2008zzb}.}
\label{ewscheme}
\end{center}
\end{table}

Starting from version 5.2 of the code, the default scheme has been
changed from {\tt ewscheme=-1} (as in previous versions) to
{\tt ewscheme=+1}. As described below, this corresponds to a scheme
in which the top quark mass is an input parameter so that it is
more suitable for many processes now included in the program.

The choice of ({\tt ewscheme=-1}) enforces the use of an effective field
theory approach, which is valid for scales below the top mass. In this
approach there are 4 independent parameters (which we choose to be
$G_F$, $\alpha(M_Z)$, $M_W$ and $M_Z$). For further details,
see Georgi~\cite{Georgi:1991ci}.

For all the other schemes ({\tt ewscheme=0,1,2}) the top mass is simply
an additional input parameter and there are 3 other independent
parameters from the remaining 5. The variable {\tt ewscheme} then performs
exactly the same role as {\tt idef} in MadEvent~\cite{Maltoni:2002qb}.
{\tt ewscheme=0} is the old MadEvent default and {\tt ewscheme=1} is the
new MadEvent default, which is also the same as that used in 
Alpgen~\cite{Alpgen} and LUSIFER~\cite{Lusifer} 
For processes in which the top quark is directly produced  it is 
preferable to use  the schemes ({\tt ewscheme=0,1,2}), since in these schemes
one can adjust the top mass to its physical value (in the input file,
{\tt input.DAT}).

\begin{table}
\begin{center}
\begin{tabular}{|c|c|c|} \hline
Parameter & Fortran name & Default value \\ 
\hline
$m_\tau$         & {\tt mtau}      & 1.777 GeV            \\
$m^2_\tau$& {\tt mtausq}  & 3.1577 GeV$^2$     \\
$\Gamma_\tau$    & {\tt tauwidth}& 2.269$\times$10$^{-12}$~GeV \\
$\Gamma_W$       & {\tt wwidth}  & 2.093 GeV               \\
$\Gamma_Z$       & {\tt zwidth}  & 2.4952 GeV               \\
$V_{ud}$         & {\tt Vud}     & 0.975                  \\
$V_{us}$         & {\tt Vus}     & 0.222             \\
$V_{ub}$         & {\tt Vub}     & 0.                     \\
$V_{cd}$         & {\tt Vcd}     & 0.222             \\
$V_{cs}$         & {\tt Vcs}     & 0.975                  \\
$V_{cb}$         & {\tt Vcb}     & 0.                     \\
\hline
\end{tabular}
\caption{Default values for the remaining parameters in \MCFM.
$\Gamma_W$ and $\Gamma_Z$ from ref.~\cite{Amsler:2008zzb}.}
\label{default} 
\end{center}
\end{table}

In the same file ({\tt mdata.f}) one can also choose the definition
that the program uses for computing transverse quantities, namely
transverse momentum or transverse energy. These are defined by,
\begin{eqnarray}
\mbox{transverse momentum:} & \sqrt{p_x^2+p_y^2} \; ,\nonumber \\
\mbox{transverse energy:}   &
 \frac{E \sqrt{p_x^2+p_y^2}}{\sqrt{p_x^2+p_y^2+p_z^2}} \; .
\end{eqnarray}
The two definitions of course coincide for massless particles.
The chosen definition is used for all cuts that are applied to the
process and it is the one that is used in the default set of histograms.

\subsection{Parton distributions}
The value of $\alpha_S(M_Z)$ is not adjustable; it is hardwired with the
parton distribution. In addition, the parton distribution also specifies
the number of loops that should be used in the running of $\alpha_S$.
The default mode of operation is to choose from a
collection of modern parton distribution functions that are included with
\MCFM.  The distributions, together with their associated $\alpha_S(M_Z)$
values, are given in Table~\ref{pdlabelrecent}.   In addition to these
choices, a number of historical PDF sets are also available;  for details, see
Appendix~\ref{olderPDFs}.  Note that, due to the memory requirements for
using the NNPDF sets, in OpenMP operation it is usually necessary to increase
the value of the environment variable {\tt OMP\_STACKSIZE} to avoid
segmentation faults.
%
\begin{table}[h]
\begin{center}
\begin{tabular}{|c|c|c|c|}
\hline
{\tt pdlabel}  & $\alpha_S(M_Z)$ & order & reference \\
\hline
{\tt mstw8lo}  & 0.1394 & 1     & \cite{Martin:2009iq} \\
{\tt mstw8nl}  & 0.1202 & 2     & \cite{Martin:2009iq} \\
{\tt mstw8nn}  & 0.1171 & 3     & \cite{Martin:2009iq} \\
{\tt MMHT\_lo}  & 0.135  & 1     & \cite{Harland-Lang:2014zoa} \\
{\tt MMHT\_nl}  & 0.120  & 2     & \cite{Harland-Lang:2014zoa} \\
{\tt MMHT\_nn}  & 0.118  & 3     & \cite{Harland-Lang:2014zoa} \\
\hline
{\tt CT10.00}  & 0.118  & 2     & \cite{Lai:2010vv} \\
{\tt CT14.LL}  & 0.130  & 1     & \cite{Dulat:2015mca} \\
{\tt CT14.NL}  & 0.118  & 2     & \cite{Dulat:2015mca} \\
{\tt CT14.NN}  & 0.118  & 3     & \cite{Dulat:2015mca} \\
\hline
{\tt NN2.3NL}  & 0.118  & 2     & \cite{Ball:2012cx} \\
{\tt NN2.3NN}  & 0.118  & 3     & \cite{Ball:2012cx} \\
{\tt NN3.0LO}  & 0.118  & 1     & \cite{Ball:2014uwa} \\
{\tt NN3.0NL}  & 0.118  & 2     & \cite{Ball:2014uwa} \\
{\tt NN3.0NN}  & 0.118  & 3     & \cite{Ball:2014uwa} \\
\hline
\end{tabular}
\end{center}
\caption{Modern PDF sets that are available in the code,
their corresponding values of $\alpha_S(M_Z)$ and order of running,
and a reference to the paper
that describes their origin.  Further sets, of a more historical nature, are
listed in Appendix~\ref{olderPDFs}.
\label{pdlabelrecent}}
\end{table}

By editing the {\tt Makefile}, it is straightforward to switch to
either the {\tt PDFLIB} or the {\tt LHAPDF} parton distribution
function implementations.

To use {\tt PDFLIB}, one must first set the variable {\tt CERNLIB}
in the makefile to point to the directory that contains
{\tt libpdflib804.a} and then modify {\tt PDFROUTINES} to
take the value {\tt PDFLIB}. The parameters to choose the
pdf set are then specified in {\tt Bin/input.DAT}.

To use {\tt LHAPDF}, one must first set the variable {\tt LHAPDFLIB}
in the makefile to point to the directory that contains
{\tt libLHAPDF.a} and then modify {\tt PDFROUTINES} to
take the value {\tt LHAPDF}. Note that, in newer versions of LHAPDF,
it may be easier to link against the static LHAPDF libraries by passing
a ``{\tt -static}'' flag to the compiler via the {\tt FFLAGS} variable
in the makefile, rather than using the (default) shared libraries.
Particularly when compiling against both LHAPDF and CERNLIB, it may
be useful to link only LHAPDF in a static manner. This can be achieved
using g77, for instance, by replacing
 ``{\tt -lLHAPDF}'' with ``{\tt -Wl,-Bstatic -lLHAPDF -Wl,-Bdynamic}''
in the makefile. This version of \MCFM has been explicitly tested
against {\tt LHAPDF-5.5.0}.

If at the first execution of mcfm the following error appears:
\begin{verbatim}
./mcfm: error while loading shared libraries: libLHAPDF.so.0: cannot open
shared object file: No such file or directory
\end{verbatim}

you should add the following lines to the shell (bash) login script :
\begin{verbatim}
export LHAPDFSYS=/yourpath/LHAPDF-X.Y.Z
export PATH=${PATH}:${LHAPDFSYS}/bin
export LD_LIBRARY PATH=${LD_LIBRARY PATH}:${LHAPDFSYS}/lib
\end{verbatim}

The parameters to choose the
pdf set are then provided in {\tt Bin/input.DAT} - 
the name of the group and the integer specifying 
the set.
\MCFM expects to find the LHAPDF grids in a sub-directory of {\tt Bin} called
{\tt PDFsets}. It is easiest to simply create a symbolic link of the directory where the
grids actually reside to a subdirectory of {\tt Bin} called {\tt PDFsets}.

One may always return to the built-in distributions by resetting
{\tt PDFROUTINES} to take the value {\tt NATIVE}  in the makefile,
(and recompiling).

\section{Benchmark results at NNLO }
\label{sec:NNLO}
We perform all benchmark calculations with the default set of EW parameters
and for the LHC operating at $\sqrt s = 14$~TeV.  We allow all
vector bosons to be off-shell ({\tt zerowidth} is {\tt .false.})
and include their decays ({\tt removebr} is {\tt .false.}).
For each Higgs boson process we consider the decay $H \to \tau^- \tau^+$.
For parameters that are set in the input file we use,
\begin{eqnarray}
m_H = 125~\mbox{GeV} \,, \quad
m_t = 173.3~\mbox{GeV} \,, \quad 
m_b = 4.66~\mbox{GeV} \,, 
\end{eqnarray}
and we use the NNLO CT14 pdf set (i.e. {\tt pdlabel} is {\tt CT14.NN}) with
$\mu_F = \mu_R = Q^2$ (i.e. we set {\tt dynamicscale} equal to
either {\tt m(34)} or {\tt m(3456)}, as appropriate).
Our generic set of cuts is,
\begin{eqnarray}
&& p_T(\mbox{lepton}) > 20~\mbox{GeV} \,, \quad
|\eta(\mbox{lepton})| < 2.4 \,, \quad \nonumber \\
&& p_T(\mbox{photon 1}) > 40~\mbox{GeV} \,, \quad
   p_T(\mbox{photon 2}) > 25~\mbox{GeV} \,, \quad \nonumber \\
&&|\eta(\mbox{photon})| < 2.5 \,, \quad 
\Delta R(\mbox{photon 1, photon 2}) > 0.4 \,, \quad \nonumber \\
&& E_T^{\mbox{miss}} > 30~\mbox{GeV} \,, \quad
\end{eqnarray}
For $Z$ production we also impose a minimum $Z^*$ virtuality ({\tt m34min})
of $40$~GeV.

For providing benchmark runs we choose a set of integration parameters
that provides approximately 1\% (or smaller) Monte Carlo uncertainties.
These are {\tt itmx1 = 4}, {\tt itmx2 = 10} and {\tt ncall1 = ncall2 = 400000}.
We set {\tt taucut} to {\tt 1\%acc} to achieve a similar level of uncertainty
in the total NNLO predictions from power corrections that have been neglected. 
Our benchmarks are shown in Table~\ref{NNLObenchmarks}.  These benchmark cross-sections
may not be sufficiently accurate for all phenomenological applications but they should
be able to be reproduced relatively easily, even on a desktop machine.

\begin{table}
\begin{center}
\begin{tabular}{|l|l|l|l|} \hline
Process & {\tt nproc} & $\sigma_{NLO} \pm \delta\sigma_{NLO}^{MC} $ & $\sigma_{NNLO} \pm \delta\sigma_{NNLO}^{MC} \pm \delta\sigma_{NNLO}^{pc}$ \\ 
\hline
$W^+$ & {\tt 1}    & $4.218 \pm 0.002$ nb & $4.327 \pm 0.038 \pm 0.043$ nb\\
$W^-$ & {\tt 6}    & $3.314 \pm 0.001$ nb & $3.256 \pm 0.035 \pm 0.033$ nb\\
$Z  $ & {\tt 31}   & $884.9 \pm 0.3$ pb & $897.1 \pm 5.2 \pm 9.0$ pb\\
$H  $ & {\tt 112}  & $1.395 \pm 0.001$ pb & $1.857 \pm 0.007 \pm 0.019$ pb\\
$\gamma\gamma  $ & {\tt 285}  & $27.90 \pm 0.01$ pb & $43.62 \pm 0.09 \pm 0.44$ pb\\
$W^+H$ & {\tt 91}    & $2.203 \pm 0.002$ fb & $2.279 \pm 0.011 \pm 0.023$ fb\\
$W^-H$ & {\tt 96}    & $1.495 \pm 0.001$ fb & $1.533 \pm 0.006 \pm 0.015$ fb\\
$ZH$   & {\tt 110}   & $0.7537 \pm 0.0004$ fb & $0.8453 \pm 0.0021 \pm 0.0085$ fb\\
\hline
\end{tabular}
\caption{Benchmark cross-sections at NLO and NNLO, using the parameters
and settings described in the text.  $\delta\sigma^{MC}$ represents the uncertainty
from Monte Carlo statistics, while $\delta\sigma^{pc}$ is an estimate of the
uncertainty due to neglected power corrections at NNLO.}
\label{NNLObenchmarks} 
\end{center}
\end{table}

\section{Runtime options}

{\tt mcfm} execution is performed in the {\tt Bin/} directory,
with syntax:
\begin{center}
{\tt mcfm [}{\it mydir}{\tt ] [}{\it myfile}{\tt .DAT]}
\end{center}
The executable {\tt mcfm} is automatically moved to {\tt Bin} by the makefile.
If no command line options are given, then {\tt mcfm} will default
to using the file {\tt input.DAT} in the current directory for
choosing options\footnote{Note that this is very different from
previous versions of \MCFM. All auxiliary input files from v3.2 and
earlier have now been incorporated into a single file.}.
The different possibilities are summarized in Table~\ref{clopts}.
\begin{table}
\begin{center}
\begin{tabular}{l|cl}
Command executed && Location of input file \\
\hline
{\tt mcfm}                      && {\tt input.DAT} \\
{\tt mcfm myfile.DAT}           && {\tt myfile.DAT} \\
{\tt mcfm mydir}                && {\tt mydir/input.DAT} \\
{\tt mcfm mydir myfile.DAT}     && {\tt mydir/myfile.DAT} \\
\end{tabular}
\end{center}
\caption{Summary of command line options for running {\tt mcfm}.}
\label{clopts}
\end{table}
In addition, if a working directory {\it mydir} is specified then
output files will also be produced in this directory. By using these
options one may, for instance, keep all input and output files for
different processes in separate directories.

Each parameter in the input file is specified by a line such as
\begin{displaymath}
{\tt value} \hspace{3cm} {\tt [parameter]}
\end{displaymath}
and we will give a description of all the parameters below, together with
valid and/or sensible inputs for ${\tt value}$. Groups of parameters
are separated by a blank line and a description of that section, for
readability.

\begin{itemize}
\item {\tt file version number}. This should match the version number
that is printed when {\tt mcfm} is executed.

\begin{center}
\{blank line\} \\
{\tt [Flags to specify the mode in which \MCFM is run] }
\end{center}

\item {\tt nevtrequested}. The default for this parameter is {\tt -1} and for the following three
parameters it is {\tt .false.}.  This corresponds to the usual mode
of operation where the program produces a cross section and a selection of histograms.
It is possible to generate n-tuples instead of histograms,
as well as unweighted events, for some processes. Please refer to
Section~\ref{subsec:otheroutput} for further details.
\item {\tt creatent}. {\it See above.}
\item {\tt dswhisto}. {\it See above.}
\item {\tt creategrid}. Flag to control whether or not to write out a grid file suitable for
further processing by APPLgrid.  Please contact the APPLgrid authors for further information.
\item {\tt writerefs}. Flag to control whether or not the program writes a list
of appropriate references to the output at the end of the run. 
\item {\tt writetop}. Flag to control whether or not a Topdrawer histogram output file
is produced. Please refer to Section~\ref{sec:output} for further details.
\item {\tt writedat}. Flag to control whether or not the plain histogram output file
is produced. Please refer to Section~\ref{sec:output} for further details.
\item {\tt writegnu}. Flag to control whether or not a gnuplot histogram output file
is produced. Please refer to Section~\ref{sec:output} for further details.
\item {\tt writeroot}. Flag to control whether or not a ROOT script for plotting
histograms is produced. Please refer to Section~\ref{sec:output} for further details.
\item {\tt writepwg}. Flag to control whether a powheg-style analysis file is produced.
This option is available only for a limited number of processes. As currently implemented it
should be viewed as a development tool, not yet fully supported for the general user.

\begin{center}
\{blank line\} \\
{\tt [General options to specify the process and execution] }
\end{center}

\item {\tt nproc}.
The process to be studied is given by
choosing a process number, according to Table~\ref{nproctable}
in Appendix~\ref{MCFMprocs}.
$f(p_i)$ denotes a generic partonic jet. Processes denoted as
``LO'' may only be calculated in the Born approximation. For photon
processes, ``NLO+F'' signifies that the calculation may be performed
both at NLO and also including the effects of photon fragmentation
and experimental isolation. In contrast, ``NLO'' for a process involving
photons means that no fragmentation contributions are included and isolation
is performed according to the procedure of Frixione~\cite{Frixione:1998jh}.
\item {\tt part}.
This parameter has 5 possible values, described below:
\begin{itemize}
\item {\tt lo} (or {\tt lord}).
The calculation is performed at leading order only.
\item {\tt virt}.
Virtual (loop) contributions to the next-to-leading order result are
calculated (+counterterms to make them finite), including also the
lowest order contribution.
\item {\tt real}.
In addition to the loop diagrams calculated by {\tt virt}, the full
next-to-leading order results must include contributions from diagrams
involving real gluon emission (-counterterms to make them finite).
Note that only the sum of the {\tt real} and the {\tt virt} contributions
is physical.
\item {\tt nlo} (or {\tt tota}).
For simplicity, the {\tt nlo} option simply runs the {\tt virt} and
{\tt real} real pieces in series before performing a sum to obtain
the full next-to-leading order result. In this case, the number of
points specified by {\tt ncall1} and {\tt ncall2} is automatically
increased when performing the {\tt real} calculation. Sometimes
it may be more efficient to do run the pieces separately by hand, 
(c.f. {\tt ncall} below. For photon processes that include fragmentation,
{\tt nlo} also includes the calculation of the fragmentation ({\tt frag})
contributions.
\item {\tt nlocoeff}.
This computes only the contribution of the NLO coefficient;  it is equivalent
to running {\tt nlo} and then subtracting the result of {\tt lo}.
\item {\tt todk}
Processes 114, 161, 166, 171, 176, 181, 186, 141, 146, 149, 233, 238, 501, 511 only, see sections~\ref{subsec:stop} and
\ref{subsec:wt} below.
\item {\tt frag}.
Processes 280, 285, 290, 295, 300-302, 305-307,  820-823 only, see sections~\ref{subsec:gamgam}, \ref{subsec:wgamma} and
\ref{subsec:zgamma} below.
\item {\tt nnlo} (and {\tt nnlocoeff}).
The computation of the NNLO prediction (or the NNLO coefficient in the
expansion) is described separately below.
\end{itemize}

\item {\tt runstring}.
When \MCFM is run, it will write output to several files. The
label {\tt runstring} will be appended to the names of these files.

\item {\tt sqrts}. This is the centre-of-mass energy, $\sqrt{s}$ of
the colliding particles, measured in GeV.

\item {\tt ih1}, {\tt ih2}. The identities of the incoming hadrons
may be set with these parameters, allowing simulations for both
$p{\bar p}$ (such as the Tevatron) and $pp$ (such as the LHC). 
Setting {\tt ih1} equal to ${\tt +1}$ corresponds to
a proton, whilst ${\tt -1}$ corresponds to an anti-proton.
Values greater than {\tt 1000d0} represent a nuclear collision,
as described in Section~\ref{sec:nucleus}.

\item {\tt hmass}. For processes involving the Higgs boson, this
parameter should be set equal to the value of $M_H$.

\item {\tt scale}. This parameter may be used to adjust the value
of the {\it renormalization} scale. This is the scale
at which $\alpha_S$ is evaluated and will typically be set to
a mass scale appropriate to the process ($M_W$, $M_Z$, $M_t$ for
instance). For processes involving vector bosons, setting this
scale to {\tt -1d0} chooses a scale equal to the average mass of
the bosons involved.

\item {\tt facscale}. This parameter may be used to adjust the value
of the {\it factorization} scale and will typically be set to
a mass scale appropriate to the process ($M_W$, $M_Z$, $M_t$ for
instance). As above, setting it to {\tt -1d0} will choose an
appropriate value for certain processes.

\item {\tt dynamicscale} This character string is used to specify whether
the renormalization, factorization and fragmentation scales are dynamic, i.e. recalculated
on an event-by-event basis. If this string is set to either `{\tt .false.}',
`{\tt no}' or `{\tt none}' then the scales are fixed for all events at the values
specified by {\tt scale}, {\tt facscale} and {\tt frag\_scale} in the input file.

The type of dynamic scale to be used is selected by using a particular string
for the variable {\tt dynamicscale}, as indicated in Table~\ref{dynamicscales}.
Not all scales are defined for each process, with program execution halted if
an invalid selection is made in the input file.
The selection chooses a reference scale, $\mu_0$. The actual scales used in
the code are then,
\begin{equation}
\mu_{\rm ren} = {\tt scale} \times \mu_0 \;, \qquad
\mu_{\rm fac} = {\tt facscale} \times \mu_0 \;, \qquad
\mu_{\rm frag} = \mu_{\rm ren} \;. \nonumber
\end{equation}
Note that, for simplicity, the fragmentation scale (relevant only for processes
involving photons) is set equal to the renormalization scale.
In some cases it is possible for the dynamic scale to become very large. This can cause problems 
with the interpolation of data tables for the PDFs and fragmentation functions. As a result if a dynamic scale 
exceeds a maximum of $60$ TeV (PDF) or $990$ GeV (fragmentation) this value is set by default to the maximum. 
%
\begin{table}
\begin{center}
\begin{tabular}{|l|l|l|}
\hline
{\tt dynamic scale} & $\mu_0^2$ & comments\\
\hline 
{\tt m(34)} & $(p_3+p_4)^2$ & \\
{\tt m(345)} & $(p_3+p_4+p_5)^2$ & \\
{\tt m(3456)} & $(p_3+p_4+p_5+p_6)^2$ & \\
{\tt sqrt(M\pow 2+pt34\pow 2)} & $M^2 + (\vec{p_T}_3 + \vec{p_T}_4)^2$ & $M=$~mass of particle 3+4 \\
{\tt sqrt(M\pow 2+pt345\pow 2)} & $M^2 + (\vec{p_T}_3 + \vec{p_T}_4 + \vec{p_T}_5)^2$ & $M=$~mass of particle 3+4+5 \\
{\tt sqrt(M\pow 2+pt5\pow 2)} & $M^2 + \vec{p_T}_5^2$ & $M=$~mass of particle 3+4 \\
{\tt sqrt(M\pow 2+ptj1\pow 2)} & $M^2 + \vec{p_T}_{j_1}^2$ & $M=$~mass(3+4), $j_1=$ leading $p_T$ jet \\
{\tt pt(photon)} & $\vec{p_T}_\gamma^2$ & \\
{\tt HT} & $\sum_{i=1}^n {p_T}_i$ & $n$ final state particles (partons, not jets) \\
\hline 
\hline\end{tabular}
\end{center}
\caption{Choices of the input parameter {\tt dynamicscale} that result in an event-by-event
calculation of all relevant scales using the given reference scale-squared $\mu_0^2$.
\label{dynamicscales}}
\end{table}

Although not really dynamic scales, note that the strings `{\tt MW}', `{\tt MZ}',
`{\tt MH}' and `{\tt mt}' may be used as shorthand to indicate values of $\mu_0$ equal to
$M_W$, $M_Z$, $M_H$ and $m_t$. 

\item {\tt zerowidth}. When set to {\tt .true.} then all vector
bosons are produced on-shell. This is appropriate for calculations
of {\it total} cross-sections (such as when using {\tt removebr} equal
to {\tt .true.}, below). When interested in decay products of the
bosons this should be set to {\tt .false.}.

\item {\tt removebr}. When set to {\tt .true.} the branching ratios are 
removed for unstable particles such as vector bosons or top quarks. See the
process notes in Section~\ref{sec:specific} below for further details.

\item {\tt itmx1}, {\tt itmx2}. The program will perform two runs of
{\tt VEGAS} - once for pre-conditioning and then the final run to
collect the total cross-section and fill histograms. The number of
sweeps for each run is given by {\tt itmx1} (pre-conditioning)
and {\tt itmx2} (final). The default value for both is {\tt 10}.


\item {\tt ncall1}, {\tt ncall2}. For every sweep of {\tt VEGAS},
the number of events generated will be {\tt ncall1} in the
pre-conditioning stage and {\tt ncall2} in the final run. The number
of events required depends upon a number of factors. The error
estimate on a total cross-section will often be reasonable for a
fairly small number of events, whereas accurate histograms will
require a longer run. As the number of particles in the final state
increases, so should the number of calls per sweep. Typically one
might make trial runs with {\tt part} set to {\tt lord} to determine
reasonable values for {\tt ncall1} and {\tt ncall2}. Such values
should also be appropriate for the {\tt virt} piece of
next-to-leading order and should probably be increased by a factor of
$\sim 5$ for the {\tt real} calculation.

\item {\tt taucut}. This sets the value of the jettiness variable
$T_0$ that separates the resolved and unresolved regions in NNLO
calculations that use zero-jettiness.  Recommended values of this
parameter that yield results that are affected by power corrections
at the $1$\% and $0.2$\% level can be obtained by using the
strings {\tt 1\%acc} and {\tt 0.2\%acc} respectively.

\item {\tt ij}. This is the seed for the {\tt VEGAS} integration
and can be altered to give different results for otherwise identical
runs.

\item {\tt dryrun}. The default value of this parameter is
{\tt .false.}. When set to {\tt .true.} the pre-conditioning sweeps
in the {\tt VEGAS} integration are skipped, with the reported
results coming from a single run, (ie {\tt itmx1} iterations of 
{\tt ncall1} points each)
\item {\tt Qflag}. This only has an effect when running a
$W+2$~jets or $Z+2$~jets process. Please see section~\ref{subsec:w2jets}
below.

\item {\tt Gflag}. This only has an effect when running a
$W+2$~jets or $Z+2$~jets process. Please see section~\ref{subsec:w2jets}
below.

\begin{center}
\{blank line\} \\
{\tt [Heavy quark masses] }
\end{center}
\item {\tt top mass}. The top quark pole mass (in GeV).
\item {\tt bottom mass}. The bottom quark pole mass (in GeV).
\item {\tt charm mass}. The charm quark pole mass (in GeV).

\begin{center}
\{blank line\} \\
{\tt [Pdf selection] }
\end{center}

\item {\tt pdlabel}. The choice of parton distribution is made by
inserting the appropriate 7-character code from Table~\ref{pdlabelmrs}
or~\ref{pdlabelcteq} here.
As mentioned above, this also sets the value of $\alpha_S(M_Z)$.

\item {\tt NGROUP, NSET}. These integers choose the parton distribution
functions to be used when using the PDFLIB package.
\item {\tt LHAPDF group, LHAPDF set}. These choose the parton
distribution functions to be used when using the LHAPDF package --
the group is specified by a character string and the set by an integer.
Please see {\tt http://durpdg.dur.ac.uk/lhapdf/} for further details.
For appropriate PDF sets choosing a value of -1 for the set number ({\tt  LHAPDF set}) 
will perform the calculation of the PDF uncertainties (see also
Section~\ref{sec:histos}, especially the caveat regarding using
non-grid PDF sets).

\begin{center}
\{blank line\} \\
{\tt [Jet definition and event cuts] }
\end{center}

\item {\tt m34min}, {\tt m34max}, {\tt m56min}, {\tt m56max}, {\tt m3456min}, {\tt m3456max}.
These parameters represent a basic set of cuts that may be applied
to the calculated cross-section. The only events that contribute to
the cross-section will have, for example,
{\tt m34min} $<$ {\tt m34} $<$ {\tt m34max} where {\tt m34} is the
invariant mass of particles 3 and 4 that are specified by {\tt nproc}.
{\tt m34min}~$> 0$ is obligatory for processes which can involve a virtual
photon, such as {\tt nproc=31}.
\item {\tt inclusive}.  This logical parameter chooses whether the
calculated cross-section should be inclusive in the number of jets
found at NLO. An {\em exclusive}
cross-section contains the same number of jets at next-to-leading
order as at leading order. An {\em inclusive} cross-section may
instead contain an extra jet at NLO.

\item {\tt algorithm.} This specifies the jet-finding algorithm that
is used, and can take the values
{\tt ktal} (for the Run II $k_T$-algorithm), {\tt ankt} (for the
``anti-$k_T$'' algorithm~\cite{Cacciari:2008}), {\tt cone} (for
a midpoint cone algorithm), {\tt hqrk} (for a simplified cone
algorithm designed for heavy quark processes) and {\tt none} (to
specify no jet clustering at all). The latter option is only a
sensible choice when the leading order cross-section is well-defined
without any jet definition: e.g. the single top process,
$q{\bar q^\prime} \to t{\bar b}$, which is finite as
$p_T({\bar b}) \to 0$.

\item {\tt ptjet\_min, |etajet|\_min, |etajet|\_max}. These specify the values
of $p_{T,{\rm min}}$, $|\eta|_{\rm min}$ and $|\eta|_{\rm max}$ for the
jets that are found by the algorithm. 

\item {\tt Rcut\_jet}. If the final state of the chosen process contains
either quarks or gluons then for each event an attempt will be made
to form them into jets. For this it is necessary to define the
jet separation $\Delta R=\sqrt{{\Delta \eta}^2 + {\Delta \phi}^2}$
so that after jet combination, all jet pairs are separated by
$\Delta R >$~{\tt Rcut\_jet}.

\item {\tt makecuts}. If this parameter is set to {\tt .false.} then
no additional cuts are applied to the events and the remaining
parameters in this section are ignored. Otherwise, events will
be rejected according to a set of cuts that is specified below.
Further options may be implemented by editing {\tt src/User/gencuts.f}.

\item {\tt ptlepton\_min, |etalepton|\_max}. These specify the values
of $p_{T,{\rm min}}$ and $|\eta|_{\rm max}$ for one of the leptons produced
in the process.

\item {\tt etalepton\_veto}. This should be specified as a pair of double
precision numbers that indicate a rapidity range that should be excluded
for the lepton that passes the above cuts.

\item {\tt ptmin\_missing}. Specifies the minimum missing transverse
momentum (coming from neutrinos).

\item {\tt ptlepton(2nd+)\_min, |etalepton(2nd+)|\_max}. These specify
the values of $p_{T,{\rm min}}$ and $|\eta|_{\rm max}$ for the remaining
leptons in the process. This allows for staggered cuts where, for
instance, only one lepton is required to be hard and central.

\item {\tt etalepton(2nd+)\_veto}. This should be specified as a pair of double
precision numbers that indicate a rapidity range that should be excluded
for the remaining leptons.

\item {\tt mtrans34cut}. For general processes, this specifies the
minimum transverse mass of particles 3 and 4,
\begin{equation}
\mbox{general}: \quad 2 p_T(3) p_T(4) \left( 1 - \frac{\vec{p_T}(3) \cdot \vec{p_T}(4)}{p_T(3) p_T(4)} \right) 
> {\tt mtrans34cut} 
\end{equation}
For the $W(\to \ell \nu)\gamma$ process the role of this cut changes, to become
instead a cut on the transverse cluster mass of the $(\ell\gamma,\nu)$ system,
\begin{eqnarray}
 W\gamma: && \left[ \sqrt{m_{\ell\gamma}^2 + |\vec{p_T}(\ell)+\vec{p_T}(\gamma)|^2} + p_T(\nu) \right]^2
  \nonumber \\ &&
  -|\vec{p_T}(\ell)+\vec{p_T}(\gamma)+\vec{p_T}(\nu)|^2 > {\tt mtrans34cut}^2
\end{eqnarray}
For the $Z\gamma$ process this parameter specifies a simple invariant mass cut,
\begin{equation}
 Z\gamma: \quad m_{Z\gamma} > {\tt mtrans34cut}
\end{equation}
A final mode of operation applies to the $W\gamma$ process and is triggered by a negative value
of {\tt mtrans34cut}. This allows simple access to the cut that was employed in v6.0 of the code:
\begin{eqnarray}
 W\gamma, \mbox{obsolete}: &&
   \left[ p_T(\ell) +  p_T(\gamma) +  p_T(\nu) \right]^2 \nonumber \\ 
  &-&|\vec{p_T}(\ell)+\vec{p_T}(\gamma)+\vec{p_T}(\nu)|^2 > |{\tt mtrans34cut}|
\end{eqnarray}
In each case the screen output indicates the cut that is applied.

\item {\tt R(jet,lept)\_min}. Using the definition of $\Delta R$ above,
requires that all jet-lepton pairs are separated by
$\Delta R >$~{\tt R(jet,lept)\_min}.

\item {\tt R(lept,lept)\_min}. When non-zero, all lepton-lepton pairs
must be separated by $\Delta R >$~{\tt R(lept,lept)\_min}.

\item {\tt Delta\_eta(jet,jet)\_min}. This enforces a pseudo-rapidity
gap between the two hardest jets $j_1$ and $j_2$, so that: \\
$|\eta^{j_1} - \eta^{j_2}| >$~{\tt Delta\_eta(jet,jet)\_min}.

\item {\tt jets\_opphem}. If this parameter is set to {\tt .true.},
then the two hardest jets are required to lie in opposite hemispheres,
$\eta^{j_1} \cdot \eta^{j_2} < 0$.

\item {\tt lepbtwnjets\_scheme}. This integer parameter provides no
additional cuts when it takes the value {\tt 0}. When equal to
{\tt 1} or {\tt 2}, leptons are required to lie between the two
hardest jets. With the ordering $\eta^{j_-} < \eta^{j_+}$ for the
pseudo-rapidities of jets $j_1$ and $j_2$: \\
{\tt lepbtwnjets\_scheme = 1} : 
 $\eta^{j_-} < \eta^{\rm leptons} < \eta^{j_+}$; \\
{\tt lepbtwnjets\_scheme = 2} :
 $\eta^{j_-}+{\tt Rcut\_jet} < \eta^{\rm leptons} < \eta^{j_+}-{\tt Rcut\_jet}$.

\item {\tt ptmin\_bjet,  etamax\_bjet}. If {\tt makecuts} is {\tt .true.}
and a process involving $b$-quarks is being calculated, then these can
be used to specify {\em stricter} values of $p_T^{\rm min}$
and $|\eta|^{\rm max}$ for $b$-jets.

\begin{center}
\{blank line\} \\
{\tt [Settings for photon processes] }
\end{center}
\item {\tt frag}. This parameter is a logical variable that determines whether the production of photons by a parton 
fragmentation process is included. If {\tt frag} is set to {\tt .true.} the code uses a a standard cone isolation
procedure (that includes LO fragmentation contributions in the NLO calculation).
If {\tt frag} is set to {\tt .false.} the code implements
a Frixione-style photon cut~\cite{Frixione:1998jh},
\begin{eqnarray}
\sum_{i \in R_0} E_{T,i}^j  < \epsilon_h E_{T}^{\gamma} \bigg(\frac{1-\cos{R_{i\gamma}}}{1-\cos{R_0}}\bigg)^{n} \;.
\label{frixeq}
\end{eqnarray}
In this equation, $R_0$, $\epsilon_h$ and $n$ are defined by {\tt cone\_ang}, {\tt epsilon\_h} 
and {\tt n\_pow}  respectively (see below).
$E_{T,i}^{j}$ is the transverse energy of a parton, $E_{T}^\gamma$ is the
transverse energy of the photon and $R_{i\gamma}$ is the separation between the photon and the parton using the usual definition 
$R=\sqrt{\Delta\phi^2+\Delta\eta^2}$. $n$ is an integer parameter which by default is set to 1 but can be changed by editing the 
file {\tt src/User/frix.f}. 

\item {\tt fragset}. A character*8 variable that is used to choose the particular photon fragmentation set.
Currently implemented fragmentation functions can be called with `{\tt BFGSet\_I}', `{\tt BFGSetII}'~\cite{Bourhis:1997yu}
or `{\tt GdRG\_\_LO}'~\cite{GehrmannDeRidder:1998ba}.  

\item {\tt frag\_scale}. A double precision variable that will be used to choose the scale 
at which the photon fragmentation is evaluated. 

\item {\tt ptmin\_photon}. This specifies the value
of $p_T^{\rm min}$ for the photon with the largest transverse momentum.
Note that this cut, together with all the photon cuts specified in this section
of the input file, are applied even if {\tt makecuts} is set to {\tt .false.}.

\item {\tt etamax\_photon}. This specifies the value
of $|y|^{\rm max}$ for any photons produced in the process.

\item {\tt ptmin\_photon(2nd)} and {\tt ptmin\_photon(3rd)}. These specify the values
of $p_T^{\rm min}$ for the second and third photons, ordered by $p_T$.

\item {\tt R(photon,lept)\_min}. Using the usual definition of $\Delta R$,
this requires that all photon-lepton pairs are separated by
$\Delta R >$~{\tt R(photon,lept)\_min}. This parameter must be non-zero
for processes in which photon radiation from leptons is included.

\item {\tt R(photon,photon)\_min}. Using the usual definition of $\Delta R$,
this requires that all photon pairs are separated by
$\Delta R >$~{\tt R(photon,photon)\_min}.

\item {\tt R(photon,jet)\_min}. Using the usual definition of $\Delta R$,
this requires that all photon-jet pairs are separated by
$\Delta R >$~{\tt R(photon,jet)\_min}.

\item {\tt cone\_ang}. A double precision variable that fixes the cone size ($R_0$) for photon isolation.
This cone is used in both forms of isolation. 

\item {\tt epsilon\_h}. This cut controls the amount of radiation allowed in cone when  {\tt frag} is set to {\tt .true.}. If  {\tt epsilon\_h} $ < 1$ then the photon is isolated using
$\sum_{\in R_0} E_T{\rm{(had)}} < \epsilon_h \, p^{\gamma}_{T}.$ Otherwise {\tt epsilon\_h}  $ > 1$ sets $E_T(max)$ in  $\sum_{\in R_0} E_T{\rm{(had)}} < E_T(max)$.  
If the user wishes to always use a scaling or fixed isolation cut, independent of the value of {\tt epsilon\_h}, the routine
{\tt src/User/iso.f} may be edited and the value of the variable {\tt imode} changed according to the comments.
When {\tt frag} is set to {\tt .false.}, $\epsilon_h$ controls the amount of hadronic energy allowed inside the cone using the
Frixione isolation prescription (see above, Eq.~(\ref{frixeq})). 

\item {\tt n\_pow}. When using the Frixione isolation prescription, the exponent $n$ in Eq.~(\ref{frixeq}).

%These constitute a photon isolation
%cut which ensures that the amount of hadronic
%transverse momentum in a cone around each photon is less than
%a specified fraction of the photon's $p_T$.
%\begin{displaymath}
%\sum_{R < R_0} p_T^{\rm hadronic} < f \times p_T^{photon},
%\end{displaymath}
%where $R_0$ and $f$ are specified by {\tt cone\_photon} and
%{\tt cone\_ptcut} respectively.

\begin{center}
\{blank line\} \\
{\tt [Anomalous couplings of the W and Z] }
\end{center}

\item {\tt Delta\_g1(Z)}. {\it See section~\ref{subsec:diboson}.}
\item {\tt Delta\_K(Z)}. {\it See section~\ref{subsec:diboson}.}
\item {\tt Delta\_K(gamma)}. {\it See sections~\ref{subsec:diboson} and~\ref{subsec:wgamma}.}
\item {\tt Lambda(Z)}. {\it See section~\ref{subsec:diboson}.}
\item {\tt Lambda(gamma)}. {\it See sections~\ref{subsec:diboson} and~\ref{subsec:wgamma}.}
\item {\tt h1(Z)}. {\it See section~\ref{subsec:zgamma}.}
\item {\tt h1(gamma)}. {\it See section~\ref{subsec:zgamma}.}
\item {\tt h2(Z)}. {\it See section~\ref{subsec:zgamma}.}
\item {\tt h2(gamma)}. {\it See section~\ref{subsec:zgamma}.}
\item {\tt h3(Z)}. {\it See section~\ref{subsec:zgamma}.}
\item {\tt h3(gamma)}. {\it See section~\ref{subsec:zgamma}.}
\item {\tt h4(Z)}. {\it See section~\ref{subsec:zgamma}.}
\item {\tt h4(gamma)}. {\it See section~\ref{subsec:zgamma}.}
\item {\tt Form-factor scale, in TeV}. {\it See section~\ref{subsec:diboson}.} \\
No form-factors are applied to the anomalous couplings if this value is negative.

\begin{center}
\{blank line\} \\
{\tt [Anomalous width of the Higgs] }
\end{center}

\item {\tt Gamma\_H/Gamma\_H(SM)}. For processes {\tt 123}--{\tt 126}, {\tt 128}--{\tt 133} only,
this variable provides a rescaling of the width of the Higgs boson.  Couplings are rescaled such that the
corresponding cross section close to the Higgs boson peak is unchanged.  Further details of this procedure are given in
{\tt arXiv:1311.3589}.

\begin{center}
\{blank line\} \\
{\tt [How to resume/save a run] }
\end{center}

\item {\tt readin}. If {\tt .true.}, the program will read in a
previously saved {\tt VEGAS} grid from the file specified by
{\tt ingridfile.grid}. Note that this, and the following 3 options,
have no effect if {\tt part} is set to {\tt tota} (in this case, grids
are automatically saved and loaded as part of the calculation).

\item {\tt writeout}. If {\tt .true.}, the program will write out
the {\tt VEGAS} grid at the end of the run, to the file specified by
{\tt outgridfile.grid}.

\item {\tt ingridfile}.  {\it See above.}

\item {\tt outgridfile}.  {\it See above.}

\end{itemize}

The final section of the input file contains settings for various technical parameters that should not normally need to be changed. Prior to version 5.5, these were set in {\tt technical.DAT}. For backwards compatibility they may still be specified in that file too, although they will be over-ridden by any settings here.
\begin{center}
\{blank line\} \\
{\tt [Technical parameters that should not normally be changed]}
\end{center}
\begin{itemize}
\item {\tt debug}.  
A logical variable which can be used during a 
debugging phase to mandate special behaviours. 
Passed by common block \\
{\tt common/debug/debug}.
\item {\tt verbose}.  
A logical variable which can be used during a debugging phase to write 
special information. Passed in common block \\
{\tt common/verbose/verbose}.
\item {\tt new\_pspace}.  
A logical variable which can be used during a debugging phase to test alternative versions of the phase space.
Passed in common block {\tt common/new\_pspace/new\_pspace}.
\item {\tt virtonly}.  
A logical variable. The default value for this variable is false.
If virtonly is set to true, during the running of the real part,
the effect of real radiation is neglected, and only the effect of the
integrated dipoles is retained. 
\item {\tt realonly}.  
A logical variable. The default value for this variable is false.
If realonly is set to true, during the running of the real part,
the effect of integrated dipoles is neglected, and only the effect of the
real radiation is retained. 
\item {\tt spira}.  
A logical variable. If {\tt spira} is true, we calculate the 
width of the Higgs boson by interpolating from a table
calculated using the NLO code of M. Spira.
Otherwise the LO value valid for low Higgs masses only is used.
\item {\tt noglue}.  
A logical variable. 
The default value is false. If set to true, no processes
involving initial gluons are included.
\item {\tt ggonly}.  
A logical variable. 
The default value is false. If set to true, 
only the processes
involving initial gluons in both hadrons are included.
\item {\tt gqonly}.  
The default value is false. If set to true, 
only the processes
involving an initial gluon in one hadron and an initial quark
or antiquark in the other hadron (or vice versa) are included.
\item {\tt omitgg}.  
A logical variable. 
The default value is false. If set to true, the gluon-gluon
initial state is not included.
\item {\tt vanillafiles}.  
A logical variable. The default value is false. If set to true, 
the output files have the generic names 
{\tt mcfm-output.top} and {\tt mcfm-output.dat}. In addition
the path to the parton distribution files is truncated so that
they are expected to be found in the same directory as the executable
{\tt mcfm}.
% \item {\tt nmin}
% A technical parameter used in alternative phase space generating routines.
% \item {\tt nmax}
% A technical parameter used in alternative phase space generating routines.
\item {\tt clustering}
This logical parameter determines whether clustering is performed to yield
jets. Only during a debugging phase should this variable be set to false. 
\item {\tt realwt}.  
This is a logical parameter that in general should be set to false.
If set to true, mcfm samples the integral according to the
unsubtracted real emission weight.
\item {\tt colourchoice}.  
If colourchoice=0, all colour structure are included ($W,Z+2$~jets).
If colourchoice=1, only the leading 
colour structure is included ($W,Z+2$~jets).
\item {\tt rtsmin}.  
A minimum value of $\sqrt{s_{12}}$, which ensures that the invariant mass
of the incoming partons can never be less than {\tt rtsmin}.
\item {\tt cutoff}.  
A minimum value of $s_{ij}$, which ensures that the invariant mass squared
of any pair of partons can never be less than {\tt cutoff}.
\item {\tt aii}.  
A double precision variable which can be used to
limit the kinematic range for the subtraction of initial-initial dipoles
as suggested by Trocsanyi and Nagy~\cite{Nagy:2003tz}.   
The value {\tt aii=1d0} corresponds 
to standard Catani-Seymour subtraction.
\item {\tt aif}.  
A double precision variable which can be used to
limit the kinematic range for the subtraction of initial-final dipoles
as suggested by Trocsanyi and Nagy~\cite{Nagy:2003tz}.   
The value {\tt afi=1d0} corresponds 
to standard Catani-Seymour subtraction.
\item {\tt afi}.  
A double precision variable which can be used to
limit the kinematic range for the subtraction of final-initial dipoles
as suggested by Trocsanyi and Nagy~\cite{Nagy:2003tz}.   
The value {\tt afi=1d0} corresponds 
to standard Catani-Seymour subtraction.
\item {\tt aff}. 
A double precision variable which can be used to
limit the kinematic range for the subtraction of final-final dipoles
as suggested by Trocsanyi and Nagy~\cite{Nagy:2003tz}.   
The value {\tt aff=1d0} corresponds 
to standard Catani-Seymour subtraction.
\item {\tt bfi}. 
A double precision variable which can be used to
limit the kinematic range for the subtraction of final-initial dipoles
in the photon fragmentation case. In this version it is not yet operative.
\item {\tt bff}. 
A double precision variable which can be used to
limit the kinematic range for the subtraction of final-final dipoles
in the photon fragmentation case. In this version it is not yet operative.
\end{itemize}

\section{Nuclear collisions}
\label{sec:nucleus}

It is possible to specify nuclear collisions by choosing values
of {\tt ih1} and/or {\tt ih2} above {\tt 1000d0}. In that case,
the identity of the nucleus is specified by the atomic number
and mass ($Z$ and $A$ respectively) as follows:
\begin{equation}
{\tt ih} = 1000Z+A.
\end{equation}
For example, to choose an incoming lead beam one would set
{\tt ih1=+82207d0}, corresponding to $Z=82$ and $A=207$.
When running the program, the value of {\tt sqrts} should also be
changed. This must be done by hand and is not automatically taken
care of by the
program. The centre-of-mass energy is decreased by a factor of
$\sqrt{Z/A}$ for each nuclear beam. 

The nucleon PDF's are calculated by applying the correction
factors of EKS98~\cite{Eskola:1998df} on top of the PDF set that is selected.
This construction simply corrects each parton distribution by
a factor that depends on the value of $(x,\mu)$ in the event.
This parametrization is limited to the region $\mu < 100$~GeV and
any value above that threshold will instead default to $100$~GeV.

Note that the cross-section reported by the program at the end
of the run is given per nucleon per beam. Therefore the
appropriate factors of $A$ should be applied in order to obtain
the total cross section.

\section{Output}
\label{sec:output}

In addition to the direct output of the program to {\tt stdout}, after
the final sweep of {\tt VEGAS} the program can output additional files
as specified below.
If a working directory was specified in the command line, then these
output files will be written to that directory.

The standard output will detail the iteration-by-iteration best estimate
of the total cross-section, together with the accompanying error estimate.
After all sweeps have been completed, a final summary line will be printed.
In the {\tt npart}~$=$~{\tt tota} case, this last line will actually be the
sum of the two separate real and virtual integrations.
If the {\tt LHAPDF} package is being used and the value of
{\tt LHAPDF set} is equal to {\tt -1},  to indicate a calculation using
PDF uncertainties, then the computed PDF uncertainty is
supplied in the regular output and also echoed to the file,
{\tt pdfuncertainty.res}. The appropriate method for computing the PDF
uncertainties is chosen according to the PDF set that is being used.
For NNPDF sets the uncertainty is computed according to the so-called
MC prescription, described in detail in Appendix B of
Ref.~\cite{Ball:2008by}~\footnote{
The authors thank M. Ubiali and collaborators for providing their
implementation of the MC method within the \MCFM framework.}.
For the sets of Alekhin et al. the uncertainty is computed using the
symmetric Hessian method.  For all other sets -- those from CTEQ and
MSTW -- the asymmetric Hessian uncertainties are computed using
the formula given explicitly in
Eqn. (43) of Ref.~\cite{Campbell:2006wx}.

Other output files may be produced containing various histograms associated
with the calculated process. The write-out of the different output files
is controlled by logical variables at the top of the input file. The various options are:
\begin{itemize}
\item {\tt writetop}:  write out the histograms as a {\tt TOPDRAWER} file,
{\tt outputname.top}.
\item {\tt writedat}:  write out the histograms in a raw format 
which may be read in by a plotting package of the user's choosing,
{\tt outputname.dat}.
\item {\tt writegnu}:  write out the histograms as a {\tt gnuplot} file\footnote{
For information on obtaining and using {\tt gnuplot}, visit http://www.gnuplot.info/.},
{\tt outputname.gnu}.  This can be processed by running the command
`{\tt gnuplot outputname.gnu}', producing a postscript version of the histograms
in  {\tt outputname.ps}.
\item {\tt writeroot}:  write out a script, {\tt outputname.C}, that
can be executed by ROOT. Opening ROOT and running `{\tt .x outputname.C}'
produces histograms in the file
{\tt outputname.root}. The histograms can be subsequently inspected or manipulated
as usual, e.g. by opening the graphical browser using `{\tt TBrowser b;}'.
\end{itemize}

All of the output files include a summary of the options file ({\tt input.DAT}) in the form of
comments at the beginning. The structure
of {\tt outputname} is as follows:
\begin{displaymath}
{\tt procname\_part\_pdlabel\_scale\_facscale\_runstring}
\end{displaymath}
where {\tt procname} is a label assigned by the program corresponding to
the calculated process; the remaining labels are as input by the user
in the file {\tt input.DAT}.

The histograms are filled via the file {\tt src/User/G.f}. 
For the convenience of the user a dummy routine {\tt src/User/userplotter.f}
has been provided. The user may substitute their own routine to do the plotting,
by writing and compiling a routine of this name. See section \ref{user}.
 
For some processes
a specific routine has been written to plot relevant kinematic quantities. In that case
a further routine is called, e.g. {\tt nplotter\_W\_only.f} for inclusive $W$ production.
In all other cases the filling of the histograms is performed by a routine in
{\tt src/User/nplotter\_auto.f}. The arguments of the process-specific plotting subroutines
are { \tt p,wt,wt2,switch}. {\tt p} contains the momenta of all the particles
(i.e. the four momenta of the leptons and jets). The order of the jets is not
necessarily the  order specified in process.DAT.  However in the case that we have a 
$b$-quark or antiquark they are labelled by {\tt bq} and {\tt ba} respectively
in the array jetlabel. {\tt wt} ({\tt wt2}) is the weight of the event (squared).  

\subsection{Histograms}
\label{sec:histos}

Extra histograms may be added to the plotting files in
a fairly straightforward manner. Each histogram is filled by making
a call to the routine {\tt bookplot} and updating the histogram
counter {\tt n} by 1. For example, the pseudorapidity of particle $3$
may be plotted using the following code fragment:

\begin{verbatim}
  eta3=etarap(3,p)
  call bookplot(n,tag,'eta3',eta3,wt,wt2,-4d0,4d0,0.1d0,'lin')
  n=n+1
\end{verbatim}
The first two arguments of the call should not be changed. The third
argument is a string which is used as the title of the plot in the
output files. The fourth argument carries the variable to
be plotted, which has been previously calculated. The arguments {\tt
wt} and {\tt wt2} contain information about the phase-space weight and
should not be changed. The
last arguments tell the histogramming routine to use bins of size {\tt
0.1} which run from {\tt -4} to {\tt 4}, and use a linear scale for
the plot. A logarithmic scale may be used by changing the final
argument to {\tt 'log'}.

If the {\tt LHAPDF} package is being used and the value of
{\tt LHAPDF set} is equal to {\tt -1},  to indicate a calculation using
PDF uncertainties, then errors on distributions may also be accumulated.
Note that, due to limitations within the LHAPDF distribution, calculations
using error PDF sets are impractical unless the grid versions of the sets
are used. The grid versions are available in LHAPDF v.3 onwards and may be identified
by the {\tt .LHgrid} extension in the {\tt PDFsets} directory. To use the
grid version, simply pass the PDF set name, including this extension, as
the value of {\tt LHAPDF group} in the input file.
 
To accumulate errors in distributions, add an extra
line to {\tt nplotter.f} after the
call to {\tt bookplot} but before the counter is incremented. For
example, to calculate the PDF uncertainties on the distribution
of {\tt eta3} one would simply add:
\begin{verbatim}
   call ebookplot(n,tag,eta3,wt)
\end{verbatim}
The third argument contains the variable to plot and the other entries
should not be changed. The other parameters for the plot are exactly
those specified on the previous line, in the call to {\tt bookplot}.
Since each PDF error distribution takes up quite a lot of memory
during execution, there is a limit of 4 on the number of distributions
with errors that can be calculated at one time. When calculating
PDF uncertainties on distributions, the program will produce an
additional file which contains the results for each PDF error set
individually. In addition, the main file will include the uncertainty
limits on the distribution, obtained using the appropriate PDF uncertainty
prescription as described above, on a bin-by-bin basis. Thus
the resulting error limits are not simply described by a single PDF
set.

\subsection{Other output modes}
\label{subsec:otheroutput}
As noted in the description of the input file, there are a number of other
output modes which may be useful in certain situations. In particular, the
ability to output n-tuples can be used to generate a large event record that 
can be subsequently analyzed according to the user's needs. Much of the code
for generating these outputs can be found in {\tt src/User/dswhbook.f}; some
additional work may be required, depending on the process under study.

The simplest alternative output mode is obtained by changing the flag
{\tt dswhisto} to {\tt .true.} . In this way, the {\tt TOPDRAWER} output file is
replaced by the file {\tt outputname.rz} which contains the histograms in {\tt HBOOK}
format.

\subsubsection{Simple n-tuple output}
To obtain the simplest n-tuple output, the flag {\tt creatent} should be set
to {\tt .true.} and the parameter {\tt NTUPLES} in the {\tt Makefile} should be changed
to either {\tt YES} or {\tt FROOT}. When changed to {\tt YES}, each event that enters a histogram
is also recorded as an n-tuple in the file {\tt outputname\_batchno.rz}. The {\tt batchno}
starts at zero and is incremented by one every one million events. Each event
is a simple row-wise n-tuple consisting of the 4-momenta of each of the final
state particles ($p_x$, $p_y$, $p_z$ and $E$, in that order) followed by 5 numbers
representing the event weight. The first number represents the total event weight
and the others, the contribution from gluon-gluon, quark-gluon (and antiquark-gluon),
quark-quark (and antiquark-antiquark) and quark-antiquark initial states. If
PDF uncertainties are being calculated (using {\tt LHAPDF}) then the total event weight
corresponding to each of the additional PDF sets is also written out at this stage.
Single precision is used, for economy. A simple way to analyze 
these n-tuples is to use the {\tt h2root} command and then perform
manipulations with the ROOT package. Note that these n-tuples contain
no information about either the flavour or the colour of the initial or final
state particles. Summation and averaging over these variables has already been
performed. Furthermore, the `events' are {\it weighted} - so they are not events
in the traditional event generator sense.

\subsubsection{n-tuples using FROOT}
Output is similar when using the {\tt FROOT} option. In this case, the program will directly
fill a ROOT n-tuple using the FROOT interface of P. Nadolsky ({\tt nadolsky@pa.msu.edu}, a version
of which is included with \MCFM (in the directory {\tt src/User/froot.c}). The structure
of the ntuples is slightly different to that above, with entries:
\begin{itemize}
\item {\tt E\_i}, {\tt pxi}, {\tt pyi}, {\tt pzi} to specify the particle momenta, with {\tt i}
 looping over all members of the final state.
\item {\tt wt\_ALL}, {\tt wt\_gg}, {\tt wt\_gq}, {\tt wt\_qq}, {\tt wt\_qqb} for the total event
 weight and the weights in each of the parton-parton subchannels.
\item {\tt PDFjj}, where {\tt jj} loops over all the PDF uncertainty sets (only written if
 appropriate).
\end{itemize}

\subsubsection{Unweighted events}

In order to obtain {\it unweighted} events, one must change the value of
{\tt nevtrequested} to an integer greater than zero, corresponding to the
number of unweighted events that is required. This option is only available at lowest order
at present and only for a limited number of processes.
In this mode the program will first perform a run to obtain the
maximum weight and then perform a simple unweighting procedure against this
number.   As a result this procedure is rather inefficient.
Identities are assigned to the
partons in the initial state according to the relative parton-parton
luminosities.  If the warm-up stage of the calculation is not sufficiently long, it 
is possible to find events with a weight greater than the maximum in the unweighting
phase.  In this case a warning message is written to the screen and the events are
not truly unweighted without further processing by the user.

The events are written to a file with the extension {\tt .lhe} using the LHE format.
The routines that handle most of the processing can be found in {\tt src/Need/mcfm\_writelhe.f}.
At present this feature is still under development. 

\section{Other user options}
\label{user}
There are a number of other user options which are included in the file 
{\tt src/User/usercode\_f77.f}
\begin{itemize}
\item {\tt logical function userincludedipole(nd, ppart, mcfm\_result)}

   Variables passed are                                         
\begin{itemize}                                               
\item  nd:           index of the dipole
\item ppart:        momenta of incoming and outgoing particles
\item mcfm\_result:  the decision that was taken by mcfm about whether to keep this event
\end{itemize}                                                                                                                                   

The program works in such a way that the cuts are applied after the
phase space point has been generated but before the matrix element has
been calculated. This true for the real, virtual lord and real pieces
including the contributions from the subtraction dipoles. This
operation is performed by the \MCFM logical function {\tt mcfmincludedipole}.  
However the (currently dummy) logical function {\tt userincludedipole} 
is also  always called.  This allows the user to
veto events that might otherwise pass the \MCFM cuts.  It can be used,
e.g. to force \MCFM to generate only events that are above some large
HT threshold, which comes in useful when trying to get precision on
the tails of some distributions.
                                                           
\item {\tt subroutine userplotter(pjet, wt,wt2, nd)}
This subroutine that is called to allow the user to bin their own       
histograms.                                                              
                                                                                                                                   
Variables passed to this routine:                                                                                               
                                                                                                                                   
\begin{itemize}
\item        p:  4-momenta of incoming partons(i=1,2), outgoing leptons and                                                             
            jets in the format p(i,4) with the particles numbered                                                                  
            according to the input file and components labelled by                                                                 
            (px,py,pz,E).  
                                                                                                                                   
\item        wt:  weight of this event                                                                                                   
                                                                                                                                   
\item       wt2:  weight$^2$ of this event                                                                                                 
                                                                                                                                   
\item        nd:  an integer specifying the dipole number of this contribution                                                           
            (if applicable), otherwise equal to zero.
\end{itemize}

\item {\tt subroutine userwriteinfo(unitno,comment\_string,xsec,xsec\_err,itno)}
This (currently dummy) subroutine that gets called after \MCFM has written its comments to
one of the output files. It allows the user to write their own comments                                                         
to that same file.
                                                                                                                                   
   Variables passed to this routine:                                                                                               
                                                                                                                                   
\begin{itemize}
\item  unitno: the unit number to which output is being sent                                                                         
\item comment\_string: a comment character that precedes each line of output                                                         
\item xsec, xsec: the cross section and its error (in case you care!)                                                               
\item itno: the iteration number (0 at the end of the last iteration)                                                               
\end{itemize}
\end{itemize}

A similar set of routines written in fortran90 by Gavin Salam are in the file {\tt  usercode.f90}.
Various C routines that may be of use are included in {\tt cxxusercode.cc}.

\section{Notes on specific processes}

\label{sec:specific}

Note that, as of version 4.0, the version of each process described in the file{\tt process.DAT} includes 
all appropriate boson decays. This is the calculation
that is performed when the parameter {\tt removebr} is set to {\tt .false.} ,
as indicated above.

In many cases a more simple calculation can be performed by setting this
parameter to {\tt .true.}, in which case these decays are not performed.
Technically the full calculation including the decays
is still performed but cuts are not performed on the decay products and the
branching ratio is divided out, thus yielding the cross section before decay.
In the notes below we indicate the simpler processes thus obtained. When running in
this mode, the parameter {\tt zerowidth} should be set to {\tt .true.} also,
for consistency. However in certain circumstances, for the sake of comparison,
it may be useful to run with it set to {\tt .false.} .

\subsection{$W$-boson production, processes 1,6}
\label{subsec:wboson}

These processes represent the production of a $W$ boson which subsequently
decays leptonically. The calculation may be performed at NLO.

When {\tt removebr} is true, the $W$ boson does not decay.

\subsection{$W+$~jet production, processes 11,16}
\label{subsec:w1jet}

These processes represent the production of a $W$ boson which subsequently
decays leptonically, in association with a single jet.
The calculation may be performed at NLO.

When {\tt removebr} is true, the $W$ boson does not decay.

\subsection{$W+b$ production, processes 12,17}
\label{subsec:wb}

These processes represent the production of a $W$ boson which
subsequently decays leptonically, in association with a single bottom
quark, exploiting the weak transitions $c \to b$ and $u \to b$.
This is produced at leading order by an initial state which
contains a charm quark (or the CKM  suppressed $u$ quark) and a
gluon.  The effect of the bottom quark mass is included throughout the
calculation.  
For this case the CKM matrix elements $V_{cb}$ and $V_{ub}$,
(if they are equal to zero in the input data file, {\tt mdata.f})
are set equal to $0.041$ and $0.00347$ respectively. 
Otherwise the non-zero values specified in {\tt mdata.f} are used. 
The calculation of this process may
be performed at NLO.

When {\tt removebr} is true, the $W$ boson does not decay.

\subsection{$W+c$ production, processes 13,18}
\label{subsec:wc}

These processes represent the production of a $W$ boson which
subsequently decays leptonically, in association with a charm
quark. This is produced at leading order by an initial state which
contains a strange quark (or Cabibbo suppressed $d$ quark) and a
gluon.  The effect of the charm quark mass is included throughout the
calculation.  As of version 5.2, the calculation of this process may
be performed at NLO.

When {\tt removebr} is true, the $W$ boson does not decay.

\subsection{$W+c$ production ($m_c=0$), processes 14,19}
\label{subsec:wcmassless}

These processes are identical to {\tt 13} and {\tt 18} except for the fact
that the charm quark mass is neglected. The calculation can currently be
performed at LO only.

\subsection{$W+b{\bar b}$ production, processes 20,25}
\label{subsec:wbb}

These processes represent the production of a $W$ boson which subsequently
decays leptonically, in association with a $b{\bar b}$ pair. The effect of
the bottom quark mass is included throughout the calculation.  
Beginning with \MCFM version 6.0 this calculation may be performed at NLO, thanks to
the incorporation of the virtual corrections from ref.~\cite{Badger:2010mg}.
When {\tt removebr} is true, the $W$ boson does not decay.

To select final states in which one of the $b$-quarks may be unobserved the
user can employ processes 401--408 instead (see section~\ref{subsec:wbbfilter}).
These processes use the same matrix
elements but make specific requirements on the kinematics of the $b$-quarks
and QCD radiation. 

\subsection{$W+b{\bar b}$ production ($m_b=0$), processes 21,26}
\label{subsec:wbbmassless}

These processes are identical to {\tt 20} and {\tt 25} except for the fact
that the bottom quark mass is neglected. This allows the calculation to be
performed up to NLO, with currently calculated virtual matrix elements. These 
processes run considerably faster than the corresponding processes with the mass
for the $b$ quark, (20,25). In circumstances where both $b$ quarks are at large 
transverse momentum, the inclusion of the mass for the $b$-quark is not mandatory
and a good estimate of the cross section may be obtained by using these processes.

When {\tt removebr} is true, the $W$ boson does not decay.

\subsection{$W+2$~jets production, processes 22,27}
\label{subsec:w2jets}

\begin{center}
[{\it For more details on this calculation, please refer to \break
 hep-ph/0202176 and hep-ph/0308195}]
\end{center}
This process represents the production of a $W$ boson and $2$ jets,
where the $W$ boson decays leptonically. The calculation may be
performed up to NLO, as detailed below. Virtual amplitudes are
taken from ref.~\cite{Bern:1997sc}.

For these processes (and also for $Z+2$~jet production, {\tt nproc=44,46})
the next-to-leading order matrix elements are
particularly complex and so they have been divided into two groups.
The division is according to the lowest order diagrams from which they
originate:
\begin{enumerate}
\item Diagrams involving two external quark lines and two external gluons,
the ``{\tt Gflag}'' contribution. The real diagrams in this case thus
involve three external gluons.

\item Diagrams where all four external lines are quarks,
the ``{\tt Qflag}'' contribution. The real diagrams in this case 
involve only one gluon.
\end{enumerate}

By specifying {\tt Gflag} and {\tt Qflag} in {\tt input.DAT} one may
select one of these options at a time. The full result may be obtained
by straightforward addition of the two individual pieces, with no
meaning attached to either piece separately. 
Both of these may be set to {\tt .true.} simultaneously, however this
may result in lengthy run-times for sufficient convergence of the integral.

When {\tt removebr} is true, the $W$ boson does not decay.

\subsection{$W+3$~jets production, processes 23,28}
\label{subsec:w3jets}

This process represents the production of a $W$ boson and $3$ jets,
where the $W$ boson decays leptonically. The calculation may be
performed at LO only.

When {\tt removebr} is true, the $W$ boson does not decay.

\subsection{$W+b{\bar b}+$~jet production ($m_b=0$), processes 24,29}
\label{subsec:wbbjetmassless}

These processes represent the production of a $W$ boson which subsequently
decays leptonically, in association with a $b{\bar b}$ pair and an
additional jet. The effect of the bottom quark mass is neglected throughout
and the calculation may be performed at LO only.

When {\tt removebr} is true, the $W$ boson does not decay.

\subsection{$Z$-boson production, processes 31--33}
\label{subsec:zboson}

These processes represent the production of a $Z$ boson which subsequently
decays either into electrons ({\tt nproc=31}), neutrinos ({\tt nproc=32})
or bottom quarks ({\tt nproc=33}). Where appropriate, the effect of a virtual
photon is also included. As noted above, in these latter cases {\tt m34min > 0}
is obligatory. The calculation may be performed at NLO,
although the NLO calculation of process {\tt 33} does not include radiation
from the bottom quarks (i.e.\ radiation occurs in the initial state only).

When {\tt removebr} is true in process {\tt 31}, the $Z$ boson does not decay.

\subsection{$Z$-boson production decaying to jets, processes 34--35}
Radiation from the final state quarks is not included in this process.

\subsection{$t \bar{t}$ production mediated by $Z/\gamma^*$-boson exchange, process 36}

These processes represent the production of a virtual $Z$ boson or photon 
which subsequently decays into $t \bar{t}$.
The leptonic decays of the top quarks are included.
Switching {\tt zerowidth} from {\tt .true.} to {\tt .false.} only affects
the $W$ bosons from the top quark decay.
Note that {\tt m34min > 0} is obligatory due to the inclusion of the
virtual photon diagrams. The calculation may be only be performed at LO.


\subsection{$Z+$~jet production, processes 41--43}
\label{subsec:zjet}

These processes represent the production of a $Z$ boson and a single jet,
where the $Z$ subsequently
decays either into electrons ({\tt nproc=41}), neutrinos ({\tt nproc=42})
or bottom quarks ({\tt nproc=43}). Where appropriate, the effect of a virtual
photon is also included. The calculation may be performed at NLO,
although the NLO calculation of process {\tt 43} does not include radiation
from the bottom quarks.

When {\tt removebr} is true in process {\tt 41}, the $Z$ boson does not decay.

\subsection{$Z+2$~jets production, processes 44, 46}
\label{subsec:z2jets}

\begin{center}
[{\it For more details on this calculation, please refer to \break
 hep-ph/0202176 and hep-ph/0308195}]
\end{center}

These processes represents the production of a $Z$ boson and $2$ jets,
including also the effect of a virtual photon ({\tt nproc=44} only). The $Z/\gamma^*$ decays
to an $e^+ e^-$ pair ({\tt nproc=44}) or into three species of neutrino ({\tt nproc=46}).
The calculation may be performed up to NLO --
please see the earlier Section~\ref{subsec:w2jets} for more details,
especially the discussion regarding {\tt Qflag} and {\tt Gflag}.
As of version 6.0, both of these may be set to {\tt .true.} simultaneously but this
may result in lengthy run-times for sufficient convergence of the integral.
Virtual amplitudes are taken from ref.~\cite{Bern:1997sc}.

When {\tt removebr} is true, the $Z$ boson does not decay.


\subsection{$Z+3$~jets production, processes 45, 47}
\label{subsec:z3jets}

These processes represent the production of a $Z$ boson and $3$ jets,
including also the effect of a virtual photon ({\tt nproc=45} only). The $Z/\gamma^*$ decays
to an $e^+ e^-$ pair ({\tt nproc=45}) or into three species of neutrino ({\tt nproc=47}).
The calculation may be performed at LO only.

When {\tt removebr} is true, the $Z$ boson does not decay.

\subsection{$Z+b{\bar b}$ production, process 50}
\label{subsec:zbb}

These processes represent the production of a $Z$ boson (or virtual photon)
which subsequently decays leptonically, in association
with a $b{\bar b}$ pair. The effect of
the bottom quark mass is included throughout the calculation.  
The calculation may be performed at LO only.

When {\tt removebr} is true, the $Z$ boson does not decay.

\subsection{$Z+b{\bar b}$ production ($m_b=0$), processes 51--53}
\label{subsec:zbbmassless}

Process {\tt 51} is identical to {\tt 50} except for the fact
that the bottom quark mass is neglected. This allows the calculation to be
performed up to NLO. The other processes account for the decays into
neutrinos ({\tt nproc=52}) and bottom quarks ({\tt nproc=53}). Note that
the NLO calculation of process {\tt 53} does not currently 
include radiation from the
bottom quarks produced in the decay.

When {\tt removebr} is true in process {\tt 51}, the $Z$ boson does not decay.

\subsection{$Z+b{\bar b}+$~jet production ($m_b=0$), process 54}
\label{subsec:zbbjetmassless}

This process represents the production of a $Z$ boson (and virtual photon)
which subsequently decays leptonically, in association
with a $b{\bar b}$ pair and an additional jet.
The effect of the bottom quark mass is neglected throughout
and the calculation may be performed at LO only.

When {\tt removebr} is true, the $Z$ boson does not decay.

\subsection{$Z+c{\bar c}$ production ($m_c=0$), process 56}
\label{subsec:zccmassless}

Process {\tt 56} is the equivalent of {\tt 51}, with the bottom quarks
replaced by charm. Although the charm mass is neglected, the calculation
contains diagrams with two gluons in the initial state and a
$Z$ coupling to the heavy quark line -- hence the dependence upon the quark
flavour.

When {\tt removebr} is true in process {\tt 56}, the $Z$ boson does not decay.

\subsection{Di-boson production, processes 61--89}
\label{subsec:diboson}

\begin{center}
[{\it For more details on these calculations, please refer to hep-ph/9905386
and arXiv:1105.0020 [hep-ph]}]
\end{center}

These processes represent the production of a diboson pair $V_1 V_2$,
where $V_1$ and $V_2$ may be either a $W$ or $Z/\gamma^*$. 
All the processes in this section may be calculated at NLO, with the exception
of {\tt nproc=66,69}. There are various
possibilities for the subsequent decay of the bosons, as specified in the
sections below. Amplitudes are taken from ref.~\cite{Dixon:1998py}.
Where appropriate, these processes include glue-glue initiated box diagrams
which first contribute at order $\alpha_s^2$ but are included here in the
NLO calculation. We also include singly resonant diagrams at NLO for all processes
in the case {\tt zerowidth = .false.}.

For processes {\tt 62}, {\tt 63}, {\tt 64}, {\tt 65}, {\tt 74}
and {\tt 75} the default behaviour is that the hadronic decay products
of the bosons are clustered into jets using the supplied jet
algorithm parameters, but no cut is applied on the number of jets.
This behaviour can be altered by changing the value of the
variable {\tt notag} in the file {\tt src/User/setnotag.f}.
 
\subsubsection{$WW$ production, processes 61-64, 69}

For $WW$ production, both $W$'s can decay leptonically ({\tt nproc=61}) or one
may decay hadronically ({\tt nproc=62} for $W^-$ and {\tt nproc=64} for $W^+$).
Corresponding to processes {\tt 62,64}, processes {\tt 63,65} implement radiation in 
decay from the hadronically decaying W's.
Process {\tt 69} implements the matrix elements for the leptonic decay of
both $W$'s but where no polarization information is retained. It is included
for the sake of comparison with other calculations.
Processes {\tt 62} and {\tt 64} may be run at NLO with the option {\tt todk},
including radiation in the decay of the hadronically decaying $W$.
Processes {\tt 63} and {\tt 65} give the effect of radiation in the decay alone
by making the choices {\tt virt},  {\tt real} or {\tt tota}.

Note that, in processes
{\tt 62} and {\tt 64}, the NLO corrections include radiation from the
hadronic decays of the $W$.

The NLO calculations include contributions from the process $gg \to WW$
that proceeds through quark loops. The calculation of loops containing the third quark generation
includes the effect of the top quark mass (but $m_b=0$), while the first two
generations are considered massless. For numerical stability, a small cut on the
transverse momentum of the $W$ bosons is applied: $p_T(W)>0.05$~GeV for loops
containing massless (first or second generation) quarks, $p_T(W)>2$~GeV for $(t,b)$
loops. This typically removes less than $0.1$\% of the total cross section. The
values of these cutoffs can be changed by editing ${\tt src/WW/gg\_WW.f}$ and recompiling.

When {\tt removebr} is true in processes {\tt 61} and {\tt 69},
the $W$ bosons do not decay.

\subsubsection{$WW$+jet production, process 66}

This process is only implemented for the leptonic decay modes of both $W$
bosons and is currently limited to LO accuracy only. When {\tt removebr} is true,
the $W$ bosons do not decay.

\subsubsection{$WZ$ production, processes 71--80}

For $WZ$ production, the $W$ is chosen to decay leptonically. The $Z$ (or
virtual photon, when appropriate) may decay into electrons
({\tt nproc=71},{\tt 76}), neutrinos ({\tt nproc=72},{\tt 77}), a
pair of bottom quarks ({\tt nproc=73},{\tt 78}), three generations of down-type
quarks ({\tt nproc=74},{\tt 79}) or two generations of up-type quarks ({\tt nproc=75},{\tt 80}).
In process {\tt 78} the mass of the $b$-quark is neglected.
These processes will be observed
in the final state as $W$-boson + two or three jets.
In processes {\tt 72} and {\tt 77}, a sum is performed over all three species of neutrinos.

When {\tt removebr} is true in processes {\tt 71} and {\tt 76},
neither the $W$ or the $Z$ boson decays.

\subsubsection{$ZZ$ production, processes 81--84, 86--90}

For $ZZ$ production, there are two sets of processes corresponding to the
inclusion of a virtual photon when appropriate ({\tt nproc=81}--{\tt 84})
and the case where it is neglected ({\tt nproc=86}--{\tt 89}).
Thus {\tt nproc=86}--{\tt 89} are really for diagnostic purposes only.

The $Z$'s can either both decay leptonically ({\tt nproc=81},{\tt 86}),
one can decay leptonically while the other decays into neutrinos
({\tt nproc=82},{\tt 87}) or bottom quarks ({\tt nproc=83},{\tt 88}), or
one decays into neutrinos and the other into a bottom quark pair
({\tt nproc=84},{\tt 89}).
In process {\tt 83} the mass of the $b$-quark is neglected. Note that, in processes
{\tt 83}--{\tt 84} and {\tt 88}--{\tt 89}, the NLO corrections do not include
radiation from the bottom quarks that are produced by the $Z$ decay.

The NLO calculations include contributions from the process $gg \to ZZ$
that proceeds through quark loops. The calculation of loops containing the third quark generation
includes the effect of both the top and the bottom quark mass ($m_t,m_b \neq 0$), while the first two
generations are considered massless. For numerical stability, a small cut on the
transverse momentum of the $Z$ bosons is applied: $p_T(Z)>0.1$~GeV.
This typically removes less than $0.1$\% of the total cross section. The
values of these cutoffs can be changed by editing ${\tt src/ZZ/getggZZamps.f}$ 
and recompiling.

When {\tt removebr} is true in processes {\tt 81} and {\tt 86},
neither of the $Z$ bosons decays.

In process {\tt 90} the two $Z$ bosons decay to identical charged leptons,
and interference effects between the decay products of the two $Z$ bosons
are included. This process may be calculated at LO only.

\subsubsection{$ZZ$+jet production, process 85}

This process is only implemented for the case when one $Z$ boson decays to
electrons and the other to neutrinos (i.e. the companion of {\tt nproc=82}).
It may only be calculated at LO. When {\tt removebr} is true, the $Z$ bosons
do not decay.

\subsubsection{Anomalous couplings}

\label{sec:anomalous}
As of version 3.0, it is possible to specify anomalous trilinear
couplings for the $W^+W^-Z$ and $W^+W^-\gamma$ vertices that are
relevant for $WW$ and $WZ$ production. To run in this mode, one
must set {\tt zerowidth} equal to {\tt .true.}
and modify the appropriate lines for the couplings in {\tt input.DAT}
(see below). Note that, at present, the effect of anomalous couplings is not included
in the gluon-gluon initiated contributions to the $WW$ process.

The anomalous couplings appear in the Lagrangian,
${\cal L} = {\cal L}_{SM} + {\cal L}_{anom}$ as follows
(where ${\cal L}_{SM}$ represents the usual Standard Model Lagrangian and
${\cal L}_{anom}$ is taken from Ref.~\cite{Dixon:1999di}):
\begin{eqnarray}
{\cal L}_{anom} & = & i g_{WWZ} \Biggl[
 \Delta g_1^Z \left( W^*_{\mu\nu}W^\mu Z^\nu - W_{\mu\nu}W^{*\mu} Z^\nu \right)
+\Delta\kappa^Z W^*_\mu W_\nu Z^{\mu\nu} \nonumber \\
 & &+
 \frac{\lambda^Z}{M_W^2} W^*_{\rho\mu} W^\mu_\nu Z^{\nu\rho} \Biggr]
+i g_{WW\gamma} \Biggl[ 
 \Delta\kappa^\gamma W^*_\mu W_\nu \gamma^{\mu\nu}
+\frac{\lambda^\gamma}{M_W^2} W^*_{\rho\mu} W^\mu_\nu\gamma^{\nu\rho}
 \Biggr], \nonumber
\end{eqnarray}
where $X_{\mu\nu} \equiv \partial_\mu X_{\nu} - \partial_\nu X_{\mu}$
and the overall coupling factors are $g_{WW\gamma}=-e$,
$g_{WWZ}=-e\cot\theta_w$.
This is the most general Lagrangian that conserves $C$ and $P$
separately and electromagnetic gauge invariance requires that there
is no equivalent of the $\Delta g_1^Z$ term for the photon coupling.

In order to avoid a violation of unitarity, these couplings are often
included only after suppression by dipole form factors,
\begin{displaymath}
\Delta g_1^Z \rightarrow \frac{\Delta g_1^Z}{(1+\hat{s}/\Lambda^2)^2}, \qquad
\Delta \kappa^{Z/\gamma} \rightarrow
 \frac{\Delta \kappa_1^{Z/\gamma}}{(1+\hat{s}/\Lambda^2)^2}, \qquad
\lambda^{Z/\gamma} \rightarrow
 \frac{\Delta \lambda^{Z/\gamma}}{(1+\hat{s}/\Lambda^2)^2},
\end{displaymath}
where $\hat{s}$ is the vector boson pair invariant mass and $\Lambda$
is an additional parameter giving the scale of new physics, which should
be in the TeV range.
These form factors should be produced by the new physics associated with the
anomalous couplings and this choice is somewhat arbitrary. The use of the form
factors can be disabled as described below.

The file {\tt input.DAT} contains the values of the $6$ parameters
which specify the anomalous couplings:
\begin{verbatim}
    0.0d0           [Delta_g1(Z)]
    0.0d0           [Delta_K(Z)]
    0.0d0           [Delta_K(gamma)]
    0.0d0           [Lambda(Z)]
    0.0d0           [Lambda(gamma)]
    2.0d0           [Form-factor scale, in TeV]
\end{verbatim}
with the lines representing $\Delta g_1^Z$, $\Delta \kappa^Z$,
$\Delta \kappa^\gamma$, $\lambda^Z$, $\lambda^\gamma$ and
$\Lambda$~[TeV] respectively. By setting the first 5 parameters to zero,
as above, one recovers the Standard Model result.
If the input file contains a negative value for the form-factor scale
then the suppression factors described above are not applied.

\subsection{$WH$ production, processes 91-94, 96-99}
\label{subsec:wh}

These processes represent the production of a $W$ boson which subsequently
decays leptonically, in association with a Standard Model Higgs boson that
decays into a bottom quark pair ({\tt nproc=91, 96}), 
a pair of $W$ bosons ({\tt nproc=92, 97}), 
a pair of $Z$ bosons ({\tt nproc=93, 98}), or a pair of photons ({\tt nproc=94, 99}).  
Note that in the cases of Higgs decay to $W$,($Z$) pairs, 
below the $W$,($Z$) pair threshold
one of the $W$,($Z$) bosons is virtual 
and therefore one must set {\tt zerowidth=.false.}.
The calculation may be performed at NLO.
Note that the bottom quarks are considered massless and radiation from the
bottom quarks in the decay is not included.

When {\tt removebr} is true, neither the $W$ boson nor the Higgs decays.

\subsection{$ZH$ production, processes 101--109}
\label{subsec:zh}

These processes represent the production of a $Z$ boson (or virtual photon)
in association with a Standard Model Higgs boson that
decays into a bottom quark pair ({\tt nproc=101-103}),
or decays into a pair of photons ({\tt nproc=104-105}) 
or a pair of $W$ bosons ({\tt nproc=106-108}),
or a pair of $Z$ bosons ({\tt nproc=109}). 
The $Z$ subsequently decays into 
either an $e^+ e^-$ pair ({\tt nproc=101, 106, 109}), neutrinos ({\tt nproc=102, 107})
or a bottom quark pair ({\tt nproc=103, 108}).
The calculation may be performed at NLO, although radiation from the
bottom quarks in the decay of the Higgs (or the $Z$, for processes
{\tt 103, 108}) is not included.

When {\tt removebr} is true in processes {\tt 101, 106, 109}, neither the $Z$ boson
nor the Higgs decays.

\subsection{Higgs production, processes 111--121}
\label{subsec:h}

These processes represent the production of a Standard Model Higgs
boson that decays either into a bottom quark
pair ({\tt nproc=111}), a pair of tau's ({\tt nproc=112}), 
a $W^+W^-$ 
pair that further decays leptonically ({\tt nproc=113}) 
a $W^+W^-$ pair where the $W^-$ decays hadronically ({\tt nproc=114,115}) 
or a $ZZ$ pair ({\tt nproc=116-118}) . In addition, the loop-level decays of the Higgs 
into a pair of photons ({\tt nproc=119}) and the $Z\gamma$ decay are included
({\tt nproc=120,121}).

For the case of $W^+W^-$ process {\tt nproc=115} gives the contribution 
of radiation from the hadronically decaying $W^-$.
Process {\tt 114} may be run at NLO with the option {\tt todk},
including radiation in the decay of the hadronically decaying $W^-$.~\footnote{
We have not included the case of a hadronically decaying $W^+$; it can
be obtained from processes {\tt nproc=114,115} by performing the
substitutions $\nu \to e^-$ and $e^+ \to \bar{\nu}$.}
For the case of a $ZZ$ decay,
the subsequent decays can either be into a pair of muons and a pair of electrons
({\tt nproc=116)}, a pair of muons and neutrinos ({\tt nproc=117}) or
a pair of muons and a pair of bottom quarks ({\tt nproc=118}).

At LO the relevant diagram
is the coupling of two gluons to the Higgs via a top quark loop.
This calculation is performed in the limit of infinite top quark mass, so that 
the top quark loop is replaced by an effective operator. This corresponds
to the effective Lagrangian,
\begin{equation}
\mathcal{L} = \frac{1}{12\pi v} \, G^a_{\mu\nu} G^{\mu\nu}_a H \;,
\label{eq:HeffL}
\end{equation}
where $v$ is the Higgs vacuum expectation value and $G^a_{\mu\nu}$ the
gluon field strength tensor.
The calculation may be performed at NLO, although radiation from the
bottom quarks in the decay of processes {\tt 111} and {\tt 118} is not yet included.

At the end of the output the program will also display the cross section rescaled
by the constant factor,
\begin{equation}
\frac{\sigma_{\rm LO}(gg \to H, \mbox{finite}~m_t)}{\sigma_{\rm LO}(gg \to H, m_t \to \infty)} \;.
\label{eqn:hrescale}
\end{equation}
For the LO calculation this gives the exact result when retaining a finite value for $m_t$,
but this is only an approximation at NLO. The output histograms are not rescaled in this way.

When {\tt removebr} is true in processes {\tt 111,112,113,118},
the Higgs boson does not decay.

Process {\tt 119} implements the decay of the Higgs boson into two photons
via loops of top quarks and $W$-bosons.
The decay is implemented using the formula Eq.(11.12) from ref.~\cite{Ellis:1991qj}.
When {\tt removebr} is true in process {\tt 119} the Higgs boson does not decay.

Processes {\tt 120} and {\tt 121} implement the decay of the Higgs boson into an lepton-antilepton
pair and a photon. As usual the production of a charged lepton-antilepton pair is mediated by a 
$Z/\gamma^*$ (process {\tt 120}) and the production of three types of neutrinos 
$\sum  \nu \bar{\nu}$ by a $Z$-boson (process {\tt 121}). These processes are implemented 
using a generalization of the formula of \cite{Djouadi:1996yq}. (Generalization to take into
account off-shell $Z$-boson and adjustment of the sign of $C_2$ in their Eq.(4)).


\subsection{$H \to W^+W^-$ production, processes 123-126}
These processes represent the production of a Higgs boson that decays to $W^+ W^-$,
with subsequent decay into leptons. For process {\tt 123}, the exact form of the triangle
loop coupling a Higgs boson to two gluons is included, with both top and bottom quarks
circulating in the loop. This is to be contrasted with process {\tt 113} in which only the
top quark contribution is included in the effective coupling approach.

Process {\tt 124} includes only the effect of the interference of the
Higgs and $gg \to W^+W^-$ amplitudes, as described in ref.~\cite{Campbell:2011cu}.
The calculation is available at LO only. LO corresponds to $O(\alpha_s^2)$ in this case.
The calculation of loops containing the third quark generation
includes the effect of the top quark mass (but $m_b=0$), while the first two
generations are considered massless. For numerical stability, a small cut on the
transverse momentum of the $W$ bosons is applied: $p_T(W)>0.05$~GeV for loops
containing massless (first or second generation) quarks, $p_T(W)>2$~GeV for $(t,b)$
loops. This typically removes less than $0.1$\% of the cross section. The
values of these cutoffs can be changed by editing ${\tt src/HWW/gg\_WW\_int.f}$ and recompiling.

Process {\tt 125} includes all $gg$-intitiated diagrams that have a Higgs boson in the $s$-channel,
namely the square of the $s$-channel Higgs boson production and the interference with the diagrams
that do not contain a Higgs boson, (i.e. $gg \to W^+W^- \to \nu_e e^+ e^- \bar{\nu_e}$).

The result for the square of the box diagrams alone, i.e. the process
$gg \to W^+W^- \to \nu_e e^+ e^- \bar{\nu_e}$, may be obtained by running process
{\tt nproc=61} with {\tt part=virt} and {\tt ggonly=.true.} 

Process {\tt 126} calculates the full result for this process from  $gg$-intitiated diagrams.
This includes diagrams that have a Higgs boson in the $s$-channel, the continuum $W^+W^-$
diagrams described above and their interference. 


\subsection{$H \to ZZ \to e^- e^+ \mu^- \mu^+$ production, processes 128-133}
These processes represent the production of a Higgs boson that decays to $Z Z$,
with subsequent decay into charged leptons. For process {\tt 128}, the exact form of the triangle
loop coupling a Higgs boson to two gluons is included, with both top and bottom quarks
circulating in the loop. This is to be contrasted with process {\tt 116} in which only the
top quark contribution is included in the effective coupling approach.

Process {\tt 129} includes only the effect of the interference of the
Higgs and $gg \to ZZ$ amplitudes.
The calculation is available at LO only. LO corresponds to $O(\alpha_s^2)$ in this case.
The calculation of loops containing the third quark generation
includes the effect of both the top quark mass and the bottom quark, while the first two
generations are considered massless. For numerical stability, a small cut on the
transverse momentum of the $Z$ bosons is applied: $p_T(Z)>0.05$~GeV.
This typically removes less than $0.1$\% of the cross section. The
values of these cutoffs can be changed by editing ${\tt src/ZZ/getggZZamps.f}$ and recompiling.

Process {\tt 130} includes all $gg$-intitiated diagrams that have a Higgs boson in the $s$-channel,
namely the square of the $s$-channel Higgs boson production and the interference with the diagrams
that do not contain a Higgs boson, (i.e. $gg \to Z/\gamma^*+Z/\gamma^* \to e^- e^+ \mu^- \mu^+$).

Process {\tt 131} calculates the full result for this process from  $gg$-intitiated diagrams.
This includes diagrams that have a Higgs boson in the $s$-channel, the continuum $ Z/\gamma^*+Z/\gamma^*$
diagrams described above and their interference. 

Process {\tt 132}  gives the result for the square of the box diagrams alone, i.e. the process
$gg \to Z/\gamma^*+Z/\gamma^* \to e^- e^+ \mu^- \mu^+$.

Process {\tt 133} calculates the interference for the $qg$ initiated process.
\subsection{$H+b$ production, processes 136--138}
\label{subsec:Hb}

\begin{center}
[{\it For more details on this calculation, please refer to hep-ph/0204093}]
\end{center}

These processes represent the production of a Standard Model Higgs
boson that decays into a pair of bottom quarks,
in association with a further bottom quark. The initial state at lowest order
is a bottom quark and a gluon.
The calculation may be performed at NLO, although radiation from the
bottom quarks in the Higgs decay is not included.

For this process, the matrix elements are divided up into a number of
different sub-processes, so the user must sum over these after performing
more runs than usual. At lowest order one can proceed as normal, using
{\tt nproc=136}. For a NLO calculation, the sequence of runs is as follows:
\begin{itemize}
\item Run {\tt nproc=136} with {\tt part=virt} and {\tt part=real} (or, both
at the same time using {\tt part=tota});
\item Run {\tt nproc=137} with {\tt part=real}.
\end{itemize}
The sum of these yields the cross-section with one identified $b$-quark in
the final state. To calculate the contribution with two $b$-quarks in the
final state, one should use {\tt nproc=138} with {\tt part=real}.

When {\tt removebr} is true, the Higgs boson does not decay.

\subsection{$t\bar{t}$ production with 2 semi-leptonic decays, processes 141--145}
\label{subsec:ttbar}

These processes describe $t \bar{t}$ production including semi-leptonic
decays for both the top and the anti-top. 
In version 6.2 we have updated this to use the one-loop amplitudes of
ref.~\cite{Badger:2011yu}. The code for the virtual amplitudes now runs
about three times faster than earlier versions where the virtual
amplitudes of ref.~\cite{Korner:2002hy} were used.  
Switching {\tt zerowidth} from {\tt .true.} to {\tt .false.} only affects 
the $W$ bosons from the top quark decay, because our method of including spin
correlations requires the top quark to be on shell.

Process {\tt 141} includes all corrections, i.e.\ both radiative corrections
to the decay and to the production. This process is therefore the
basic process for the description of top production where both quarks
decay semi-leptonically.  When {\tt removebr} is true in process {\tt 141},
the top quarks do not decay.
When one wishes to calculate observables related to the decay of the top
quark, {\tt removebr} should be false in process {\tt 141}.
The LO calculation proceeds as normal. 
At NLO, there are two options:
\begin{itemize}
\item {\tt part=virt, real} or {\tt tota} : final state radiation is included
in the production stage only
\item {\tt part = todk} : radiation is included in the decay of the top
quark also and the final result corresponds to the sum of real and virtual
diagrams. 
Note that these runs automatically perform an extra integration, so
will take a little longer.
\end{itemize}

Process {\tt 142} includes only the corrections in
the semileptonic decay of the top quark. Thus it is of primary
interest for theoretical studies rather than for physics applications.  
Because of the method that we have used to include the radiation in the decay,
the inclusion of the corrections in the decay does not change the
total cross section. This feature is explained in section 6 of ref.~\cite{Campbell:2012uf}.

In the case of process {\tt 145}, there are no spin correlations in
the decay of the top quarks. The calculation is performed by
multiplying the spin summed top production cross section, by the decay
matrix element for the decay of the $t$ and the $\bar{t}$. These
processes may be used as a diagnostic test for the importance of the
spin correlation.



\subsection{$t\bar{t}$ production with decay and a gluon, process 143}
This process describes lowest order $t \bar{t}+g$ production 
including two leptonic decays $t \to b l \nu$. 
When {\tt removebr} is true, the top quarks do not decay.
This LO process only includes radiation only includes radiation in production.

\subsection{$t\bar{t}$ production with one hadronic decay, processes 146--151}

These processes describe the hadronic production of a pair of top
quarks, with one quark decaying hadronically and one quark decaying
semileptonics.  For processes {\tt 146--148}, the top quark decays
semileptonically whereas the anti-top quark decays hadronically.  For
processes {\tt 149--151}, the top quark decays hadronically whereas the
anti-top quark decays semi-leptonically.  The base processes for
physics are process {\tt 146} and {\tt 149} which include
radiative corrections in both production and decay.  Switching {\tt zerowidth} from 
{\tt .true.} to {\tt .false.} only affects the $W$ bosons from the top
quark decay, because our method of including spin correlations
requires the top quark to be on shell.
When one wishes to calculate observables related to the decay of the top
quark, {\tt removebr} should be false in processes {\tt 146} and {\tt 149}.
The LO calculation proceeds as normal. At NLO, there are two options:
\begin{itemize}
\item {\tt part=virt, real} or {\tt tota} : final state radiation is included
in the production stage only
\item {\tt part = todk} : radiation is included in the decay of the top
quark also and the final result corresponds to the sum of real and virtual
diagrams. 
Note that these runs automatically perform an extra integration, so
will take a little longer.
\end{itemize}


Processes {\tt 147} and {\tt 150} include only the radiative 
corrections in the decay of the top quark without including 
the radiative corrections in the hadronic decay of the $W$-boson.
Because of the method that we have used to include the radiation in the decays,
the inclusion of the corrections in this stage of the decay does not change the
total cross section.
Process {\tt 148} ({\tt 151}) includes only the radiative corrections
in the hadronic decay of the $W$-boson coming from the anti-top (top).
The inclusion of the corrections in this stage of the decay increases the
partial width by the normal $\alpha_s/\pi$ factor.

\subsection{$Q\overline{Q}$ production, processes 157--159}
These processes calculate the production of heavy quarks
({\tt 157} for top, {\tt 158} for bottom and {\tt 159} for charm) up to NLO 
using the matrix elements of ref.~\cite{Nason:1987xz}. No decays
are included.

\subsection{$t{\bar t}+$~jet production, process 160}
This process calculates the production of top quarks and a single jet
at LO, without any decay of the top quarks.

\subsection{Single top production, processes 161--177}
\label{subsec:stop}

\begin{center}
[{\it For more details on this calculation, please refer to hep-ph/0408158}]
\end{center}

These processes represent single top production and may be calculated up to
NLO as described below.

Single top production is divided as usual into $s$-channel 
(processes {\tt 171-177}) and $t$-channel ({\tt 161-167})
diagrams. Each channel includes separately the production of a top
and anti-top quark, which is necessary when calculating rates at the LHC.
Below we illustrate the different use of these processes by considering
$t$-channel top production ({\tt 161,162}), although the procedure is the same
for anti-top production ({\tt 166,167}) and the corresponding $s$-channel
processes ({\tt 171,172}) and ({\tt 176,177}).


To calculate cross-sections that do not include any decay of the (anti-)top
quark, one should use process {\tt 161}
(or, correspondingly, {\tt 166}, {\tt 171} and {\tt 176}) with {\tt removebr}
true. The procedure is exactly the same
as for any other process.
Switching {\tt zerowidth} from {\tt .true.} to {\tt .false.} only affects
the $W$ boson from the top quark decay.

For processes {\tt 161}, {\tt 162}, {\tt 163}, {\tt 166}, {\tt 167}
and {\tt 168} the default behaviour when {\tt removebr} is true is that
partons are clustered into jets using the supplied jet
algorithm parameters, but no cut is applied on the number of jets.
This behaviour can be altered by changing the value of the
variable {\tt notag} in the file {\tt src/User/setnotag.f}.
 
When one wishes to calculate observables related to the decay of the top
quark, {\tt removebr} should be false.
The LO calculation proceeds as normal. At NLO, there are two options:
\begin{itemize}
\item {\tt part=virt, real} or {\tt tota} : final state radiation is included
in the production stage only
\item {\tt part = todk} : radiation is included in the decay of the top
quark also and the final result corresponds to the sum of real and virtual
diagrams. This process can only be performed at NLO with 
{\tt zerowidth = .true}. This should be set automatically.
Note that these runs automatically perform an extra integration, so
will take a little longer.
\end{itemize}

The contribution from radiation in the decay may be calculated separately using
process {\tt 162}. This process number can be used with {\tt part=virt,real}
only. To ensure consistency, it is far simpler to use {\tt 161}
and this is the recommended approach.

A further option is provided for the $t-$channel single top process (when no
top quark decay is considered), relating to NLO real radiation diagrams that
contain a bottom quark. In the processes above the bottom quark is taken to
be massless. To include the effect of $m_b > 0$, one can run process
{\tt 163} ({\tt 168}) in place of {\tt 161} ({\tt 166}) and additionally include
process $\tt 231$ ({\tt 236}) at leading order.
The non-zero bottom quark mass has little effect on
the total cross section, but enables a (LO) study of the bottom quark kinematics.
Higher order corrections to the bottom quark kinematics can only be studied by running
process {\tt 231} ({\tt 236}) at NLO.

\subsection{$Wt$ production, processes 180--187}
\label{subsec:wt}

\begin{center}
[{\it For more details on this calculation, please refer to hep-ph/0506289}]
\end{center}

These processes represent the production of a $W$ boson that decays leptonically
in association with a top quark. The lowest order diagram involves a gluon and
a bottom quark from the PDF, with the $b$-quark radiating a $W$ boson and
becoming a top quark. The calculation can be performed up to NLO.

Processes {\tt 180} and {\tt 185} produce a top quark that does not decay,
whilst in processes {\tt 181} and {\tt 186} the top quark decays leptonically.
Consistency with
the simpler processes ({\tt 180,185}) can be demonstrated by running process
{\tt 181,186} with {\tt removebr} set to true.

At next-to-leading order, the calculation includes contributions from diagrams
with two gluons in the initial state, $gg \rightarrow Wtb$. The $p_T$ of the
additional $b$ quark is vetoed according to the value of the parameter
{\tt ptmin\_bjet} which is specified in the input file. The contribution from
these diagrams when the $p_T$ of the $b$ quark is above {\tt ptmin\_bjet}
is zero. The values of this parameter and the factorization scale ({\tt facscale})
set in the input file should be chosen carefully. Appropriate values for both
(in the range $30$-$100$~GeV) are discussed in the associated paper.

When one wishes to calculate observables related to the decay of the top
quark, {\tt removebr} should be false.
The LO calculation proceeds as normal. At NLO, there are two options:
\begin{itemize}
\item {\tt part=virt, real} or {\tt tota} : final state radiation is included
in the production stage only
\item {\tt part = todk} : radiation is included in the decay of the top
quark also and the final result corresponds to the sum of real and virtual
diagrams. This process can only be performed at NLO with 
{\tt zerowidth = .true}. This should be set automatically.
Note that these runs automatically perform an extra integration, so
will take a little longer.
\end{itemize}

The contribution from radiation in the decay may be calculated separately using
processes {\tt 182,187}. These process numbers can be used with {\tt part=virt,real}
only. To ensure consistency, it is far simpler to use {\tt 181,186}
and this is the recommended approach.

\subsection{$H+$~jet production, processes 201--210}
\label{subsec:hjet}

These processes represent the production of a Higgs boson in association
with a single jet, with the subsequent decay of the Higgs to either
a pair of bottom quarks (processes {\tt 201,203,206}) 
or to a pair of tau's ({\tt 202,204,207}),
or to a pair of $W$'s which decay leptonically ({\tt 208}),
or to a pair of $Z$'s which decay leptonically ({\tt 209}),
or to a pair of photons ({\tt 210}).

The Higgs boson couples to a pair of gluons via a loop of heavy fermions
which, in the Standard Model, is accounted for almost entirely by including
the effect of the top quark alone. For processes {\tt 201,202,206,207}, the
matrix elements include the full dependence on the top quark mass.
The calculation can only be performed at LO. 
However, the Higgs boson can either be the Standard Model one
(processes {\tt 201,202}) or a pseudoscalar ({\tt 206,207}).
Note that the pseudoscalar case corresponds, in the heavy top limit, to the effective Lagrangian,
\begin{equation}
\mathcal{L} = \frac{1}{8\pi v} \, G^a_{\mu\nu} \widetilde G^{\mu\nu}_a A \;,
\end{equation}
where $\widetilde G^{\mu\nu}_a = i\epsilon^{\mu\nu\alpha\beta}
 G_{\alpha\beta}^a$.
The interaction differs from the scalar case in Eq.~{\ref{eq:HeffL}} by a factor of $3/2$
and hence the rate is increased by a factor of $(3/2)^2$.


For processes {\tt 203,204,208,209,210}, the calculation is performed in the
limit of infinite top quark mass, so that NLO results can be obtained.
The virtual matrix elements have been implemented from
refs~\cite{Ravindran:2002dc} and~\cite{Schmidt:1997wr}.
Phenomenological results have previously been 
given in refs.~\cite{deFlorian:1999zd},\cite{Ravindran:2002dc} 
and \cite{Glosser:2002gm}.
Note that the effect of radiation from the bottom quarks in process {\tt 203}
is not included.

When {\tt removebr} is true in processes {\tt 201}, {\tt 203}, {\tt 206}, {\tt 208}, {\tt 209}
and {\tt 210}, the Higgs boson does not decay.

\subsection{Higgs production via WBF, processes 211--217}
\label{subsec:wbf}

\begin{center}
[{\it For more details on this calculation, please refer to hep-ph/0403194}]
\end{center}

These processes provide predictions for the production of a Higgs boson in
association with two jets via weak-boson fusion (WBF). The Higgs boson
subsequently decays to either a pair of bottom quarks
(processes {\tt 211, 216}), to a pair of tau's ({\tt 212, 217}), 
to a pair of $W$ bosons ({\tt 213}),
to a pair of $Z$ bosons ({\tt 214}),
or to a pair of photons ({\tt 215}).

Calculations can be performed up to NLO for processes {\tt 211}--{\tt 215}.
In addition to this, processes {\tt 216} and {\tt 217} provide the lowest
order calculation of the WBF reaction which radiates an additional jet. 

When {\tt removebr} is true, the Higgs boson does not decay.

\subsection{$\tau^+\tau^-$ production, process 221}
\label{subsec:tautau}

This process provides predictions for the production of a tau lepton
pair mediated by $\gamma^*/Z$, with subsequent leptonic decays. The calculation is available at LO
only. The relevant matrix elements are adapted from the ones in
ref.~\cite{Kleiss:1988xr}.

When {\tt removebr} is true, the tau leptons do not decay.


%
\subsection{$e^-e^+ \nu_{\mu} \bar\nu_{\mu} $-pair production via WBF, processes 222}
The {\it weak} processes occur in $O(\alpha^6)$, whereas the {\it strong} processes occur in $O(\alpha^4 \alpha_S^2)$.
These processes can currently only be calculated at lowest order.
\begin{eqnarray}
&222 &  f(p_1)+f(p_2) \to Z(e^-(p_3),e^+(p_4))Z(\nu_\mu(p_5),\bar{\nu}_\mu(p_6)))+f(p_7)+f(p_8) [WBF]  \nonumber \\
&2201 & f(p_1)+f(p_2) \to Z(e^-(p_3),e^+(p_4))Z(\mu^-(p_5),\mu^+(p_6)))+f(p_7)+f(p_8) [strong] \nonumber \\
&2221 & f(p_1)+f(p_2) \to Z(e^-(p_3),e^+(p_4))Z(\nu_\mu(p_5),\bar{\nu}_\mu(p_6)))+f(p_7)+f(p_8) [strong]  \nonumber
\end{eqnarray}
%
\subsection{$\nu_e  e^+ \mu^- \mu^+$-pair production via WBF, processes 223,2231}
The {\it weak} processes occur in $O(\alpha^6)$, whereas the {\it strong} processes occur in $O(\alpha^4 \alpha_S^2)$.
These processes can currently only be calculated at lowest order.
\begin{eqnarray}
& 223  & f(p_1)+f(p_2) \to W^+(\nu_e(p_3),e^+(p_4))Z(\mu^-(p_5),\mu^+(p_6)))+f(p_7)+f(p_8) [weak]  \nonumber \\
& 2231 & f(p_1)+f(p_2) \to W^+(\nu_e(p_3),e^+(p_4))Z(\mu^-(p_5),\mu^+(p_6)))+f(p_7)+f(p_8) [strong]  \nonumber
\end{eqnarray}

\subsection{$e^- \bar\nu_{e} \nu_{\mu} \mu^+$-pair production via WBF, processes 224,2241}
The {\it weak} processes occur in $O(\alpha^6)$, whereas the {\it strong} processes occur in $O(\alpha^4 \alpha_S^2)$.
These processes can currently only be calculated at lowest order.
\begin{eqnarray}
&224   & f(p_1)+f(p_2) \to W^-(e^-(p_3),\bar{\nu}_e(p_6)))W^+(\nu_\mu(p_5),\mu^+(p_4))+f(p_7)+f(p_8) [WBF]    \nonumber \\
&2241  & f(p_1)+f(p_2) \to W^-(e^-(p_3),\bar{\nu}_e(p_6)))W^+(\nu_\mu(p_5),\mu^+(p_4))+f(p_7)+f(p_8) [strong]  \nonumber   
\end{eqnarray}
%
\subsection{$e^- \bar\nu_{e} \mu^- \mu^+$-pair production via WBF, processes 225,2251}
The {\it weak} processes occur in $O(\alpha^6)$, whereas the {\it strong} processes occur in $O(\alpha^4 \alpha_S^2)$.
These processes can currently only be calculated at lowest order.
\begin{eqnarray}
&225   & f(p_1)+f(p_2) \to W^-(e^-(p_3),\bar{\nu}_e(p_4))Z(\mu^-(p_5),\mu^+(p_6)))+f(p_7)+f(p_8) [weak]    \nonumber \\
&2251  &  f(p_1)+f(p_2) \to W^-(e^-(p_3),\bar{\nu}_e(p_4))Z(\mu^-(p_5),\mu^+(p_6)))+f(p_7)+f(p_8) [strong]  \nonumber  
\end{eqnarray}


\subsection{$e^- e^+ \bar\nu_{e} \nu_{e}$-pair production via WBF, processes 226}
The {\it weak} processes occur in $O(\alpha^6)$, whereas the {\it strong} processes occur in $O(\alpha^4 \alpha_S^2)$.
This process can currently only be calculated at lowest order.
\begin{eqnarray}
&226  &  f(p_1)+f(p_2) \to e-(p_3)+e^+(p_4)+\nu_e(p_5)+\bar{\nu}_e(p_6)+f(p_7)+f(p_8) [WBF]    \nonumber
\end{eqnarray}

\subsection{$\nu_{e} e^+ \nu_{\mu} \mu^+ $-pair production via WBF, processes 228,2281}
This is pure electroweak process, occuring in $O(\alpha^6)$.
These processes can currently only be calculated at lowest order.
\begin{eqnarray}
&228  &  f(p_1)+f(p_2) \to W^+(\nu_e(p_3),e^+(p_4)))W^+(\nu_\mu(p_5),\mu^+(p_6))+f(p_7)+f(p_8) [WBF]    \nonumber \\
&2281 &  f(p_1)+f(p_2) \to W^+(\nu_e(p_3),e^+(p_4)))W^+(\nu_\mu(p_5),\mu^+(p_6))+f(p_7)+f(p_8) [strong]     \nonumber
\end{eqnarray}
\subsection{$e^- \bar{\nu}_{e} \mu^- \bar{\nu}_{\mu} $-pair production via WBF, processes 229,2291}
This is pure electroweak process, occuring in $O(\alpha^6)$.
These processes can currently only be calculated at lowest order.
\begin{eqnarray}
& 229 &   f(p_1)+f(p_2) \to W^-(e^-(p_3),\bar{\nu}_e(p_4)))W^+(\mu^-(p_5),\bar{\nu}_\mu(p_6))+f(p_7)+f(p_8) [WBF]   \nonumber \\
&2291 &  f(p_1)+f(p_2) \to W^-(e^-(p_3),\bar{\nu}_e(p_4)))W^-(\mu^-(p_5),\bar{\nu}_\mu(p_6))+f(p_7)+f(p_8) [strong]  \nonumber  
\end{eqnarray}

\subsection{$t$-channel single top with an explicit $b$-quark, processes 231--240}
\label{subsec:stopb}

\begin{center}
[{\it For more details on this calculation, please refer to arXiv:0903.0005~[hep-ph]}]
\end{center}

These represent calculations of the $t$-channel single top~({\tt 231}) and anti-top~({\tt 231})
processes in a scheme with four flavours of quark in the proton, so that $b$-quarks are not present in the proton.
The $b$-quark is instead explicitly included in the final state.

Processes {\tt 232} and {\tt 236} represent $t$-channel single top production in association
with a further jet and may be calculated at LO only.

Processes {\tt 233} and {\tt 238} are the complete four-flavour scheme $t$-channel single top production processes.
These are therefore the processes that should be used for most physics applications.
When one wishes to calculate observables related to the decay of the top
quark, {\tt removebr} should be false in processes {\tt 233} and {\tt 236}.
The LO calculation proceeds as normal. At NLO, there are two options:
\begin{itemize}
\item {\tt part=virt, real} or {\tt tota} : final state radiation is included
in the production stage only
\item {\tt part = todk} : radiation is included in the decay of the top
quark also and the final result corresponds to the sum of real and virtual
diagrams. 
Note that these runs automatically perform an extra integration, so
will take a little longer.
\end{itemize}


Processes {\tt 234} and {\tt 239} give the extra contribution due to radiation 
in top decay. These processes are mainly of theoretical interest. 

Processes {\tt 235} and {\tt 240} are the leading order single top processes with an 
extra radiated parton. These processes do not includes jets produced in the decay process.

\subsection{$W^+W^++$jets production, processes 251,252}
These processes represent the production of two $W^+$ 
bosons in association with two (process {\tt 251}) or three (process {\tt 252})
jets.  The lowest order at which two positively charged $W$ bosons 
can be produced is with two jets. 
This process is only implemented for leptonic decays of the 
$W$ particles. The calculation is available at LO only.
The calculation and code are from ref.~\cite{Melia:2010bm}.
{\tt removebr} is not implemented and has no effect.

\subsection{$Z+Q$ production, processes 261--267}
\label{subsec:ZQ}

\begin{center}
[{\it For more details on this calculation, please refer to hep-ph/0312024}]
\end{center}

These processes represent the production of a $Z$
boson that decays into a pair of electrons,
in association with a heavy quark, $Q$.

For processes {\tt 261}, {\tt 262}, {\tt 266} and {\tt 267} the initial
state at lowest order is the heavy quark and a gluon and 
the calculation may be performed at NLO.
As for $H+b$ production, the matrix elements are divided into two
sub-processes at NLO. Thus the user must sum over these after performing
more runs than usual. At lowest order one can proceed as normal, using
{\tt nproc=261} (for $Z+b$) or {\tt nproc=262} (for $Z+c$).
For a NLO calculation, the sequence of runs is as follows:
\begin{itemize}
\item Run {\tt nproc=261} (or {\tt 262}) with {\tt part=virt} and
{\tt part=real} (or, both at the same time using {\tt part=tota});
\item Run {\tt nproc=266} (or {\tt 267}) with {\tt part=real}.
\end{itemize}
The sum of these yields the cross-section with one identified heavy quark in
the final state when {\tt inclusive} is set to {\tt .false.} . To calculate the
rate for at least one heavy quark, {\tt inclusive} should be {\tt .true.}.

For processes {\tt 263} and
{\tt 264}, the calculation uses the matrix elements for the production
of a $Z$ and a heavy quark pair and demands that one of the heavy quarks
is not observed. It may either lie outside the range of $p_T$ and $\eta$
required for a jet, or both quarks may be contained in the same jet.
Due to the extra complexity (the calculation must retain the full
dependence on the heavy quark mass), this can only be computed at LO.

When {\tt removebr} is true, the $Z$ boson does not decay.

\subsection{$H + 2$~jet production, processes 270--274}

These processes represent the production of a Standard Model Higgs boson
in association with two jets. The Higgs boson
subsequently decays to either a pair of photons ({\tt nproc=270}), a bottom quark pair ({\tt nproc=271}), 
a pair of tau's ({\tt nproc=272}), a pair of leptonically decaying $W$'s ({\tt nproc=273}) 
or a pair of leptonically decaying $Z$'s ({\tt nproc=274}).

The matrix elements are included in the infinite top mass limit
using the effective Lagrangian approach. 

When {\tt removebr} is true, the Higgs boson does not decay.

\subsection{$H + 3$~jet production, processes 275-278}

These processes represent the production of a Standard Model Higgs boson
in association with three jets. The Higgs boson
subsequently decays to either a bottom quark pair ({\tt nproc=275}), 
or a pair of tau's ({\tt nproc=276})
or a pair of $W$'s that decay leptonically into a single generation of leptons ({\tt nproc=278})
or a pair of $Z$'s that decay leptonically into a single generation of leptons ({\tt nproc=279}).
The matrix elements are included in the infinite top mass limit
using an effective Lagrangian approach. These calculations can be
performed at LO only.

When {\tt removebr} is true, the Higgs boson does not decay.

\subsection{Direct $\gamma$ production, processes 280--282}
\label{subsec:dirphot}

These processes represent the production a real photon.
Since this process includes a real photon, the cross section diverges
when the photon is very soft or in the direction of the beam.
Thus in order to produce sensible results, the input file must supply values for both
{\tt ptmin\_photon} and {\tt etamax\_photon}. This will ensure that
the cross section is well-defined.

The calculation of process {\tt 282} is only available at leading order.

\subsection{Direct $\gamma$ + heavy flavour production, processes 283--284}
\label{subsec:heavyfl}

These processes represent the production a real photon with a $b$ quark
or a charm quark
Since this process includes a real photon, the cross section diverges
when the photon is very soft or in the direction of the beam.
Thus in order to produce sensible results, the input file must supply values for both
{\tt ptmin\_photon} and {\tt etamax\_photon}. This will ensure that
the cross section is well-defined.

The calculation of process {\tt 283}--{\tt 284} is only available at leading order.

\subsection{$\gamma\gamma$ production, processes 285-286}
\label{subsec:gamgam}

Process {\tt 285} represents the production of a pair of real photons.
Since this process includes two real photons, the cross section diverges
when one of the photons is very soft or in the direction of the beam.
Thus in order to produce sensible results, the input file must supply values for both
{\tt ptmin\_photon} and {\tt etamax\_photon}. This will ensure that
the cross section is well-defined.

The calculation of process {\tt 285} may be performed at NLO using either the
Frixione algorithm~\cite{Frixione:1998jh} or standard cone isolation.  This process also includes
the one-loop gluon-gluon contribution as given in
ref.~\cite{Bern:2002jx}.  The production of a photon via parton fragmentation is included at NLO and 
can be run separately by using the {\tt frag} option in {\tt part}. This option includes the contributions from the integrated 
photon dipole subtraction terms and the LO QCD matrix element multiplied by the fragmentation function.  

%Process {\tt 285} can be run using different cuts for each photon. Setting the first 9 characters of the runstring to 
%{\tt Stag\_phot} will apply the following default cuts:
%\begin{eqnarray*}
%p_T^{\gamma_1} > 40~\mbox{GeV}, \; p_T^{\gamma_2} > {\tt ptmin\_photon}~\mbox{GeV}, \; |\eta^{\gamma_i}| < {\tt etamax\_photon}
%\end{eqnarray*} 
%These values can be changed by editing the file photon\_cuts.f in src/User. 

The phase space cuts for the final state photons are defined in {\tt{input.DAT}}, for multiple photon processes such 
as {\tt 285 - 287} the $p_T$'s of the individual photons (hardest, second hardest and third hardest or softer) can be controlled independently. 
The remaining cuts on $R_{\gamma j}$, $\eta_{\gamma}$ etc. are applied universally to all photons. Users wishing to alter
this feature should edit the file {\tt{photon\_cuts.f}} in the directory {\tt{src/User}}. 


Process {\tt 286}, corresponding to $\gamma\gamma+$jet production, can be computed at leading order only.

\subsection{$\gamma\gamma\gamma$ production, process 287}
\label{subsec:trigam}

Process {\tt 287} represents the production of three real photons.
The cross section diverges
when one of the photons is very soft or in the direction of the beam.
Thus in order to produce sensible results, the input file must supply values for both
{\tt ptmin\_photon} and {\tt etamax\_photon}. This will ensure that
the cross section is well-defined.

The calculation of process {\tt 285} may be performed at NLO using either the
Frixione algorithm~\cite{Frixione:1998jh} or standard cone isolation.  The production of a photon via parton fragmentation is included at NLO and 
can be run separately by using the {\tt frag} option in {\tt part}. This option includes the contributions from the integrated 
photon dipole subtraction terms and the LO QCD matrix element multiplied by the fragmentation function.  
The phase space cuts for the final state photons are defined in {\tt{input.DAT}}, for multiple photon processes such 
as {\tt 285 - 287} the $p_T$'s of the individual photons (hardest, next-to hardest and softest) can be controlled independently. 
The remaining cut on $R_{\gamma j}$, $\eta_{\gamma}$ etc. are applied universally to all photons. Users wishing to alter
this feature should edit the file {\tt{photon\_cuts.f}} in the directory {\tt{src/User}}. 


\subsection{$\gamma\gamma\gamma\gamma$ production, process 289}
\label{subsec:fourgam}

Process {\tt 289} represents the production of four real photons.
The cross section diverges
when one of the photons is very soft or in the direction of the beam.
Thus in order to produce sensible results, the input file must supply values for both
{\tt ptmin\_photon} and {\tt etamax\_photon}. This will ensure that
the cross section is well-defined.

The calculation of process {\tt 289} may be performed at NLO using either the
Frixione algorithm~\cite{Frixione:1998jh} or standard cone isolation.  The production of a photon via parton fragmentation is included at NLO and
can be run separately by using the {\tt frag} option in {\tt part}. This option includes the contributions from the integrated
photon dipole subtraction terms and the LO QCD matrix element multiplied by the fragmentation function.
The phase space cuts for the final state photons are defined in {\tt{input.DAT}}, for multiple photon processes such
as {\tt 285 - 289} the $p_T$'s of the individual photons (hardest, next-to hardest and softest) can be controlled independently.
The remaining cut on $R_{\gamma j}$, $\eta_{\gamma}$ etc. are applied universally to all photons. Users wishing to alter
this feature should edit the file {\tt{photon\_cuts.f}} in the directory {\tt{src/User}}.

Note that for this process the second softest and softest photons are forced to have equal minimum $p_T$, defined
by the {\tt{[ptmin\_photon(3rd)]}} variable in the input file.


\subsection{$W\gamma$ production, processes 290-297}
\label{subsec:wgamma}

These processes represent the production of a $W$ boson which subsequently
decays leptonically, in association with a real photon.
Since this process includes a real photon, the cross section diverges
when the photon is very soft or in the direction of the beam.
Thus in order to produce sensible results, the input file must supply values for both
{\tt ptmin\_photon} and {\tt etamax\_photon}. Moreover, when the parameters {\tt zerowidth}
and {\tt removebr} are set to {\tt .false.} the decay $W \to \ell \nu$ will include
photon radiation from the lepton, so that a non-zero {\tt R(photon,lept)\_min} should
also be supplied. This will ensure that the cross section is well-defined.
Virtual amplitudes are taken from ref.~\cite{Dixon:1998py}.

The calculation of processes {\tt 290} and {\tt 295} may be performed
at NLO using the Frixione algorithm~\cite{Frixione:1998jh} or standard isolation. 

For processes {\tt 290} and {\tt 295} the role of {\tt mtrans34cut} changes to become a cut 
on the transverse mass on the $M_{345}$ system, i.e. the photon is included with the leptons in the cut. 

\subsubsection{Anomalous $WW\gamma$ couplings}
Processes {\tt 290}-{\tt 297} may also be computed including the effect of anomalous $WW\gamma$ couplings, making
use of the amplitudes calculated in Ref.~\cite{DeFlorian:2000sg}. Including only dimension 6 operators
or less and demanding gauge, $C$ and $CP$ invariance gives the general form of the anomalous vertex~\cite{DeFlorian:2000sg},
\begin{eqnarray}
 \Gamma^{\alpha \beta \mu}_{W W \gamma}(q, \bar q, p) &=& 
  {\bar q}^\alpha g^{\beta \mu} 
    \biggl( 2 + \Delta\kappa^\gamma + \lambda^\gamma {q^2\over M_W^2} \biggr) 
 - q^\beta g^{\alpha \mu}
    \biggl( 2 + \Delta\kappa^\gamma + \lambda^\gamma {{\bar q}^2\over M_W^2}
\biggr) \nonumber \\  
&& \hskip 1 cm
 + \bigl( {\bar q}^\mu - q^\mu \bigr) 
 \Biggl[ - g^{\alpha \beta} \biggl( 
   1 + {1\over2} p^2 \frac{\lambda^\gamma}{M_W^2} \biggr) 
 +\frac{\lambda^\gamma}{M_W^2} p^\alpha p^\beta \Biggr] \,,
\end{eqnarray}
where the overall coupling has been chosen to be $-|e|$. The parameters that
specify the anomalous couplings, $\Delta\kappa^\gamma$ and $\lambda^\gamma$, are
specified in the input file as already discussed in Section~\ref{subsec:diboson}.
If the input file contains a negative value for the form-factor scale $\Lambda$
then no suppression factors are applied to the anomalous couplings.
Otherwise, the couplings are included
in \MCFM only after suppression by dipole form factors,
\begin{displaymath}
\Delta \kappa^{\gamma} \rightarrow
 \frac{\Delta \kappa_1^{\gamma}}{(1+\hat{s}/\Lambda^2)^2}, \qquad
\lambda^{\gamma} \rightarrow
 \frac{\Delta \lambda^{\gamma}}{(1+\hat{s}/\Lambda^2)^2} \;,
\end{displaymath}
where $\hat{s}$ is the $W\gamma$ pair invariant mass.

The Standard Model cross section is obtained by setting $\Delta\kappa^\gamma = \lambda^\gamma = 0$.

\subsection{$Z\gamma$, production, processes 300, 305}
\label{subsec:zgamma}


Processes {\tt 300} and {\tt 305} represent the production of a $Z$ boson (or virtual photon for process {\tt 300})
in association with a real photon. The $Z/\gamma^*$ subsequently decays into 
either an $e^+ e^-$ pair ({\tt nproc=300}) or neutrinos ({\tt nproc=305}).
Since these processes include a real photon, the cross section diverges
when the photon is very soft or in the direction of the beam.
Thus in order to produce sensible results, the input file must supply values for both
{\tt ptmin\_photon} and {\tt etamax\_photon}. Moreover, when the parameters {\tt zerowidth}
and {\tt removebr} are set to {\tt .false.} the decay $Z \to e^- e^+$ ({\tt nproc=300})
will include photon radiation from both leptons, so that a non-zero {\tt R(photon,lept)\_min}
should also be supplied. This will ensure that the cross section is well-defined.
Virtual amplitudes are taken from ref.~\cite{Dixon:1998py}.
The calculation of processes {\tt 300} may be performed
at NLO using the Frixione algorithm~\cite{Frixione:1998jh} or standard isolation. 
%Processes {\tt 302} and {\tt 307}  represents the production of a $Z$ boson (or virtual photon)
%in association with a real photon and an additional jet. These processes are also available at NLO including 
%the full fragmentation processes. Anomalous couplings are not available for these processes. 
%Processes {\tt 304} and {\tt 309}  represents the production of a $Z$ boson (or virtual photon)
%in association with a real photon and and two additional jets. These processes are available at leading order only. 
%When {\tt removebr} is true in process {\tt 300} or {\tt 302} the $Z$ boson does not decay.

For the process {\tt 300}  the role of {\tt mtrans34cut} changes to become a cut 
on the invariant mass on the $M_{345}$ system, i.e. the photon is included with the leptons in the cut. 

\subsubsection{Anomalous $ZZ\gamma$ and $Z\gamma\gamma$ couplings}
Processes {\tt 300}-{\tt 305} may also be computed including the effect of anomalous couplings between $Z$ bosons
and photons, making use of the amplitudes calculated in Ref.~\cite{DeFlorian:2000sg}.
Note that, at present, the effect of anomalous couplings is not included in the gluon-gluon
initiated contributions.

The anomalous $Z\gamma Z$ vertex (not present at all in the Standard Model),
considering operators up to dimension 8, is given by~\cite{DeFlorian:2000sg},
\begin{eqnarray}
 && \Gamma^{\alpha \beta \mu}_{Z \gamma Z}(q_1, q_2, p) = 
   \frac{i(p^2-q_1^2)}{M_Z^2} \Biggl( 
   h_1^Z \bigl( q_2^\mu g^{\alpha\beta} - q_2^\alpha g^{\mu \beta}
   \bigr)
    \nonumber \\ && + \frac{h_2^Z}{M_Z^2} p^\alpha \Bigl( p\cdot q_2\ g^{\mu\beta} -
            q_2^\mu p^\beta \Bigr)
   - h_3^Z \varepsilon^{\mu\alpha\beta\nu} q_{2\, \nu} 
   - \frac{h_4^Z}{M_Z^2} \varepsilon^{\mu\beta\nu\sigma} p^\alpha
p_\nu q_{2\, \sigma} \Biggl)
\end{eqnarray}
where the overall coupling has been chosen to be $|e|$ (and
$\epsilon^{0123}=+1$). The non-standard $Z_\alpha(q_1) \gamma_\beta(q_2)
\gamma_\mu(p)$ momentum-space vertex can be obtained from
this equation by setting $q_1^2 \to 0$ and replacing $h_i^Z \to
h_i^\gamma$. 
The parameters that
specify the anomalous couplings, $h_i^Z$ and $h_i^\gamma$ (for $i=1\ldots 4$), are
specified in the input file as, e.g. {\tt h1(Z)} and {\tt h1(gamma)}.
If the input file contains a negative value for the form-factor scale $\Lambda$
then no suppression factors are applied to these anomalous couplings.
Otherwise, the couplings are included
in \MCFM only after suppression by dipole form factors,
\begin{displaymath}
h_{1,3}^{Z/\gamma} \rightarrow
 \frac{h_{1,3}^{Z/\gamma}}{(1+\hat{s}/\Lambda^2)^3}, \qquad
h_{2,4}^{Z/\gamma} \rightarrow
 \frac{h_{2,4}^{Z/\gamma}}{(1+\hat{s}/\Lambda^2)^4}, \qquad
\end{displaymath}
where $\hat{s}$ is the $Z\gamma$ pair invariant mass. Note that these form factors are slightly
different from those discussed in Sections~\ref{subsec:diboson} and~\ref{subsec:wgamma}.

The Standard Model cross section is obtained by setting $h_i^Z = h_i^\gamma = 0$ for $i=1\ldots 4$.

\subsection{$Z\gamma\gamma$ production processes, 301, 306} 

Processes {\tt{301}} and {\tt{306}} represent the production of a $Z$ boson 
(or virtual photon for process {\tt 301}) in association with two photons.   The $Z/\gamma^*$ subsequently decays into 
either an $e^+ e^-$ pair ({\tt nproc=301}) or neutrinos ({\tt nproc=306}).
Since these processes include real photons, the cross section diverges
when either of the photons is very soft or in the direction of the beam.
Thus in order to produce sensible results, the input file must supply values for both
{\tt ptmin\_photon} and {\tt etamax\_photon}. Moreover, when the parameters {\tt zerowidth}
and {\tt removebr} are set to {\tt .false.} the decay $Z \to e^- e^+$ ({\tt nproc=301})
will include photon radiation from both leptons, so that a non-zero {\tt R(photon,lept)\_min}
should also be supplied. This will ensure that the cross section is well-defined.
Anomalous couplings are not currently implemented for these processes. 

\subsection{$Z\gamma j$, production, processes 302, 307}
\label{subsec:zgammajet}
Processes {\tt 302} and {\tt 307} represent the production of a $Z$ boson (or virtual photon)
in association with a real photon and at least one jet. 
The $Z/\gamma^*$ subsequently decays into 
either an $e^+ e^-$ pair ({\tt nproc=302}) or neutrinos ({\tt nproc=307}).
Since these processes include a real photon and a jet, the cross section diverges
when the photon or jet is very soft or in the direction of the beam.
Thus in order to produce sensible results, the input file must supply values for both
{\tt ptmin\_photon} and {\tt etamax\_photon}, and {\tt ptjet\_min} and {\tt etajet\_max}.
 Moreover, when the parameters {\tt zerowidth}
and {\tt removebr} are set to {\tt .false.} the decay $Z \to e^- e^+$ ({\tt nproc=302})
will include photon radiation from both leptons, so that a non-zero {\tt R(photon,lept)\_min}
should also be supplied. This will ensure that the cross section is well-defined.
The calculation of processes {\tt 302} and {\tt 307} may be performed
at NLO using the Frixione algorithm~\cite{Frixione:1998jh} or standard isolation. 
Anomalous couplings are not currently implemented for these processes. 


\subsection{$Z\gamma\gamma j$ and $Z\gamma j j $, 303, 304, 308 and 309}

These processes are available at LO only. The $Z/\gamma^*$ subsequently decays into 
either an $e^+ e^-$ pair ({\tt nproc=303,304}) or neutrinos ({\tt nproc=308,309}). 
Since these processes include a real photon and a jet, the cross section diverges
when a photon or a jet is very soft or in the direction of the beam.
Thus in order to produce sensible results, the input file must supply values for both
{\tt ptmin\_photon} and {\tt etamax\_photon}, and {\tt ptjet\_min} and {\tt etajet\_max}.
 Moreover, when the parameters {\tt zerowidth}
and {\tt removebr} are set to {\tt .false.} the decay $Z \to e^- e^+$ ({\tt nproc=303, 304})
will include photon radiation from both leptons, so that a non-zero {\tt R(photon,lept)\_min}
should also be supplied. This will ensure that the cross section is well-defined.
Anomalous couplings are not currently implemented for these processes. 



%Processes {\tt 303} and {\tt 308}  represents the production of a $Z$ boson (or virtual photon)
%in association with a two photons and and an additional jet. These processes are available at leading order only. 
%These processes do not currently have anomalous couplings implemented. 

\subsection{$W+Q+$~jet production processes 311--326}
\label{subsec:wQj}

These processes represent the production of a $W$
boson that decays leptonically,
in association with a heavy quark, $Q$ and an additional light jet. In
processes {\tt 311} and {\tt 316} $Q$ is a bottom quark, whilst
processes {\tt 321} and {\tt 326} involve a charm quark.
In these processes the quark $Q$ occurs as parton PDF in the initial state. 
The initial state in these processes consists of a light quark and a heavy 
quark, with the light quark radiating the $W$ boson. These calculations may
be performed at LO only.

When {\tt removebr} is true, the $W$ boson does not decay.

\subsection{$W+c+$~jet production, processes 331, 336}
\label{subsec:wcj}

These processes represent the production of a $W$
boson that decays leptonically,
in association with a charm quark and an additional light jet. 

In contrast to processes {\tt 321} and {\tt 326} described above, the initial
state in this case consists of two light quarks, one of which is a
strange quark which radiates the $W$ boson. The calculation may
be performed at LO only.

When {\tt removebr} is true, the $W$ boson does not decay.

\subsection{$Z+Q+$jet production, processes 341--357}
\label{subsec:ZQj}

\begin{center}
[{\it For more details on this calculation, please refer to hep-ph/0510362}]
\end{center}

These processes represent the production of a $Z$
boson that decays into a pair of electrons,
in association with a heavy quark, $Q$ and an untagged jet.

For processes {\tt 341} and {\tt 351} the initial state at lowest
order is the heavy quark and a gluon and the calculation may be
performed at NLO.  Thus in these processes the quark $Q$ occurs as
parton PDF in the initial state.  As for $H+b$ and $Z+Q$ production,
the matrix elements are divided into two sub-processes at NLO. Thus
the user must sum over these after performing more runs than usual. At
lowest order one can proceed as normal, using {\tt nproc=341} (for
$Zbj$) or {\tt nproc=351} (for $Zcj$).  For a NLO calculation, the
sequence of runs is as follows:
\begin{itemize}
\item Run {\tt nproc=341} (or {\tt 351}) with {\tt part=virt} and
{\tt part=real} (or, both at the same time using {\tt part=tota});
\item Run {\tt nproc=342} (or {\tt 352}) with {\tt part=real}.
\end{itemize}
The sum of these yields the cross-section with one identified heavy
quark and one untagged jet in the final state when {\tt inclusive} is
set to {\tt .false.} . To calculate the rate for at least one heavy
quark and one jet (the remaining jet may be a heavy quark, or
untagged), {\tt inclusive} should be {\tt .true.}.

Processes {\tt 346,347} and {\tt 356,357} are the lowest order processes that enter
the above calculation in the real contribution. They can be computed only at LO.

When {\tt removebr} is true, the $Z$ boson does not decay.

\subsection{$c \overline s \to W^+$, processes 361--363}
\label{subsec:csbar}
These processes represent the production of a $W^+$ from a charm and anti-strange
quark at LO. The $W^+$ boson decays into a neutrino and a positron.

The NLO corrections to this LO process include a contribution of the form,
$g\overline s \to W^+ \overline c$. For process {\tt 361} this contribution is
calculated in the approximation $m_c=0$ at NLO. In order to perform the NLO calculation 
for a non-zero value of $m_c$, one must instead sum the results of processes {\tt 362}
and {\tt 363} for {\tt part=tota}.

\subsection{$W\gamma\gamma$ production, processes 370-371}
\label{subsec:wgamgam}

These processes represent the production of a $W$ boson which subsequently
decays leptonically, in association with two real photons.
Since this process includes real photons, the cross section diverges
when the photon is very soft or in the direction of the beam.
Thus in order to produce sensible results, the input file must supply values for both
{\tt ptmin\_photon} and {\tt etamax\_photon}. Moreover, when the parameters {\tt zerowidth}
and {\tt removebr} are set to {\tt .false.} the decay $W \to \ell \nu$ will include
photon radiation from the lepton, so that a non-zero {\tt R(photon,lept)\_min} should
also be supplied. This will ensure that the cross section is well-defined.

These processes may be computed at leading order only.

\subsection{$W+Q$ production in the 4FNS, processes 401--408}
\label{subsec:wbbfilter}
These processes represent the production of a $W$ boson and one or more jets,
at least one of which is a $b$-quark, calculated in the 4-flavour number scheme (4FNS). 
This implies that contributions that explicitly contain a $b$-quark in the initial state
are not included.
These processes all use the same matrix
elements as processes 20 and 25 (see section~\ref{subsec:wbb}), but make different
cuts on the final state. The final state is specified by the process number and
the value of the flag {\tt inclusive}, as shown in Table~\ref{table:wbbfilter}.
An additional flag is hard-coded into the file {\tt src/User/filterWbbmas.f} to control
the inclusion of the 3-jet configuration, $(b,\overline b,j)$ when {\tt inclusive} is set to {\tt .true.}.
By default this is included, {\tt veto3jets = .false.}. If this flag is set to {\tt .true.} 
then the $(b,\overline b,j)$ contribution
is not included in processes 401, 402, 406, 407.

\begin{table}
\begin{center}
\begin{tabular}{|c|c|c|} \hline
 Process ($W^+$/$W^-$) & {\tt inclusive=.false.} & {\tt inclusive=.true.} \\
\hline
{\tt 401}/{\tt 406} & $(b)$ or $(\overline b)$ & + ($b,\overline b$) or ($b,j$) or ($\overline b,j$) \\
{\tt 402}/{\tt 407} & $(B)$ & + ($B,j$) \\
{\tt 403}/{\tt 408} & $(b,\overline b)$ & \mbox{(no extra configurations)} \\
\hline
\end{tabular}
\caption{The different final states allowed in the calculation of processes 401--408. A jet containing
both $b$ and $\overline b$ quarks is denoted by $B$ and a light (quark or gluon) jet by $j$. The inclusive
(right-hand) column also allows the final states in the exclusive (middle) column.}
\label{table:wbbfilter}
\end{center}
\end{table}

As usual, jets may be unobserved as a result of them falling outside the $p_T$
and rapidity ranges specified in the input file. In addition, the number of jets
may be different from the number of partons in the matrix element calculation as
a result of merging in the jet algorithm.

\subsection{$W+Q$ production in the 5FNS, processes 411, 416}
\label{subsec:wb5FNS}

These processes represent production of a $W$ boson in association with a
$b$-jet, computed in the 5-flavour number scheme, i.e. a $b$-quark is present in
the initial state. The lowest order processes are the same as in processes {\tt 311}, {\tt 316}.
The results at NLO are not physical cross sections since part of the corrections
are not included in order to avoid double-counting with the 4FNS process (processes
{\tt 401} and {\tt 406}). To obtain combined 4FNS+5FNS predictions, the user
should select process {\tt 421} ($W^+$) or {\tt 426} ($W^-$).

\subsection{$W+Q$ production in the combined 4FNS/5FNS, processes 421, 426}
\label{subsec:wbcombined}
These processes represent the production of a $W$ boson and one or more jets,
at least one of which is a $b$-quark, calculated by combining the 4- and 5-flavour results
of processes {\tt 401}, {\tt 411} (for {\tt 421}) and {\tt 406}, {\tt 416} (for {\tt 426}).
The selection of the final state is the same as for processes {\tt 401} and {\tt 406}, as
described in Section~\ref{subsec:wbbfilter}. The procedure for combining the two
calculations is described in refs.~\cite{Campbell:2008hh,Caola:2011pz}.

\subsection{$W+b{\bar b}+$~jet production, processes 431,436}
\label{subsec:wbbjetmassive}

These processes represent the production of a $W$ boson which subsequently
decays leptonically, in association with a $b{\bar b}$ pair and an
additional jet. The effect of the bottom quark mass is included (c.f. the massless approximation
used in processes {\tt 24}, {\tt 29})
and the calculation may be performed at LO only.

When {\tt removebr} is true, the $W$ boson does not decay.

\subsection{$W+t{\bar t}$ processes 500--516}
\label{subsec:wttdecay}

These processes represent the production of a $W^\pm$ boson which subsequently
decays leptonically, in association with a $t{\bar t}$ pair. In all except processes 
{\tt 500} and {\tt 510} the decays of the top and anti-top quark are included.
Processes {\tt 501,502} and {\tt 511,512} refer to the semileptonic decay of the top and antitop quarks,
with the latter process in each pair giving the radiation in the decay of the top and antitop.
Process {\tt 503} ({\tt 513}) refers to the semileptonic decay of the top (antitop)
and the hadronic decay of the antitop (top). Processes {\tt 506}({\tt 516}) gives the semileptonic
decay of the antitop(top) and the hadronic decay of the top(antitop).  Processes {\tt 506}({\tt 516}) 
do not give same sign lepton events, so they may be of less phenomenological importance. For this reason 
we have not yet included radiation in the decay for these processes. 

For processes {\tt 503}, {\tt 506}, {\tt 513}
and {\tt 516} the default behaviour is that the hadronic decay products
are clustered into jets using the supplied jet
algorithm parameters, but no cut is applied on the number of jets.
This behaviour can be altered by changing the value of the
variable {\tt notag} in the file {\tt src/User/setnotag.f}.

The top quarks are always
produced on-shell, which is a necessity for a gauge invariant result
from this limited set of diagrams, but all spin correlations are included.
Switching {\tt zerowidth} from {\tt .true.} to {\tt .false.} only affects
the $W$ bosons (both the directly produced one and from the top quark decay).
Processes {\tt 501} and {\tt 511} may be run at NLO with the option {\tt todk},
including radiation in the decay of the top quark, see section \ref{subsec:ttbar}.






\subsection{$Zt{\bar t}$ production, processes 529-533}
\label{subsec:ztt}

These processes represent the production of a $Z$ boson in association
with a pair of top quarks.
For process {\tt 529}, neither the top quarks nor the $Z$-boson
decays. 
In processes {\tt 530-533}, the top quarks are always
produced on-shell, which is a necessity for a gauge invariant result
from this limited set of diagrams.
Switching {\tt zerowidth} from {\tt .true.} to {\tt .false.} only affects
the $Z$ and the $W$ bosons from the top quark decay.
In process {\tt 530} the $Z$ boson decays into an electron pair, whilst
in {\tt 531} the decay is into a massless bottom quark pair.
In process {\tt 532--533} the $Z$ boson decays into an electron pair, whilst
on or other of the top quark or top anti-quark decays hadronically.
The calculations can be performed at LO only.

For processes {\tt 532} and {\tt 533} the default behaviour is that the hadronic decay products
are clustered into jets using the supplied jet
algorithm parameters, but no cut is applied on the number of jets.
This behaviour can be altered by changing the value of the
variable {\tt notag} in the file {\tt src/User/setnotag.f}.

When {\tt removebr} is true in process {\tt 530}, the top quarks and the $Z$ boson do not decay.

\subsection{$Ht$ and $H\bar{t}$ production, processes 540--557}
[{\it For more details on this calculation, please refer to arXiv:1302.3856}]

\label{subsec:Ht}
These processes describe the production of a single top quark ({\tt 540}, {\tt 544}, {\tt 550},
{\tt 554}) or antiquark ({\tt 541}, {\tt 547}, {\tt 551}, {\tt 557}) by $W$ exchange in the
$t$-channel, in association with a Higgs boson. These processes can be performed at NLO.
For processes {\tt 540}, {\tt 541}, {\tt 550},
{\tt 551}, the top quark does not decay, but the
Higgs boson decays to $b\bar{b}$, ({\tt 540}, {\tt 541}), or to $\gamma \gamma$, ({\tt 550}, {\tt 551}).
Processes {\tt 544}, {\tt 547} and {\tt 554}, {\tt 557} include the decay of the top quark and antiquark
in the approximation in which the top quark is taken to be on shell allowing a clean separation
between production and decay. 

It is possible to study the effects of anomalous couplings of the Higgs boson to the top quark and $W$ bosons. These are parametrized by $c_{t\bar{t}H} = g_{t\bar{t}H}/g_{t\bar{t}H}^{SM}$ and $c_{WWH} = g_{WWH}/g_{WWH}^{SM}$ respectively, so that $c_{t\bar{t}H}=c_{WWH}=1$ in the SM. Different couplings may be chosen by modifying the variables {\tt cttH} and {\tt cWWH} in {\tt src/Need/reader$\_$input.f} and recompiling.

\subsection{$Zt$ and $Z\bar{t}$ production, processes 560--569}\
[{\it For more details on this calculation, please refer to arXiv:1302.3856}]

\label{subsec:Zt}
These processes describe the production of a single top quark (or antiquark) by $W$ exchange in the
$t$-channel, in association with a $Z$ boson. Processes {\tt 560}, {\tt 561},
{\tt 564}, {\tt 567} can be performed at NLO.
Processes {\tt  560}-{\tt 563} are for stable top quarks, whereas processes {\tt 564}-{\tt 569}
include the decay of the top quark and antiquark
in the approximation inwhich the top quark is taken to
be on shell allowing a clean separation
between production and decay.

For processes {\tt 564} and {\tt 567} the default behaviour is that the hadronic decay products
are clustered into jets using the supplied jet
algorithm parameters, but no cut is applied on the number of jets.
This behaviour can be altered by changing the value of the
variable {\tt notag} in the file {\tt src/User/setnotag.f}.

\subsection{$HH$ production, processes 601--602}
These processes represent the production of a pair of Higgs bosons.
The production proceeds through gluon-fusion one-loop diagrams involving loops 
of top quarks. The formulae implemented in the code are taken from ref.~\cite{Glover:1987nx},
where the two Higgs bosons are treated as being on-shell. To enforce this 
condition, the code sets zerowidth to true, overriding the value set in the input file.
The calculation can be performed at LO only, (i.e.\ one-loop order only).
Two decays of the Higgs bosons are currently foreseen, although other decays can easily be implemented. 
In process {\tt 601}, one Higgs boson decays to
a pair of $b$-quarks, and the other decays to a pair of $\tau$'s.
In process {\tt 602}, one Higgs boson decays to
a pair of $b$-quarks, and the other decays to a pair of photons.

\subsection{$Ht{\bar t}$ production, processes 640--660}
\label{subsec:htt}

These processes represent the production of a Higgs boson in association
with a pair of top quarks. The calculation can be performed at LO only.

For process {\tt 640}, neither the top quarks nor the Higgs boson
decays. 
In processes {\tt 641-647}, the top quarks are always
produced on-shell, which is a necessity for a gauge invariant result
from this limited set of diagrams.
Switching {\tt zerowidth} from {\tt .true.} to {\tt .false.} only affects
the Higgs and the $W$ bosons from the top quark decay.
In process {\tt 641} both the top quarks decay leptonically
and the Higgs boson decays into a pair of bottom quarks. 
Consistency with
the simpler process ({\tt 640}) can be demonstrated by running process
{\tt 641} with {\tt removebr} set to true.
In process {\tt 644} the top quark decays leptonically
and the anti-top quark decays hadronically and the Higgs boson decays into a pair of bottom quarks. 
In process {\tt 647} the anti-top quark decays leptonically
and the top quark decays hadronically and the Higgs boson decays into a pair of bottom quarks. 

Processes {\tt 651}--{\tt 657} correspond to processes {\tt 641}--{\tt 647} but with the Higgs decaying
to two photons.
Processes {\tt 661}--{\tt 667} correspond to processes {\tt 641}--{\tt 647} but with the Higgs decaying
to two $W$-bosons which subsequently decay leptonically.

\subsection{Dark Matter Processes  Mono-jet and Mono-photon 800-848} 
[{\it For more details on this calculation, please refer to arXiv:1211.6390}]

This section provides an overview of the Dark Matter (DM) processes
available in \MCFM. Since these processes are quite different in the
range of possible input parameters (reflecting the range of potential
BSM operators) the majority of the new features are controlled by the
file {\tt dm\_parameters.DAT} located in the {\tt Bin} directory.  We
begin this section by describing the inputs in this file.  Note that
these processes are still controlled, as usual by {\tt input.DAT}
which is responsible for selecting the process, order in perturbation
theory, PDFs and phase space cuts etc. The new file controls only the
new BSM parameters in the code.

\begin{itemize} 
\item 
{\tt [dm mass]} This parameter sets the mass of the dark matter particle $m_{\chi}$. 
\item 
{\tt [Lambda]} Controls the mass scale associated with the suppression of the higher dimensional operator in the effective theory approach. Note that each 
operator has a well defined scaling with respect to Lambda, so cross sections and distributions obtained with one particular value can be readily extended to 
determine those with different $\Lambda$. 
\item
{\tt [effective theory] } Is a logical variable which controls whether or not the effective field theory is used in the calculation of the DM process. If this value is set to 
{\tt .false.} then one must specify the mass of the light mediator and its width (see below for more details).
\item
{\tt [Yukawa Scalar couplings]} Is a logical variable which determines if the scalar and pseudo scalar operators scale with the factor $m_{q}/\Lambda$ ( {\tt. .true.}) 
or 1  ({\tt .false.}).  
\item
{ \tt [Left handed DM couplings] } and { \tt [Right handed DM couplings] } 
These variables determine the coupling of the
various flavours of quarks to the DM operators.  The default value is 1. 
Note that the code uses the fact that vector operators scale as
$(L+R)$ and axial operators scale as $(L-R)$ in constructing cross
sections. Therefore you should be careful if modifying these
parameters. For the axial and pseudo scalar operators the code will
set the right-handed couplings to be the negative of the left handed
input couplings (if this is not already the case from the setup) and
inform the user it has done so. The most likely reason to want to
change these values is to inspect individual flavour operators
separately, i.e.\ to investigate an operator which only couples to up
quarks one would set all couplings to 0d0 apart from {\tt [up type]}
which would be left as 1d0.
\item 
{\tt [mediator mass]} If {\tt [effective theory]} is set to {\tt .false.} this variable controls the mass of the mediating particle.
\item 
{\tt [mediator width]} If {\tt [effective theory]} is set to {\tt .false.} this variable controls the width of the mediating particle 
\item 
{\tt [g\_x]} If {\tt [effective theory]} is set to {\tt .false.} this variable controls the coupling of the mediating particle to the DM.
\item 
{\tt [g\_q]} If {\tt [effective theory]} is set to {\tt .false.} this variable controls the coupling of the mediating particle to the quarks.
\end{itemize}

We now discuss some details of the specific DM process.

\begin{itemize}
\item 
Processes 800 and 820 produce the 
mono-jet or mono-photon signature through the following vector operator, 
\begin{eqnarray}
\mathcal{O}_V&=&\frac{(\overline{\chi}\gamma_{\mu}\chi)(\overline{q}\gamma^{\mu}q)}{\Lambda^2}~,\label{eq:OV}  
\end{eqnarray}
These processes are available at NLO and include the usual treatment of photons. See for instance the $V\gamma$ processes ($\sim$ 300) in this 
manual for more details on photon setup in \MCFM. As discussed above the code will calculate left and right-handed helicity amplitudes and build the 
vector operators from $(L+R)$. Therefore you should ensure that the Left and right-handed couplings are equal in  {\tt dm\_parameters.DAT}. 
Processes 840 and 845 represent the production of DM plus two jets or DM plus one jet and one photon and are available at LO. 
\item 
Processes 801 and 821 produce the 
mono-jet or mono-photon signature through the following axial-vector operator, 
\begin{eqnarray}
\mathcal{O}_A&=&\frac{(\overline{\chi}\gamma_{\mu}\gamma_5\chi)(\overline{q}\gamma^{\mu}\gamma_5q)}{\Lambda^2}~,\label{eq:OA}
\end{eqnarray}
These processes are available at NLO and include the usual treatment
of photons. See for instance the $V\gamma$ processes ($\sim$ 300) in
this manual for more details on photon setup in \MCFM. As discussed
above the code will calculate left and right-handed helicity
amplitudes and build the axial vector operators from $(L-R)$. By
default the code will enforce the right handed couplings to equal to
the negative of the left handed couplings, if this is not
already the case in {\tt dm\_parameters.DAT}. Therefore the user does
not have to change this file when switching between vector and axial
vector operators.  Processes 841 and 846 represent the production of
DM plus two jets or DM plus one jet and one photon and are available
at LO.
\item 
Processes 802 and 822 produce the 
mono-jet or mono-photon signature through the following scalar operator, 
\begin{eqnarray}
\mathcal{O}_S&=&\frac{\Delta(\overline{\chi}\chi)(\overline{q}q)}{\Lambda^2}~,
\end{eqnarray}
These processes are available at NLO and include the usual treatment
of photons. See for instance the $V\gamma$ processes ($\sim$ 300) in
this manual for more details on photon setup in \MCFM. As discussed
above the code will calculate left and right-handed helicity
amplitudes and build the vector operators from $(L+R)$. Therefore you
should ensure that the Left and right-handed couplings are equal in
{\tt dm\_parameters.DAT}. For these processes $\Delta$ is fixed from
the value of {\tt [Yukawa Scalar Couplings] } if this is {\tt .true.}
then $\Delta=m_q/\Lambda$ else $\Delta=1$.

Processes 842 and 847 represent the production of DM plus two jets or DM plus one jet and one photon and are available at LO. 
\item 
Processes 803 and 823 produce the 
mono-jet or mono-photon signature through the following pseudo-scalar operator, 
\begin{eqnarray}
\mathcal{O}_{PS}&=&\frac{m_q(\overline{\chi}\gamma_5\chi)(\overline{q}\gamma_5q)}{\Lambda^3}\label{eq:OPS}~.
\end{eqnarray}
These processes are available at NLO and include the usual treatment
of photons. See for instance the $V\gamma$ processes ($\sim$ 300) in
this manual for more details on photon setup in \MCFM. As discussed
above the code will calculate left and right-handed helicity
amplitudes and build the pseudo scalar operators from $(L-R)$. By
default the code will enforce the right handed couplings to equal to
the negative of the left handed couplings, if this is not
already the case in {\tt dm\_parameters.DAT}. Therefore the user does
not have to change this file when switching between scalar and pseudo
scalar operators.  Processes 841 and 846 represent the production of
DM plus two jets or DM plus one jet and one photon and are available
at LO.  For these processes $\Delta$ is fixed from the value of {\tt
  [Yukawa Scalar Couplings] } if this is {\tt .true.} then
$\Delta=m_q/\Lambda$ else $\Delta=1$.

Processes 843 and 848 represent the production of DM plus two jets or DM plus one jet and one photon and are available at LO. 

\item 
Process 804 produces the 
mono-jet signature through the following gluon induced operator, 
\begin{eqnarray}
\mathcal{O}_g&=&\alpha_s\frac{(\chi\overline{\chi})(G^{\mu\nu}_aG_{a,\mu\nu})}{\Lambda^3}~,
\end{eqnarray}
This process is available at NLO. Process 844 represents the
production of DM plus two jets and is available at LO. Since this
operator is higher dimensional, extensions to a theory in which there
is a light mediator requires the definition of two new scales (one for
the EFT in the loop defining the operator). In this version we
therefore do not consider in a theory with a light mediator.
\item 
Process 805 is a separate case of the scalar operator for top quarks
\begin{eqnarray}
\mathcal{O}^{m_t}_S&=&\frac{m_t(\overline{\chi}\chi)(\overline{q}q)}{\Lambda^3}~,
\end{eqnarray}
This process is available at LO and proceeds through a gluon loop. 
\end{itemize}

\section*{Acknowledgments}
We are happy to acknowledge Fabrizio Caola, Heribertus Hartanto, Fabio
Maltoni, Raoul R{\"o}ntsch, Gavin Salam, Francesco Tramontano and
Giulia Zanderighi for their contributions to the code.

The original histogramming package mbook.f included in the code is due
to Michelangelo Mangano.  The integration routine vegas.f is due to
Peter Lepage.  The routine XLUDecomp is taken from looptools and is
adapted from Michael Rauch.



\appendix
\section{\MCFM references}
\label{MCFMrefs}

As general references for NLO computations with MCFM, please use:
\begin{itemize}
\item J.~M.~Campbell and R.~K.~Ellis, \\
  {\it ``An update on vector boson pair production at hadron colliders,''} \\
  Phys.\ Rev.\ D {\bf 60}, 113006 (1999)
  [arXiv:hep-ph/9905386].
\item J.~M.~Campbell, R.~K.~Ellis and C.~Williams, \\
  {\it ``Vector boson pair production at the LHC,''} \\
  JHEP {\bf 1107}, 018 (2011)
  [arXiv:1105.0020 [hep-ph]]. 
\item J.~M.~Campbell, R.~K.~Ellis and W.~Giele, \\
  {\it ``A Multi-Threaded Version of MCFM''}, \\
    EPJ {\bf C75}, 246 (2015)
    [arXiv:1503.06182 [hep-ph]].

\end{itemize}

When using MCFM 8.0 for NNLO calculations of color-singlet processes please refer to:
\begin{itemize}
\item 
  R.~Boughezal, J.~M.~Campbell, R.~K.~Ellis, \\
   C.~Focke, W.~Giele, X.~Liu,~F. Petriello and  C.~Williams, \\
  {\it ``Color singlet production at NNLO in MCFM''},
  arXiv:1605.08011.
\end{itemize}

Additional references to the program may be used depending on the process under study. The relevant papers are:
\begin{itemize}
\item J.~M.~Campbell and R.~K.~Ellis, \\
  {\it ``Radiative corrections to Z b anti-b production,''} \\
  Phys.\ Rev.\ D {\bf 62}, 114012 (2000)
  [arXiv:hep-ph/0006304].
\item J.~Campbell and R.~K.~Ellis, \\
  {\it ``Next-to-leading order corrections to W + 2jet and Z + 2jet production  at
   hadron colliders,''} \\
  Phys.\ Rev.\ D {\bf 65}, 113007 (2002)
  [arXiv:hep-ph/0202176].
\item J.~Campbell, R.~K.~Ellis, F.~Maltoni and S.~Willenbrock, \\
  {\it ``Higgs boson production in association with a single bottom quark,''} \\
  Phys.\ Rev.\ D {\bf 67}, 095002 (2003)
  [arXiv:hep-ph/0204093].
\item J.~Campbell, R.~K.~Ellis and D.~L.~Rainwater, \\
  {\it ``Next-to-leading order QCD predictions for W + 2jet and Z + 2jet  production
     at the CERN LHC,''} \\
  Phys.\ Rev.\ D {\bf 68}, 094021 (2003)
  [arXiv:hep-ph/0308195].
\item J.~Campbell, R.~K.~Ellis, F.~Maltoni and S.~Willenbrock, \\
  {\it ``Associated production of a Z boson and a single heavy-quark jet,''} \\
  Phys.\ Rev.\ D {\bf 69}, 074021 (2004)
  [arXiv:hep-ph/0312024].
\item E.~L.~Berger and J.~Campbell, \\
  {\it ``Higgs boson production in weak boson fusion at next-to-leading order,''} \\
  Phys.\ Rev.\ D {\bf 70}, 073011 (2004)
  [arXiv:hep-ph/0403194].
\item J.~Campbell, R.~K.~Ellis and F.~Tramontano, \\
  {\it ``Single top production and decay at next-to-leading order,''} \\
  Phys.\ Rev.\ D {\bf 70}, 094012 (2004)
  [arXiv:hep-ph/0408158].
\item J.~Campbell and F.~Tramontano, \\
  {\it ``Next-to-leading order corrections to Wt production and
  decay,''} \\
  Nucl.\ Phys.\ B {\bf 726}, 109 (2005)
  [arXiv:hep-ph/0506289].
\item J.~Campbell, R.K.~Ellis, F.~Maltoni and S.~Willenbrock, \\
  {\it ``Production of a $Z$ boson and two jets with one heavy quark tag,''} \\
  Phys.\ Rev.\ D {\bf 73}, 054007 (2006)
  [arXiv:hep-ph/0510362].
\item J.~M.~Campbell, R.~Frederix, F.~Maltoni and F.~Tramontano,\\
  {\it ``$t$-channel single top production at hadron colliders,''} \\ 
  Phys. Rev. Lett. {\bf 102} (2009) 182003,
  [arXiv:0903.0005 [hep-ph]].
\item J.~M.~Campbell, R.~K.~Ellis and G.~Zanderighi, \\
  {\it ``Next-to-leading order Higgs~$+~2$~jet production via gluon fusion,''} \\
  JHEP {\bf 0610}, 028 (2006)
  [arXiv:hep-ph/0608194].
\item J.~Campbell, R.K.~Ellis, F.~Maltoni and S.~Willenbrock, \\
  {\it ``Production of a $W$ boson and two jets with one $b$-quark
  tag,''} \\
  Phys.\ Rev.\ D {\bf 75}, 054015 (2007)
  [arXiv:hep-ph/0611348].
\item J.~M.~Campbell, R.~K.~Ellis, F.~Febres Cordero, F.~Maltoni, L.~Reina, D.~Wackeroth and S.~Willenbrock, \\
  {\it ``Associated Production of a $W$ Boson and One $b$ Jet,''} \\
  Phys.\ Rev.\  D {\bf 79}, 034023 (2009)
  [arXiv:0809.3003 [hep-ph]].
\item J.~M.~Campbell, R.~K.~Ellis and C.~Williams, \\
  {\it ``Hadronic production of a Higgs boson and two jets at next-to-leading order,''} \\
   Phys.\ Rev.\ D {\bf 81} 074023 (2010),
  [arXiv:1001.4495 [hep-ph]].
\item S.~Badger, J.~M.~Campbell and R.~K.~Ellis, \\
  {\it ``QCD corrections to the hadronic production of a heavy quark pair and a  W-boson including decay correlations,''} \\
 JHEP {\bf 1103}, 027 (2011)
  [arXiv:1011.6647 [hep-ph]].
\item F.~Caola, J.~M.~Campbell, F.~Febres Cordero, L.~Reina and D.~Wackeroth, \\
  {\it ``NLO QCD predictions for $W+1$ jet and $W+2$ jet production with at least one b jet at the 7 TeV LHC,''} \\
    arXiv:1107.3714 [hep-ph].
\item J.~M.~Campbell, R.~K.~Ellis and C.~Williams, \\
  {\it ``Gluon-gluon contributions to W+ W- production and Higgs interference effects,''} \\
  JHEP {\bf 1110}, 005 (2011),
  arXiv:1107.5569 [hep-ph].
\item J.~M.~Campbell and R.~K.~Ellis, \\
  {\it ``Top-quark processes at NLO in production and decay,''} \\
  arXiv:1204.1513 [hep-ph], FERMILAB-PUB-12-078-T.
%\cite{Campbell:2012dh}
\item J.~M.~Campbell and R.~K.~Ellis, \\
  {\it ``$t \bar{t} W^{\pm}$ production and decay at NLO,''} \\
  JHEP {\bf 1207}, 052 (2012), [arXiv:1204.5678 [hep-ph]].
\item
 J.~M.~Campbell, H.~B.~Hartanto and C.~Williams\\
  {\it {``Next-to-leading order predictions for $Z \gamma$+jet and
          Z $\gamma \gamma$ final states at the LHC"}}
   JHEP {\bf 1211}, 162 (2012), [arXiv:1208.0566 [hep-ph]].	
\item
 P.~J.~Fox  and C.~Williams\\
      {\it {``Next-to-leading order predictions for Dark Matter Production at Hadron Colliders"}}
                        	 [arXiv:1211.6390 [hep-ph]].	 
\item
  J.~M.~Campbell, R.~K.~Ellis and R.~R{\"o}ntsch,
  {\it ``Single top production in association with a Z boson at the LHC,''}
  arXiv:1302.3856 [hep-ph].

\item
 J.~M.~Campbell, R.~K.~Ellis and C.~Williams,
 {\it ``Associated Production of a Higgs Boson at NNLO''},
  arXiv:1601.00658 [hep-ph].
  
\item 
 J.~M.~Campbell, R.~K.~Ellis, Ye~Li and C.~Williams,
  {\it ``Predictions for Diphoton Production at the LHC through NNLO in QCD''},
  arXiv:1603.02663 [hep-ph].  
\end{itemize}

The following publications have also made use of calculations
implemented in MCFM, but the corresponding code has not yet been made
public. Versions of the code that contain these calculations will be
released in the future.

\begin{itemize}
\item J.~Campbell, F.~Maltoni and F.~Tramontano, \\
  {\it ``QCD corrections to $J/\psi$ and $\Upsilon$ production
  at hadron colliders,''} \\
  Phys. Rev. Lett. {\bf 98}, 252002 (2007)
  [arXiv:hep-ph/0703113].
\item R.~K.~Ellis, K.~Melnikov and G.~Zanderighi,\\
  {\it ``Generalized unitarity at work: first NLO QCD results for hadronic $W+$~3 jet production,''}
  arXiv:0901.4101 [hep-ph].
\end{itemize}

\clearpage
\section{Processes included in MCFM}
\label{MCFMprocs}
\begin{table}
\begin{center}
\hspace*{-1.5cm}
\begin{tabular}{|l|l|l|}
\hline
{\tt nproc} & $f(p_1)+f(p_2) \to \ldots $& Order \\
\hline
%  
1  & $ W^+(\to \nu(p_{3})+e^+(p_{4}))$   & NNLO \\
6  & $ W^-(\to e^-(p_{3})+\bar{\nu}(p_{4}))$   & NNLO \\
\hline 
11 & $ W^+(\to \nu(p_{3})+e^+(p_{4}))+f(p_{5})$    & NLO \\
12 & $ W^+(\to \nu(p_{3})+e^+(p_{4}))+\bar{b}(p_{5})$   & NLO \\
13 & $ W^+(\to \nu(p_{3})+e^+(p_{4}))+\bar{c}(p_{5})$   & NLO \\
14 & $ W^+(\to \nu(p_{3})+e^+(p_{4}))+\bar{c}(p_{5}) [\mbox{massless}]$   & LO \\
16 & $ W^-(\to e^-(p_{3})+\bar{\nu}(p_{4}))+f(p_{5})$   & NLO \\
17 & $ W^-(\to e^-(p_{3})+\bar{\nu}(p_{4}))+b(p_{5})$   & NLO \\
18 & $ W^-(\to e^-(p_{3})+\bar{\nu}(p_{4}))+c(p_{5})$   & NLO \\
19 & $ W^-(\to e^-(p_{3})+\bar{\nu}(p_{4}))+c(p_{5}) [\mbox{massless}]$   & LO \\
\hline 
20 & $ W^+(\to \nu(p_{3})+e^+(p_{4})) +b(p_{5})+\bar{b}(p_{6}) [\mbox{massive}]$   & NLO \\
21 & $ W^+(\to \nu(p_{3})+e^+(p_{4})) +b(p_{5})+\bar{b}(p_{6})$   & NLO \\
22 & $ W^+(\to \nu(p_{3})+e^+(p_{4})) +f(p_{5})+f(p_{6})$   & NLO \\
23 & $ W^+(\to \nu(p_{3})+e^+(p_{4})) +f(p_{5})+f(p_{6})+f(p_{7})$   & LO \\
24 & $ W^+(\to \nu(p_{3})+e^+(p_{4})) +b(p_{5})+\bar{b}(p_{6})+f(p_{7})$   & LO \\
25 & $ W^-(\to e^-(p_{3})+\bar{\nu}(p_{4})) +b(p_{5})+\bar{b}(p_{6}) [\mbox{massive}]$   & NLO \\
26 & $ W^-(\to e^-(p_{3})+\bar{\nu}(p_{4})) +b(p_{5})+\bar{b}(p_{6})$   & NLO \\
27 & $ W^-(\to e^-(p_{3})+\bar{\nu}(p_{4})) +f(p_{5})+f(p_{6})$   & NLO \\
28 & $ W^-(\to e^-(p_{3})+\bar{\nu}(p_{4})) +f(p_{5})+f(p_{6})+f(p_{7})$   & LO \\
29 & $ W^-(\to e^-(p_{3})+\bar{\nu}(p_{4})) +b(p_{5})+\bar{b}(p_{6})+f(p_{7})$   & LO \\
\hline 
31 & $ Z(\to e^-(p_{3})+e^+(p_{4}))$   & NNLO \\
32 & $ Z(\to 3\times(\nu(p_{3})+\bar{\nu}(p_{4})))$   & NNLO \\
33 & $ Z(\to b(p_{3})+\bar{b}(p_{4}))$   & NLO \\
34 & $ Z(\to 3\times(d(p_{5})+\bar{d}(p_{6})))$   & NLO \\
35 & $ Z(\to 2\times(u(p_{5})+\bar{u}(p_{6})))$   & NLO \\
36 & $  Z \to  t(\to \nu(p_{3})+e^+(p_{4})+b(p_{5}))+\bar{t}(\to \bar{b}(p_{6})+e^-(p_{7})+\bar{\nu}(p_{8}))$   & LO \\
\hline 
41 & $ Z(\to e^-(p_{3})+e^+(p_{4}))+f(p_{5})$   & NLO \\
42 & $ Z_0(\to 3\times(\nu(p_{3})+\bar{\nu}(p_{4})))+f(p_{5})$   & NLO \\
43 & $ Z(\to b(p_{3})+\bar{b}(p_{4}))+f(p_{5})$   & NLO \\
\hline 
44 & $ Z(\to e^-(p_{3})+e^+(p_{4}))+f(p_{5})+f(p_{6})$   & NLO \\
45 & $ Z(\to e^-(p_{3})+e^+(p_{4}))+f(p_{5})+f(p_{6})+f(p_{7})$   & LO \\
46 & $ Z(\to 3\times(\nu(p_{3})+\bar{\nu}(p_{4}))+f(p_{5})+f(p_{6})$   & NLO \\
47 & $ Z(\to 3\times(\nu(p_{3})+\bar{\nu}(p_{4}))+f(p_{5})+f(p_{6})+f(p_{7})$   & LO \\
\hline 
50 & $ Z(\to e^-(p_{3})+e^+(p_{4}))+\bar{b}(p_{5})+b(p_{6}) [\mbox{massive}]$   & LO \\
51 & $ Z(\to e^-(p_{3})+e^+(p_{4}))+b(p_{5})+\bar{b}(p_{6})$   & NLO \\
52 & $ Z_0(\to 3\times(\nu(p_{3})+\bar{\nu}(p_{4})))+b(p_{5})+\bar{b}(p_{6})$   & NLO \\
53 & $ Z(\to b(p_{3})+\bar{b}(p_{4}))+b(p_{5})+\bar{b}(p_{6})$   & NLO \\
54 & $ Z(\to e^-(p_{3})+e^+(p_{4}))+b(p_{5})+\bar{b}(p_{6})+f(p_{7})$   & LO \\
\hline 
\end{tabular}
\end{center}
\end{table}
\newpage
\begin{table}
\begin{center}
\begin{tabular}{|l|l|l|}
\hline
56 & $ Z(\to e^-(p_{3})+e^+(p_{4}))+c(p_{5})+\bar{c}(p_{6})$   & NLO \\
\hline 
61 & $ W^+(\to \nu(p_{3})+e^+(p_{4})) +W^-(\to e^-(p_{5})+\bar{\nu}(p_{6}))$   & NLO \\
62 & $ W^+(\to \nu(p_{3})+e^+(p_{4})) +W^-(\to q(p_{5})+\bar{q}(p_{6}))$   & NLO \\
63 & $ W^+(\to \nu(p_{3})+e^+(p_{4})) +W^-(\to q(p_{5})+\bar{q}(p_{6}))[\mbox{rad.in.dk}]$   & NLO \\
64 & $ W^-(\to e^-(p_{3})+\bar{\nu}(p_{4})) W^+(\to  q(p_{5})+ \bar{q}(p_{6}))$   & NLO \\
65 & $ W^-(\to e^-(p_{3})+\bar{\nu}(p_{4})) W^+(\to  q(p_{5})+ \bar{q}(p_{6}))[\mbox{rad.in.dk}]$   & NLO \\
66 & $ W^+(\to \nu(p_{3})+e^+(p_{4})) +W^-(\to e^-(p_{5})+\bar{\nu}(p_{6}))+f(p_{7})$   & LO \\
69 & $ W^+(\to \nu(p_{3})+e^+(p_{4})) +W^-(\to e^-(p_{5})+\bar{\nu}(p_{6})) [\mbox{no pol}]$   & LO \\
\hline 
71 & $ W^+(\to \nu(p_{3})+\mu^+(p_{4}))+Z(\to e^-(p_{5})+e^+(p_{6}))$   & NLO \\
72 & $ W^+(\to \nu(p_{3})+\mu^+(p_{4}))+Z(\to 3\times(\nu_e(p_{5})+\bar{\nu}_e(p_{6})))$   & NLO \\
73 & $ W^+(\to \nu(p_{3})+\mu^+(p_{4}))+Z(\to b(p_{5})+\bar{b}(p_{6}))$   & NLO \\
74 & $ W^+(\to \nu(p_{3})+\mu^+(p_{4}))+Z(\to 3\times(d(p_{5})+\bar{d}(p_{6})))$   & NLO \\
75 & $ W^+(\to \nu(p_{3})+\mu^+(p_{4}))+Z(\to 2\times(u(p_{5})+\bar{u}(p_{6})))$   & NLO \\
\hline 
76 & $ W^-(\to \mu^-(p_{3})+\bar{\nu}(p_{4}))+Z(\to e^-(p_{5})+e^+(p_{6}))$   & NLO \\
77 & $ W^-(\to e^-(p_{3})+\bar{\nu}(p_{4}))+Z(\to 3\times(\nu_e(p_{5})+\bar{\nu}_e(p_{6})))$   & NLO \\
78 & $ W^-(\to e^-(p_{3})+\bar{\nu}(p_{4}))+Z(\to b(p_{5})+\bar{b}(p_{6}))$   & NLO \\
79 & $ W^-(\to e^-(p_{3})+\bar{\nu}(p_{4}))+Z(\to 3\times(d(p_{5})+\bar{d}(p_{6})))$   & NLO \\
80 & $ W^-(\to e^-(p_{3})+\bar{\nu}(p_{4}))+Z(\to 2\times(u(p_{5})+\bar{u}(p_{6})))$   & NLO \\
\hline 
81 & $ Z(\to e^-(p_{3})+e^+(p_{4})) + Z(\to \mu^-(p_{5})+\mu^+(p_{6}))$   & NLO \\
82 & $ Z(\to e^-(p_{3})+e^+(p_{4})) + Z(\to 3\times(\nu(p_{5})+\bar{\nu}(p_{6})))$   & NLO \\
83 & $ Z(\to e^-(p_{3})+e^+(p_{4})) + Z(\to b(p_{5})+\bar{b}(p_{6}))$   & NLO \\
84 & $ Z(\to b(p_{3})+\bar{b}(p_{4})) + Z(\to 3\times(\nu(p_{5})+\bar{\nu}(p_{6})))$   & NLO \\
85 & $ Z(\to e^-(p_{3})+e^+(p_{4})) + Z(\to 3\times(\nu(p_{5})+\bar{\nu}(p_{6})))+f(p_{7})$   & LO \\
\hline 
86 & $ Z(\to \mu^-(p_{3})+\mu^+(p_{4}))+Z(\to e^-(p_{5})+e^+(p_{6}))[\mbox{no gamma*}]$   & NLO \\
87 & $ Z(\to e^-(p_{3})+e^+(p_{4})) + Z(\to 3\times(\nu(p_{5})+\bar{\nu}(p_{6}))) [\mbox{no gamma*}]$   & NLO \\
88 & $ Z(\to e^-(p_{3})+e^+(p_{4}))+Z(\to b(p_{5})+\bar{b}(p_{6})) [\mbox{no gamma*}]$   & NLO \\
89 & $ Z(\to b(p_{3})+\bar{b}(p_{4})) + Z(\to 3\times(\nu(p_{5})+\bar{\nu}(p_{6}))) [\mbox{no gamma*}]$   & NLO \\
90 & $ Z(\to e^-(p_{3})+e^+(p_{4})) + Z(\to e^-(p_{5})+e^+(p_{6}))$   & NLO \\
\hline 
91 & $ W^+(\to \nu(p_{3})+e^+(p_{4})) + H(\to b(p_{5})+\bar{b}(p_{6}))$   & NNLO \\
92 & $ W^+(\to \nu(p_{3})+e^+(p_{4})) + H(\to W^+(\nu(p_{5}),e^+(p_{6}))W^-(e^-(p_{7}),\bar{\nu}(p_{8})))$   & NNLO \\
93 & $ W^+(\to \nu(p_3)+e^+(p_{4})) + H(\to Z(e^-(p_{5}),e^+(p_{6}))+Z(\mu^-(p_{7}),\mu(p_{8})))$ & NNLO \\
94 & $ W^+(\to \nu(p_3)+e^+(p_{4})) + H(\to \gamma(p_{5})+\gamma(p_{6})$ & NNLO \\
96 & $ W^-(\to e^-(p_{3})+\bar{\nu}(p_{4})) + H(\to b(p_{5})+\bar{b}(p_{6}))$    & NNLO \\
97 & $ W^-(\to e^-(p_{3})+\bar{\nu}(p_{4})) + H(\to W^+(\nu(p_{5}),e^+(p_{6}))W^-(e^-(p_{7}),\bar{\nu}(p_{8})))$   & NNLO \\
98 & $ W^-(\to e^-(p_3)+\bar{\nu}(p_{4})) + H(\to Z(e^-(p_{5}),e^+(p_{6}))+Z(\mu^-(p_{7}),\mu^+(p_{8})))$ & NNLO \\
99 & $ W^-(\to e^-(p_3)+\bar{\nu}(p_{4})) + H(\to \gamma(p_{5})+\gamma(p_{6}))$ & NNLO \\

\hline 
\end{tabular}
\end{center}
\end{table}
\newpage
\begin{table}
\begin{center}
\begin{tabular}{|l|l|l|}
\hline
101 & $ Z(\to e^-(p_{3})+e^+(p_{4})) + H(\to b(p_{5})+\bar{b}(p_{6}))$   & NNLO \\
102 & $ Z(\to 3\times(\nu(p_{3})+\bar{\nu}(p_{4}))) + H(\to b(p_{5})+\bar{b}(p_{6}))$   & NLO \\
103 & $ Z(\to b(p_{3})+\bar{b}(p_{4})) + H(\to b(p_{5})+\bar{b}(p_{6}))$        & NLO \\
104 & $ Z(\to e^-(p_3)+e^+(p_{4})) + H(\to \gamma(p_{5})+\gamma(p_{6}))$ & NNLO \\
105 & $ Z(\to \to 3\times(\nu(p_3)+\bar{\nu}(p_{4}))) + H(\to \gamma(p_{5})+\gamma(p_{6}))$ & NLO \\
106 & $ Z(\to e^-(p_{3})+e^+(p_{4})) + H(\to W^+(\nu(p_{5}),e^+(p_{6}))W^-(e^-(p_{7}),\bar{\nu}(p_{8})))$   & NNLO \\
107 & $ Z(\to 3\times(\nu(p_{3})+\bar{\nu}(p_{4}))) + H(\to W^+(\nu(p_{5}),e^+(p_{6}))W^-(e^-(p_{7}),\bar{\nu}(p_{8})))$   & NLO \\
108 & $ Z(\to b(p_{3})+\bar{b}(p_{4})) + H(\to W^+(\nu(p_{5}),e^+(p_{6}))W^-(e^-(p_{7}),\bar{\nu}(p_{8})))$        & NLO \\
109 & $ Z(\to e^-(p_3)+e^+(p_{4})) + H(\to Z(e^-(p_{5}),e^+(p_{6}))+Z(\to\mu^-(p_{7}),\mu^+(p_{8})))$ & NLO \\
110 & $ Z(\to e^-(p_3)+e^+(p_{4})) + H(\to \tau^-(p_5) \tau^+_(p_6)))$ & NNLO \\
\hline 
111 & $ H(\to b(p_{3})+\bar{b}(p_{4}))$   & NNLO \\
112 & $ H(\to \tau^-(p_{3})+\tau^+(p_{4}))$   & NNLO \\
113 & $ H(\to  W^+(\nu(p_{3})+e^+(p_{4})) + W^-(e^-(p_{5})+\bar{\nu}(p_{6})))$   & NLO \\
114 & $ H(\to  W^+(\nu(p_{3})+e^+(p_{4})) + W^-(q(p_{5})+\bar{q}(p_{6})))$   & NLO \\
115 & $ H(\to  W^+(\nu(p_{3})+e^+(p_{4})) + W^-(q(p_{5})+\bar{q}(p_{6}))) [rad.in.dk]$   & NLO \\
116 & $ H(\to Z(\to e^-(p_{3})+e^+(p_{4})) + Z(\to\mu^-(p_{5})+\mu^+(p_{6}))$   & NLO \\
117 & $ H(\to Z(\to3\times(\nu(p_{3})+\bar{\nu}(p_{4})))+ Z(\to\mu^-(p_{5})+\mu^+(p_{6}))$   & NLO \\
118 & $ H(\to Z(\to\mu^-(p_{3})+\mu^+(p_{4})) + Z(\to b(p_{5})+\bar{b}(p_{6}))$   & NLO \\
119 & $ H(\to \gamma(p_{3})+\gamma(p_{4}))$   & NNLO \\
120 & $ H(\to Z(\to\mu^-(p_{3})+\mu^+(p_{4})) + \gamma(p_{5}))$   & NLO \\
121 & $ H(\to Z(\to3\times(\nu(p_{3})+\bar{\nu}(p_{4})))) + \gamma(p_{5}))$   & NLO \\
\hline 
123 & $ H(\to  W^+(\nu(p_3)+e^+(p_{4})) + W^-(e^-(p_{5})+\bar{\nu}(p_{6})))$ [t, b loops, exact] & LO \\
124 & $ H(\to  W^+(\nu(p_3)+e^+(p_{4})) + W^-(e^-(p_{5})+\bar{\nu}(p_{6})))$ [only H, gg$\rightarrow$WW int & LO \\
125 & $ H(\to  W^+(\nu(p_3)+e^+(p_{4})) + W^-(e^-(p_{5})+\bar{\nu}(p_{6})))$ [$|H|^2$ and H,gg$\rightarrow$WW int] & LO \\
126 & $ W^+(\nu(p_3)+e^+(p_{4})) + W^-(e^-(p_{5})+\bar{\nu}(p_{6}))$ [gg only, (H + gg$\rightarrow$WW) squared] & LO \\
\hline 
128 & $ H(\to Z(\to \to e^-(p3)+e^+(p4)) + Z(\to \mu^-(p5)+\mu^+(p6))$ [t, b loops, exact]& LO \\
129 & $ H(\to Z(\to e^-(p3)+e^+(p4)) + Z(\to \mu^-(p5)+\mu^+(p6))$ [only H, gg$\rightarrow$ZZ int] & LO \\
130 & $ H(\to Z(\to e^-(p3)+e^+(p4)) + Z(\to \mu^-(p5)+\mu^+(p6))$ [$|H|^2$ and H,gg$\rightarrow$ZZ int]& LO \\
131 & $ Z(\to e^-(p3)+e^+(p4)) + Z(\to \mu^-(p5)+\mu^+(p6)$ [gg only, $|H + gg \rightarrow ZZ|^2$]& LO \\
132 & $ Z(\to e^-(p3)+e^+(p4)) + Z(\to \mu^-(p5)+\mu^+(p6)$ [(gg$\rightarrow$ZZ) squared]& LO \\
133 & $ H(\to Z(\to e^-(p3)+e^+(p4)) + Z(\to \mu^-(p5)+\mu^+(p6) + f(p7))$ [intf,no $p_7$ cut]& LO \\
\hline 
136 & $ H(\to b(p_{3})+\bar{b}(p_{4})) + b(p_{5}) (+g(p_{6}))$   & NLO \\
137 & $ H(\to b(p_{3})+\bar{b}(p_{4})) + \bar{b}(p_{5}) (+b(p_{6}))$   & (REAL) \\
138 & $ H(\to b(p_{3})+\bar{b}(p_{4})) + b(p_{5}) + \bar{b}(p_{6}) [\mbox{both observed}]$   & (REAL) \\
\hline 
\end{tabular}
\end{center}
\end{table}
\newpage
\begin{table}
\begin{center}
\begin{tabular}{|l|l|l|}
\hline
141 & $ t(\to \nu(p_{3})+e^+(p_{4})+b(p_{5}))+\bar{t}(\to b~(p_{6})+e^-(p_{7})+\bar{\nu}(p_{8}))$ & NLO \\
142 & $ t(\to \nu(p_{3})+e^+(p_{4})+b(p_{5}))+\bar{t}(\to b~(p_{6})+e^-(p_{7})+\bar{\nu}(p_{8}))$ \mbox{\small [rad.in.dk]}& NLO \\
143 & $ t(\to \nu(p_{3})+e^+(p_{4})+b(p_{5}))+\bar{t}(\to b~(p_{6})+e^-(p_{7})+\bar{\nu}(p_{8}))+f(p_{9})$ & LO \\
144 & $ t(\to \nu(p_{3})+e^+(p_{4})+b(p_{5}))+\bar{t}(\to b~(p_{6})+e^-(p_{7})+\bar{\nu}(p_{8}))$ \mbox{(uncorr)} & NLO \\
145 & $ t(\to \nu(p_{3})+e^+(p_{4})+b(p_{5}))+\bar{t}(\to b~(p_{6})+e^-(p_{7})+\bar{\nu}(p_{8}))$ \mbox{\small [rad.in.dk],uncorr} & NLO \\
146 & $ t(\to \nu(p_{3})+e^+(p_{4})+b(p_{5}))+\bar{t}(\to b~(p_{6})+q(p_{7})+\bar{q}(p_{8})) $ & NLO \\
147 & $ t(\to \nu(p_{3})+e^+(p_{4})+b(p_{5}))+\bar{t}(\to b~(p_{6})+q(p_{7})+\bar{q}(p_{8})) $ \mbox{\small [rad.in.top.dk]}& NLO \\
148 & $ t(\to \nu(p_{3})+e^+(p_{4})+b(p_{5}))+\bar{t}(\to b~(p_{6})+q(p_{7})+\bar{q}(p_{8})) $ \mbox{\small [rad.in.$W$.dk]}& NLO \\
149 & $ t(\to q(p_{3})+\bar{q}(p_{4})+b(p_{5}))+\bar{t}(\to b~(p_{6})+e^-(p_{7})+\bar{\nu}(p_{8})) $ & NLO \\
150 & $ t(\to q(p_{3})+\bar{q}(p_{4})+b(p_{5}))+\bar{t}(\to b~(p_{6})+e^-(p_{7})+\bar{\nu}(p_{8})) $ \mbox{\small [rad.in.top.dk]}& NLO \\
151 & $ t(\to q(p_{3})+\bar{q}(p_{4})+b(p_{5}))+\bar{t}(\to b~(p_{6})+e^-(p_{7})+\bar{\nu}(p_{8})) $ \mbox{\small [rad.in.$W$.dk]}& NLO \\
\hline 
157 & $ t \bar{t} [\mbox{for total Xsect}]$   & NLO \\
158 & $ b \bar{b} [\mbox{for total Xsect}]$   & NLO \\
159 & $ c \bar{c} [\mbox{for total Xsect}]$   & NLO \\
160 & $ t \bar{t} + g [\mbox{for total Xsect}]$   & LO \\
\hline 
161 & $ t(\to \nu(p_{3})+e^+(p_{4})+b(p_{5}))+q(p_{6}) [\mbox{t-channel}]$   & NLO \\
162 & $ t(\to \nu(p_{3})+e^+(p_{4})+b(p_{5}))+q(p_{6}) [\mbox{decay}]$   & NLO \\
163 & $ t(\to \nu(p_{3})+e^+(p_{4})+b(p_{5}))+q(p_{6}) [\mbox{t-channel}] mb>0$   & NLO \\
166 & $ \bar{t}(\to e^-(p_{3})+\bar{\nu}(p_{4})+\bar{b}(p_{5}))+q(p_{6}) [\mbox{t-channel}]$   & NLO \\
167 & $ \bar{t}(\to e^-(p_{3})+\bar{\nu}(p_{4})+\bar{b}(p_{5}))+q(p_{6}) [\mbox{rad.in.dk}]$   & NLO \\
168 & $ \bar{t}(\to e^-(p_{3})+\bar{\nu}(p_{4})+\bar{b}(p_{5}))+q(p_{6}) [\mbox{t-channel}] mb>0$   & NLO \\
\hline 
171 & $ t(\to \nu(p_{3})+e^+(p_{4})+b(p_{5}))+\bar{b}(p_{6})) [\mbox{s-channel}]$   & NLO \\
172 & $ t(\to \nu(p_{3})+e^+(p_{4})+b(p_{5}))+\bar{b}(p_{6})) [\mbox{decay}]$   & NLO \\
176 & $ \bar{t}(\to e^-(p_{3})+\bar{\nu}(p_{4})+\bar{b}(p_{5}))+b(p_{6})) [\mbox{s-channel}]$   & NLO \\
177 & $ \bar{t}(\to e^-(p_{3})+\bar{\nu}(p_{4})+\bar{b}(p_{5}))+b(p_{6})) [\mbox{rad.in.dk}]$   & NLO \\
\hline 
180 & $ W^-(\to e^-(p_{3})+\bar{\nu}(p_{4}))+t(p_{5})$   & NLO \\
181 & $ W^-(\to e^-(p_{3})+\bar{\nu}(p_{4}))+t(\nu(p_{5})+e^+(p_{6})+b(p_{7}))$   & NLO \\
182 & $ W^-(\to e^-(p_{3})+\bar{\nu}(p_{4}))+t(\nu(p_{5})+e^+(p_{6})+b(p_{7})) [\mbox{rad.in.dk}]$   & NLO \\
183 & $ W^-(\to e^-(p_{3})+\bar{\nu}(p_{4}))+t(\nu(p_{5})+e^+(p_{6})+b(p_{7}))+b(p_{8})$   & LO \\
184 & $ W^-(\to e^-(p_{3})+\bar{\nu}(p_{4}))+t(p_{5})+b(p_{6}) [\mbox{massive b}]$   & LO \\
185 & $ W^+(\to \nu(p_{3})+e^+(p_{4}))+\bar{t}(p_{5})$   & NLO \\
186 & $ W^+(\to \nu(p_{3})+e^+(p_{4}))+\bar{t}(e^-(p_{5})+\bar{\nu}(p_{6})+\bar{b}(p_{7})$   & NLO \\
187 & $ W^+(\to \nu(p_{3})+e^+(p_{4}))+\bar{t}(e^-(p_{5})+\bar{\nu}(p_{6})+\bar{b}(p_{7}) [\mbox{rad.in.dk}]$   & NLO \\
\hline 
\end{tabular}
\end{center}
\end{table}
\newpage
\begin{table}
\begin{center}
\begin{tabular}{|l|l|l|}
\hline
201 & $ H(\to b(p_{3})+\bar{b}(p_{4})) + f(p_{5}) [\mbox{full mt dep.}]$   & LO \\
202 & $ H(\to \tau^-(p_{3})+\tau^+(p_{4})) + f(p_{5}) [\mbox{full mt dep.}]$   & LO \\
203 & $ H(\to b(p_{3})+\bar{b}(p_{4})) + f(p_{5})$   & NNLO \\
204 & $ H(\to \tau^-(p_{3})+\tau^+(p_{4})) + f(p_{5})$   & NNLO \\
206 & $ A(\to b(p_{3})+\bar{b}(p_{4})) + f(p_{5}) [\mbox{full mt dep.}]$   & LO \\
207 & $ A(\to \tau^-(p_{3})+\tau^+(p_{4})) + f(p_{5}) [\mbox{full mt dep.}]$   & LO \\
208 & $ H(\to W^+(\nu(p_{3}),e^+(p_{4}))W^-(e^-(p_{5}),\bar{\nu}(p_{6})))+f(p_{7})$   & NLO \\
209 & $ H(\to Z(\to e^-(p_{3}),e^+(p_{4}))Z(\to \mu^-(p_{5}),\mu^+(p_{6})))+f(p_{7})$   & NLO \\
210 & $ H(\to \gamma(p_{3})+\gamma(p_{4})) + f(p_{5})$   & NNLO \\
\hline 
211 & $ H(\to b(p_{3})+\bar{b}(p_{4}))+f(p_{5})+f(p_{6}) [\mbox{WBF}]$   & NLO \\
212 & $ H(\to \tau^-(p_{3})+\tau^+(p_{4}))+f(p_{5})+f(p_{6}) [\mbox{WBF}]$   & NLO \\
213 & $ H(\to W^+(\nu(p_{3}),e^+(p_{4}))W^-(e^-(p_{5}),\bar{\nu}(p_{6})))+f(p_{7})+f(p_{8}) [\mbox{WBF}]$   & NLO \\
214 & $ H(\to Z(\to e^-(p_3),e^+(p_{4}))+Z(\to \mu^-(p_{5}),\mu^+(p_{6})))+f(p_{7})+f(p_{8}) [\mbox{WBF}]$ & NLO \\
215 & $ H(\to \gamma(p_3)+\gamma(p_{4}))+f(p_{5})+f(p_{6}) [\mbox{WBF}]$ & NLO \\
216 & $ H(\to b(p_{3})+\bar{b}(p_{4}))+f(p_{5})+f(p_{6})+f(p_{7}) [\mbox{WBF+jet}]$   & LO \\
217 & $ H(\to \tau^-(p_{3})+\tau^+(p_{4}))+f(p_{5})+f(p_{6})+f(p_{7}) [\mbox{WBF+jet}]$   & LO \\
\hline 
221 & $ \tau^-(\to e^-(p_{3})+\bar{\nu}_e(p_{4})+\nu_\tau(p_{5}))+\tau^+(\to \bar{\nu}_\tau(p_{6})+\nu_e(p_{7})+e^+(p_{8}))$   & LO \\
\hline 
220 & $  Z(\to e^-(p_3),e^+(p_4))Z(\to \mu^-(p_5),\mu^+(p_6)))+f(p_7)+f(p_8) $ [weak]' & LO \\
2201 & $  Z(\to e^-(p_3),e^+(p_4))Z(\to \mu^-(p_5),\mu^+(p_6)))+f(p_7)+f(p_8) $ [strong]' & LO \\
222 & $  Z(\to e^-(p_3),e^+(p_4))Z(\to \nu_{\mu}(p_5),\bar{\nu}_{\mu}(p_6)))+f(p_7)+f(p_8) $ [weak]' & LO \\
2221 & $  Z(\to e^-(p_3),e^+(p_4))Z(\to \nu_{\mu}(p_5),\bar{\nu}_{\mu}(p_6)))+f(p_7)+f(p_8) $ [strong]' & LO \\
224 & $  W^-(e^-(p_3),\bar{\nu}_e(p_6)))W^+(\nu_{\mu}(p_5),mu^+(p_4))+f(p_7)+f(p_8) $ [weak]' & LO \\
2241 & $  W^-(e^-(p_3),\bar{\nu}_e(p_6)))W^+(\nu_{\mu}(p_5),mu^+(p_4))+f(p_7)+f(p_8) $ [strong]' & LO \\
226 & $  e^-(p_3)+e^+(p_4)+\nu_e(p_5)+\bar{\nu}_e(p_6)+f(p_7)+f(p_8) $ [weak]' & LO \\
228 & $  W^+(\nu_e(p_3),e^+(p_4)))W^+(\nu_{\mu}(p_5),mu^+(p_6))+f(p_7)+f(p_8) $ [weak]' & LO \\
2281 & $  W^+(\nu_e(p_3),e^+(p_4)))W^+(\nu_{\mu}(p_5),mu^+(p_6))+f(p_7)+f(p_8) $ [strong]' & LO \\
229 & $  W^-(e^-(p_3),\bar{\nu}_e(p_4)))W^-(mu^-(p_5),\bar{\nu}_{\mu}(p_6))+f(p_7)+f(p_8) $ [weak]' & LO \\
2291 & $  W^-(e^-(p_3),\bar{\nu}_e(p_4)))W^-(mu^-(p_5),\bar{\nu}_{\mu}(p_6))+f(p_7)+f(p_8) $ [strong]' & LO \\
223 & $  W^+(\nu_e(p_3),e^+(p_4))Z(\to \mu^-(p_5),\mu^+(p_6)))+f(p_7)+f(p_8) $ [weak]' & LO \\
2231 & $  W^+(\nu_e(p_3),e^+(p_4))Z(\to \mu^-(p_5),\mu^+(p_6)))+f(p_7)+f(p_8) $ [strong]' & LO \\
225 & $  W^-(e^-(p_3),\bar{\nu}_e(p_4))Z(\to \mu^-(p_5),\mu^+(p_6)))+f(p_7)+f(p_8) $ [weak]' & LO \\
2251 & $  W^-(e^-(p_3),\bar{\nu}_e(p_4))Z(\to \mu^-(p_5),\mu^+(p_6)))+f(p_7)+f(p_8) $ [strong]' & LO \\
\hline 
\end{tabular}
\end{center}
\end{table}
\newpage
\begin{table}
\begin{center}
\begin{tabular}{|l|l|l|}
\hline
231 & $t(p_3)+\bar{b}(p_4)+q(p_5)$ [\mbox{t-channel]} & NLO \\
232 & $t(p_3)+\bar{b}(p_4)+q(p_5)+q(p_6)$ [\mbox{t-channel]} & LO \\
233 & $t(\to \nu(p_3)+e^+(p_4)+b(p_5))+\bar{b}(p_6)+q(p_7)$ [\mbox{t-channel]} & NLO \\
234 & $t(\to \nu(p_3)+e^+(p_4)+b(p_5))+\bar{b}(p_6)+q(p_7)$ [\mbox{t-channel, rad.in.dk]} & NLO \\
235 & $t(\to \nu(p_3)+e^+(p_4)+b(p_5))+\bar{b}(p_6)+q(p_7)+f(p_8)$ [\mbox{t-channel]} & LO \\
236 & $\bar{t}(p_3)+b(p_4)+q(p_5)$ [\mbox{t-channel]} & NLO \\
237 & $\bar{t}(p_3)+b(p_4)+q(p_5)+q(p_6)$ [\mbox{t-channel]} & LO \\
238 & $\bar{t}(\to e^-(p_3)+\bar{\nu}(p_4)+\bar{b}(p_5))+b(p_6)+q(p_7)$ [\mbox{t-channel]} & NLO \\
239 & $\bar{t}(\to e^-(p_3)+\bar{\nu}(p_4)+\bar{b}(p_5))+b(p_6)+q(p_7)$ [\mbox{t-channel, rad.in.dk]} & NLO \\
240 & $\bar{t}(\to e^-(p_3)+\bar{\nu}(p_4)+\bar{b}(p_5))+b(p_6)+q(p_7)+f(p_8)$ [\mbox{t-channel]} & L0 \\
\hline 
251 & $ W^+(\to \nu(p_{3})+e^+(p_{4})) + W^+(\to \nu(p_{5})+e^+(p_{6}))+f(p_{7})+f(p_{8})$   & LO \\
252 & $ W^+(\to \nu(p_{3})+e^+(p_{4})) + W^+(\to \nu(p_{5})+e^+(p_{6}))+f(p_{7})+f(p_{8})+f(p_{9})$   & LO \\
253 & $W^+(\to\nu(p_3)+e^+(p_4)) + Z(\to e^-(p_5)+e^+(p_6))+f(p_7)+f(p_8)$ & LO  \\
254 & $W^-(\to e^-(p_3)+\bar{\nu}(p_4))+ Z(\to e^-(p_5)+e^+(p_6))+f(p_7)+f(p_8)$ & LO  \\
255 & $W^+(\to \nu(p_3)+e^+(p_4)) + Z(\to e^-(p_5)+e^+(p_6))+b(p_7)+f(p_8)$ & LO  \\
256 & $W^-(\to e^-(p_3)+\bar{\nu}(p_4))+ Z(\to e^-(p_5)+e^+(p_6))+b(p_7)+f(p_8)$ & LO  \\
259 & $W^+(\to \nu(p_3)+e^+(p_4)) + Z(\to e^-(p_5)+e^+(p_6))+b(p_7)+b~(p_8)$ & LO  \\
260 & $W^-(\to e^-(p_3)+\bar{\nu}(p_4))+ Z(\to e^-(p_5)+e^+(p_6))+b(p_7)+b~(p_8)$ & LO  \\
\hline 
261 & $ Z(\to e^-(p_{3})+e^+(p_{4}))+b(p_{5})$   & NLO \\
262 & $ Z(\to e^-(p_{3})+e^+(p_{4}))+c(p_{5})$   & NLO \\
263 & $ Z(\to e^-(p_{3})+e^+(p_{4}))+\bar{b}(p_{5})+b(p_{6}) [\mbox{1 b-tag}]$   & LO \\
264 & $ Z(\to e^-(p_{3})+e^+(p_{4}))+\bar{c}(p_{5})+c(p_{6}) [\mbox{1 c-tag}]$   & LO \\
266 & $ Z(\to e^-(p_{3})+e^+(p_{4}))+b(p_{5})(+\bar{b}(p_{6}))$   & NLO \\
267 & $ Z(\to e^-(p_{3})+e^+(p_{4}))+c(p_{5})(+\bar{c}(p_{6}))$   & NLO \\
\hline 
\end{tabular}
\end{center}
\end{table}
\newpage
\begin{table}
\begin{center}
\begin{tabular}{|l|l|l|}
\hline
270 & $ H(\gamma(p_{3})+\gamma(p_{4}))+f(p_{5})+f(p_{6}) [\mbox{in heavy top limit}]$   & NLO \\
271 & $ H(b(p_{3})+\bar{b}(p_{4}))+f(p_{5})+f(p_{6}) [\mbox{in heavy top limit}]$   & NLO \\
272 & $ H(\tau^-(p_{3})+\tau^+(p_{4}))+f(p_{5})+f(p_{6}) [\mbox{in heavy top limit}]$   & NLO \\
273 & $ H(\to W^+(\nu(p_{3}),e^+(p_{4}))W^-(e^-(p_{5}),\bar{\nu}(p_{6})))+f(p_{7})+f(p_{8})$   & NLO \\
274 & $ H(\to Z(e^-(p_{3}),e^+(p_{4}))Z(\mu^-(p_{5}),\mu^+(p_{6})))+f(p_{7})+f(p_{8})$   & NLO \\
275 & $ H(b(p_{3})+\bar{b}(p_{4}))+f(p_{5})+f(p_{6})+f(p_{7}) [\mbox{in heavy top limit}]$   & LO \\
276 & $ H(\tau^-(p_{3})+\tau^+(p_{4}))+f(p_{5})+f(p_{6})+f(p_{7}) [\mbox{in heavy top limit}]$   & LO \\
278 & $ H(\to W^+(\nu(p_3),e^+(p_4))W^-(e^-(p_5),\bar{\nu}(p_6)))+f(p_7)+f(p_8)+f(p_9)$ & LO \\
279 & $ H(\to Z(e^-(p_3),e^+(p_4))Z(\mu^-(p_5),\mu^+(p_6)))+f(p_7)+f(p_8)+f(p_9)$ & LO \\
\hline 
280 & $ \gamma(p_3)+f(p_{4})$ & NLO+F \\
282 & $ f(p_{1})+f(p_{2})\to  \gamma(p_{3})+f(p_{4})+f(p_{5})$   & LO \\
283 & $ f(p_{1})+f(p_{2})\to  \gamma(p_{3})+b(p_{4})$   & LO \\
284 & $ f(p_{1})+f(p_{2})\to  \gamma(p_{3})+c(p_{4})$   & LO \\
285 & $ f(p_{1})+f(p_{2})\to  \gamma(p_{3})+\gamma(p_{4})$   & NLO+F, NNLO \\
286 & $ f(p_{1})+f(p_{2})\to  \gamma(p_{3})+\gamma(p_{4})+f(p_{5})$   & LO \\
287 & $ f(p_{1})+f(p_{2})\to  \gamma(p_{3})+\gamma(p_{4})+\gamma(p_{5})$   & NLO+F \\
289 & $ f(p_{1})+f(p_{2})\to  \gamma(p_{3})+\gamma(p_{4})+\gamma(p_{5})+\gamma(p_{6})$   & NLO+F \\
\hline 
290 & $ W^+(\to \nu(p_{3})+e^+(p_{4}))+\gamma(p_{5})$   & NLO+F \\
292 & $ W^+(\to \nu(p_{3})+e^+(p_{4})) +\gamma(p_{5})+f(p_{6}) $   & LO \\
295 & $ W^-(\to e^-(p_{3})+\bar{\nu}(p_{4}))+\gamma(p_{5})$   & NLO+F \\
297 & $ W^-(\to e^-(p_{3})+\bar{\nu}(p_{4}))+\gamma(p_{5})+f(p_{6}) $   & LO \\
\hline 
300 & $  Z(\to e ^-(p_3)+e^+(p_4))+\gamma(p_5)$ & NLO+F \\
301 & $ Z(\to e ^-(p_3)+e^+(p_4))+\gamma(p_5)+\gamma(p_6) $& NLO +F \\
302 &$  Z(\to e ^-(p_3)+e^+(p_4))+\gamma(p_5)+f(p_6) $& NLO + F \\
303 &$   Z(\to e ^-(p_3)+e^+(p_4))+\gamma(p_5)+\gamma(p_6)+f(p_7) $& LO \\
304 & $  Z(\to e ^-(p_3)+e^+(p_4))+\gamma(p_5)+f(p_6)+f(p_7) $ & LO \\
305 & $  Z(\to 3(\nu(p_3)+\bar{\nu}(p_4)))+\gamma(p_5) $& NLO + F \\
306 & $  Z(\to 3(\nu(p_3)+\bar{\nu}(p_4)))+\gamma(p_5)+\gamma(p_6) $& NLO + F \\
307 &$  Z(\to 3(\nu(p_3)+\bar{\nu}(p_4)))+\gamma(p_5)+f(p_6) $ & NLO + F \\
308 &$   Z(\to 3(\nu(p_3)+\bar{\nu}(p_4)))+\gamma(p_5)+\gamma(p_6)+f(p_7) $  & LO \\
309 &$  Z(\to 3(\nu(p_3)+\bar{\nu}(p_4)))+\gamma(p_5)+f(p_6)+f(p_7) $ & LO \\
\hline 
311 & $ f(p_{1})+b(p_{2}) \to  W^+(\to \nu(p_{3})+e^+(p_{4}))+b(p_{5})+f(p_{6})$   & LO \\
316 & $ f(p_{1})+b(p_{2}) \to  W^-(\to e^-(p_{3})+\bar{\nu}(p_{4}))+b(p_{5})+f(p_{6})$   & LO \\
\hline 
321 & $ f(p_{1})+c(p_{2}) \to  W^+(\to \nu(p_{3})+e^+(p_{4}))+c(p_{5})+f(p_{6})$   & LO \\
326 & $ f(p_{1})+c(p_{2}) \to  W^-(\to e^-(p_{3})+\bar{\nu}(p_{4}))+c(p_{5})+f(p_{6})$   & LO \\
\hline 
331 & $ W^+(\to \nu(p_{3})+e^+(p_{4}))+c(p_{5})+f(p_{6}) [\mbox{c-s interaction}]$   & LO \\
336 & $ W^-(\to e^-(p_{3})+\bar{\nu}(p_{4}))+c(p_{5})+f(p_{6}) [\mbox{c-s interaction}]$   & LO \\
\hline 
341 & $ f(p_{1})+b(p_{2}) \to  Z(\to e^-(p_{3})+e^+(p_{4}))+b(p_{5})+f(p_{6}) [+f(p_{7})]$   & NLO \\
342 & $ f(p_{1})+b(p_{2}) \to  Z(\to e^-(p_{3})+e^+(p_{4}))+b(p_{5})+f(p_{6}) [+\bar{b}(p_{7})]$  & (REAL) \\
346 & $ f(p_{1})+b(p_{2}) \to  Z(\to e^-(p_{3})+e^+(p_{4}))+b(p_{5})+f(p_{6})+f(p_{7})$   & LO \\
347 & $ f(p_{1})+b(p_{2}) \to  Z(\to e^-(p_{3})+e^+(p_{4}))+b(p_{5})+f(p_{6})+\bar{b}(p_{7})$   & LO \\
\hline 
\end{tabular}
\end{center}
\end{table}
\newpage
\begin{table}
\begin{center}
\begin{tabular}{|l|l|l|}
\hline
351 & $ f(p_{1})+c(p_{2}) \to  Z(\to e^-(p_{3})+e^+(p_{4}))+c(p_{5})+f(p_{6}) [+f(p_{7})]$   & NLO \\
352 & $ f(p_{1})+c(p_{2}) \to  Z(\to e^-(p_{3})+e^+(p_{4}))+c(p_{5})+f(p_{6}) [+\bar{c}(p_{7})]$  & (REAL) \\
356 & $ f(p_{1})+c(p_{2}) \to  Z(\to e^-(p_{3})+e^+(p_{4}))+c(p_{5})+f(p_{6})+f(p_{7})$   & LO \\
357 & $ f(p_{1})+c(p_{2}) \to  Z(\to e^-(p_{3})+e^+(p_{4}))+c(p_{5})+f(p_{6})+\bar{c}(p_{7})$   & LO \\
\hline 
361 & $ c(p_{1})+\bar{s}(p_{2}) \to  W^+(\to \nu(p_{3})+e^+(p_{4})) [\mbox{mc=0 in NLO}]$   & NLO \\
362 & $ c(p_{1})+\bar{s}(p_{2}) \to  W^+(\to \nu(p_{3})+e^+(p_{4})) [\mbox{massless corrections only}]$   & NLO \\
363 & $ c(p_{1})+\bar{s}(p_{2}) \to  W^+(\to \nu(p_{3})+e^+(p_{4})) [\mbox{massive charm in real}]$   & NLO \\
\hline 
370 & $ W^+(\to \nu(p_{3})+e^+(p_{4}))+\gamma(p_{5})+\gamma(p_{6})$   & LO \\
371 & $ W^-(\to e^-(p_{3})+\bar{\nu}(p_{4}))+\gamma(p_{5})+\gamma(p_{6})$   & LO \\
\hline 
401 & $ W^+(\to \nu(p_{3})+e^+(p_{4}))+b(p_{5}) ~[\mbox{1,2 or 3 jets, 4FNS}]$   & NLO \\
402 & $ W^+(\to \nu(p_{3})+e^+(p_{4}))+(b+\bar{b})(p_{5}) ~[\mbox{1 or 2 jets, 4FNS}]$   & NLO \\
403 & $ W^+(\to \nu(p_{3})+e^+(p_{4}))+b(p_{5})+\bar b(p_{6}) ~[\mbox{2 or 3 jets, 4FNS}]$   & NLO \\
406 & $ W^-(\to e^-(p_{3})+\bar{\nu}(p_{4}))+b(p_{5}) ~[\mbox{1,2 or 3 jets, 4FNS}]$   & NLO \\
407 & $ W^-(\to e^-(p_{3})+\bar{\nu}(p_{4}))+(b+\bar{b})(p_{5}) ~[\mbox{1 or 2 jets, 4FNS}]$   & NLO \\
408 & $ W^-(\to e^-(p_{3})+\bar{\nu}(p_{4}))+b(p_{5})+\bar b(p_{6}) ~[\mbox{2 or 3 jets, 4FNS}]$   & NLO \\
\hline 
411 & $  f(p_1)+b(p_2) \to  W^+(\to \nu(p_3)+e^+(p_{4}))+b(p_{5})+f(p_{6})$ ~[\mbox{5FNS}] & NLO \\
416 & $  f(p_1)+b(p_2) \to  W^-(\to e^-(p_3)+\bar{\nu}(p_{4}))+b(p_{5})+f(p_{6})$ ~[\mbox{5FNS}] & NLO \\
\hline 
421 & $ W^+(\to \nu(p_{3})+e^+(p_{4}))+b(p_{5}) ~[\mbox{1,2 or 3 jets, 4FNS+5FNS}]$   & NLO \\
426 & $ W^-(\to e^-(p_{3})+\bar{\nu}(p_{4}))+b(p_{5}) ~[\mbox{1,2 or 3 jets, 4FNS+5FNS}]$   & NLO \\
\hline 
431 & $ W^+(\to \nu(p_3)+e^+(p_{4}))+b(p_{5})+\bar b(p_{6})+f(p_{7}) ~[\mbox{massive}]$ & LO \\
436 & $ W^-(\to e^-(p_3)+\bar{\nu}(p_{4}))+b(p_{5})+\bar b(p_{6})+f(p_{7}) ~[\mbox{massive}]$ & LO \\
\hline  
500 & $ W^+(\to \nu(p_3)+e^+(p_4)) +t(p_5)+\bar{t}(p_6) \mbox{[massive]}$ & NLO \\
501 & $ t(\to \nu(p_3)+e^+(p_4)+b(p_5))+\bar{t}(\to b~(p_6)+e^-(p_7)+\bar{\nu}(p_8))+W^+(\nu(p_9),\mu^+(p_{10}))$ & NLO \\
502 & $ \mbox{(same as process 501 but with radiation in decay)}$ & NLO \\
503 & $ t(\to \nu(p_3)+e^+(p_4)+b(p_5))+\bar{t}(\to b~(p_6)+q(p_7)+q~(p_8))+W^+(\nu(p_9),\mu^+(p_{10}))$ & NLO \\
506 & $ t(\to q(p_3)+q~(p_4)+b(p_5))+\bar{t}(\to b~(p_6)+e^-(p_7)+\bar{\nu}(p_8))+W^+(\nu(p_9),\mu^+(p_{10}))$ & NLO \\
\hline 
510 & $ W^-(\to e^-(p_3)+\bar{\nu}(p_4))+t(p_5)+\bar{t}(p_6) \mbox{[massive]} $ & NLO \\
511 & $ t(\to \nu(p_3)+e^+(p_4)+b(p_5))+\bar{t}(\to b~(p_6)+e^-(p_7)+\bar{\nu}(p_8))+W^-(\mu^-(p_9),\bar{\nu}(p_{10}))$ &  NLO \\
512 & $ \mbox{(same as process 511 but with radiation in decay)}$ & NLO \\
513 & $ t(\to \nu(p_3)+e^+(p_4)+b(p_5))+\bar{t}(\to b~(p_6)+q(p_7)+q~(p_8))+W^-(\mu^-(p_9),\bar{\nu}(p_{10}))$ & NLO \\
516 & $ t(\to q(p_3)+q~(p_4)+b(p_5))+\bar{t}(\to b~(p_6)+e^-(p_7)+\bar{\nu}(p_8))+W^-(\mu^-(p_9),\bar{\nu}(p_{10}))$ & NLO \\ 
529 & $ Z(\to e^-(p_3)+e^+(p_4))+t(p5)+\bar{t}(p_6)  $ & LO \\
530 & $ t(\to \nu(p_3)+e^+(p_4)+b(p_5))+\bar{t}(\to e^-(p_7)+\bar{\nu}(p_8)+b~(p_6))+Z(e^-(p_9),e^+(p_{10}))$ & LO \\
531 & $ t(\to \nu(p_3)+e^+(p_4)+b(p_5))+\bar{t}(\to e^-(p_7)+\bar{\nu}(p_8)+b~(p_6))+Z(b(p_9),b~(p_{10}))$ & LO \\
532 & $ t(\to \nu(p_3)+e^+(p_4)+b(p_5))+\bar{t}(\to q(p_7)+\bar{q}(p_8)+b~(p_6))+Z(e^-(p_9),e^+(p_{10}))$ & LO \\
533 & $ t(\to q(p_3)+\bar{q}(p_4)+b(p_5))+\bar{t}(\to e^-(p_7)+\bar{\nu}(p_8)+b~(p_6))+Z(e^-(p_9),e^+(p_{10}))$ & LO \\
\hline
\end{tabular}
\end{center}
\end{table}
\newpage
\begin{table}
\begin{center}
\begin{tabular}{|l|l|l|}
\hline
540 & $H(b(p_3)+\bar{b}(p_4))+t(p_5)+q(p_6)$      & NLO \\
541 & $H(b(p_3)+\bar{b}(p_4))+\bar{t}(p_5)+q(p_6)$      & NLO \\
544 & $H(b(p_3)+\bar{b}(p_4))+t(\nu(p_5)+e^+(p_6)+b(p_7))+q(p_9)$      & NLO \\
547 & $H(b(p_3)+\bar{b}(p_4))+\bar{t}(e^-(p_5)+\bar{\nu}(p_6)+b(p_7))+q(p_9)$      & NLO \\
\hline
550 & $H(\gamma(p_3)+\gamma(p_4))+t(p_5)+q(p_6)$      & NLO \\
551 & $H(\gamma(p_3)+\gamma(p_4))+\bar{t}(p_5)+q(p_6)$      & NLO \\
554 & $H(\gamma(p_3)+\gamma(p_4))+t(\nu(p_5)+e^+(p_6)+b(p_7))+q(p_9)$      & NLO \\
557 & $H(\gamma(p_3)+\gamma(p_4))+\bar{t}(e^-(p_5)+\bar{\nu}(p_6)+b(p_7))+q(p_9)$      & NLO \\
\hline
560 & $Z(e^-(p_3)+e^+(p_4))+t(p_5)+q(p_6)$      & NLO \\
561 & $Z(e^-(p_3)+e^+(p_4))+\bar{t}(p_5)+q(p_6)$      & NLO \\
562 & $Z(e^-(p_3)+e^+(p_4))+t(p_5)+q(p_6)+f(p_7)$       & LO \\
563 & $Z(e^-(p_3)+e^+(p_4))+\bar{t}(p_5)+q(p_6)+f(p_7)$      & LO \\
564 & $Z(e^-(p_3)+e^+(p_4))+t(\to\nu(p_5)+e^+(p_6)+b(p_7))+q(p_8)$      & NLO \\
566 & $Z(e^-(p_3)+e^+(p_4))+t(\to\nu(p_5)+e^+(p_6)+b(p_7))+q(p_8)+f(p_9)$ & LO \\
567 & $Z(e^-(p_3)+e^+(p_4))+\bar{t}(\to e^-(p_5)+\bar{\nu}(p_6)+\bar{b}(p_7))+q(p_8)$      & NLO \\
569 & $Z(e^-(p_3)+e^+(p_4))+\bar{t}(\to e^-(p_5)+\bar{\nu}(p_6)+\bar{b}(p_7))+q(p_8)+f(p_9)$  & LO \\
\hline
601 & $H(b(p_3)+\bar{b}(p_4))+H(\tau^-(p_5)+\tau^+(p_6)) $ &  LO \\
602 & $H(b(p_3)+\bar{b}(p_4))+H(\gamma(p_5)+\gamma(p_6)) $ &  LO \\
640 & $t(p_3)+\bar{t}(p_4)+H(p_5)$ & LO \\
641 & $t(\to\nu(p_3)+e^+(p_4)+b(p_5))+\bar{t}(\to\bar{\nu}(p_7)+e^-(p_8)+\bar{b}(p_6))+H(b(p_9)+\bar{b}(p_{10}))$ & LO \\
644 & $t(\to\nu(p_3)+e^+(p_4)+b(p_5))+\bar{t}(\to\bar{q}(p_7)+q(p_8)+\bar{b}(p_6))+H(b(p_9)+\bar{b}(p_{10}))$ & LO \\
647 & $t(\to q(p_3)+\bar{q}(p_4)+b(p_5))+\bar{t}(\to\bar{\nu}(p_7)+e^-(p_8)+\bar{b}(p_6))+H(b(p_9)+\bar{b}(p_{10}))$ & LO \\
651 & $t(\to\nu(p_3)+e^+(p_4)+b(p_5))+\bar{t}(\to\bar{\nu}(p_7)+e^-(p_8)+\bar{b}(p_6))+H(\gamma(p_9)+\gamma(p_{10}))$ & LO \\
654 & $t(\to\nu(p_3)+e^+(p_4)+b(p_5))+\bar{t}(\to\bar{q}(p_7)+q(p_8)+\bar{b}(p_6))+H(\gamma(p_9)+\gamma(p_{10}))$ & LO \\
657 & $t(\to q(p_3)+\bar{q}(p_4)+b(p_5))+\bar{t}(\to\bar{\nu}(p_7)+e^-(p_8)+\bar{b}(p_6))+H(\gamma(p_9)+\gamma(p_{10}))$ & LO \\
661 & $t(\to\nu(p_3) e^+(p_4) b(p_5)) +\bar{t}(\to\bar{\nu}(p_7)e^-(p_8)\bar{b}(p_6))+H(W^+(p_{9},p_{10})W^-(p_{11},p_{12}))$ & LO \\
664 & $t(\to\nu(p_3) e^+(p_4) b(p_5)) +\bar{t}(\to\bar{q}(p_7)q(p_8)\bar{b}(p_6))+H(W^+(p_{9},p_{10})W^-(p_{11},p_{12}))$ & LO \\
667 & $t(\to q(p_3) \bar{q}(p_4) b(p_5)) +\bar{t}\to(\bar{\nu}(p_7)e^-(p_8)\bar{b}(p_6))+H(W^+(p_{9},p_{10})W^-(p_{11},p_{12}))$ & LO \\
\hline 
\hline
\end{tabular}
\caption{Processes indicated by choice of the variable {\tt nproc}.}
\end{center}
\end{table}
\newpage
\begin{table}
\begin{center}
\begin{tabular}{|l|l|l|}
\hline
800 & $ V\to({\chi}(p_3)+\bar{\chi}(p_4)) +f(p_5) $ [Vector Mediator] & NLO \\
801 & $ A\to({\chi}(p_3)+\bar{\chi}(p_4)) +f(p_5)$ [Axial Vector Mediator] & NLO \\
802 & $ S\to({\chi}(p_3)+\bar{\chi}(p_4)) +f(p_5)$ [Scalar Mediator] & NLO \\
803 &  $ PS\to({\chi}(p_3)+\bar{\chi}(p_4)) +f(p_5)$ [Pseudo Scalar Mediator] & NLO \\
804 &  $ GG\to({\chi}(p_3)+\bar{\chi}(p_4)) +f(p_5)$ [Gluonic DM operator]  & NLO \\
805 & $ S--({\chi}(p_3)+\bar{\chi}(p_4)) +f(p_5)$ [Scalar Mediator, mt loops] & NLO \\
\hline
820 & $V\to({\chi}(p_3)+\bar{\chi}(p_4)) +\gamma(p_5)$ [Vector Mediator] & NLO + F \\
821 & $A\to({\chi}(p_3)+\bar{\chi}(p_4)) +\gamma(p_5) $[Axial Vector Mediator] & NLO + F \\
822 & $ S\to({\chi}(p_3)+\bar{\chi}(p_4)) +\gamma(p_5) $[Scalar Mediator] & NLO + F \\
823 &$ PS\to({\chi}(p_3)+\bar{\chi}(p_4)) +\gamma(p_5) $[Pseudo Scalar Mediator] & NLO + F \\
\hline
840 &$ V\to({\chi}(p_3)+\bar{\chi}(p_4)) +f(p_5)+f(p_6)$ [Vector Mediator] & LO\\
841 &$A\to({\chi}(p_3)+\bar{\chi}(p_4)) +f(p_5)+f(p_6)$ [Axial Vector Mediator]  & LO\\
842 &$ S\to({\chi}(p_3)+\bar{\chi}(p_4)) +f(p_5)+f(p_6)$ [Scalar Mediator]  & LO\\
843 &$ PS\to({\chi}(p_3)+\bar{\chi}(p_4)) +f(p_5)+f(p_6)$ [Pseudo Scalar Mediator]  & LO\\
844 &$ GG\to({\chi}(p_3)+\bar{\chi}(p_4)) +f(p_5)+f(p_6)$ [Gluonic DM operator]  & LO\\
\hline
845 & $V\to({\chi}(p_3)+\bar{\chi}(p_4)) +\gamma(p_5)+f(p_6)$ [Vector Mediator]  & LO\\
846 & $A\to({\chi}(p_3)+\bar{\chi}(p_4)) +\gamma(p_5)+f(p_6)$ [Axial Vector Mediator]  & LO\\
847 & $S\to({\chi}(p_3)+\bar{\chi}(p_4)) +\gamma(p_5)+f(p_6)$ [Scalar Mediator]  & LO\\
848 & $PS\to({\chi}(p_3)+\bar{\chi}(p_4)) +\gamma(p_5)+f(p_6)$ [Pseudo Scalar Mediator]  & LO\\
\hline
902 & Check of Volume of 2 particle phase space & \\  
903 & Check of Volume of 3 particle phase space & \\  
904 & Check of Volume of 4 particle phase space & \\  
905 & Check of Volume of 5 particle phase space & \\  
906 & Check of Volume of 6 particle phase space & \\  
908 & Check of Volume of 8 particle phase space & \\  
909 & Check of Volume of 4 particle massive phase space & \\  
910 & Check of Volume of 3 particle (2 massive) phase space & \\  
911 & Check of Volume of 5 particle W+t (with decay) massive phase space & \\  
912 & Check of Volume of 5 particle W+t (no decay) massive phase space & \\  
913 & Check of Volume of 5 particle W+t+g (in decay) massive phase space & \\  
914 & Check of Volume of 5 particle W+t+g (in production) massive phase space & \\  
\hline 
\hline
\end{tabular}
\caption{Processes indicated by choice of the variable {\tt nproc}.\label{nproctable}}
\end{center}
\end{table}
\clearpage


\section{Historical PDF sets}
\label{olderPDFs}
The availability of a number of historical PDF sets is retained in the code.  These should
typically not be used in modern analyses, but they may be helpful for comparison with
older codes or in specialized cases.

These distributions, together with their associated $\alpha_S(M_Z)$
values, are given in Tables~\ref{pdlabelmrs} and~\ref{pdlabelcteq}. 
For the older distributions, where the
coupling was specified by $\Lambda$ this requires 
some calculation and/or guesswork.

\begin{table}[h]
\begin{center}
\begin{tabular}{|c|c|c||c|c|c|}
\hline
  &   &  &
{\tt mrs02nl}  & 0.1197       & \mrstohtwo \\
{\tt mrs02nn}  & 0.1154       & \mrstohtwo &
{\tt mrs4nf3}  & 0.1083       & \mrstff \\
{\tt mrs4lf3}  & 0.1186       & \mrstff &
{\tt mrs4nf4}  & 0.1153       & \mrstff \\
{\tt mrs4lf4}  & 0.1251       & \mrstff &
{\tt mrs0119}  & 0.119        & \mrstohone \\
{\tt mrs0117}  & 0.117        & \mrstohone &
{\tt mrs0121}  & 0.121        & \mrstohone \\
{\tt mrs01\_j} & 0.121        & \mrstohone &
{\tt mrs01lo}  & 0.130        & \mrstohtwofirst \\ 
{\tt mrs99\_1} & 0.1175       & \mrsninenine &
{\tt mrs99\_2} & 0.1175       & \mrsninenine \\
{\tt mrs99\_3} & 0.1175       & \mrsninenine &
{\tt mrs99\_4} & 0.1125       & \mrsninenine \\    
{\tt mrs99\_5} & 0.1225       & \mrsninenine &
{\tt mrs99\_6} & 0.1178       & \mrsninenine \\    
{\tt mrs99\_7} & 0.1171       & \mrsninenine &
{\tt mrs99\_8} & 0.1175       & \mrsninenine \\    
{\tt mrs99\_9} & 0.1175       & \mrsninenine &
{\tt mrs9910}  & 0.1175       & \mrsninenine \\    
{\tt mrs9911}  & 0.1175       & \mrsninenine &
{\tt mrs9912}  & 0.1175       & \mrsninenine \\    
{\tt mrs98z1}  &  0.1175      & \mrsnineeight &  
{\tt mrs98z2}  &  0.1175      & \mrsnineeight \\ 
{\tt mrs98z3}  &  0.1175      & \mrsnineeight &  
{\tt mrs98z4}  &  0.1125      & \mrsnineeight \\  
{\tt mtungb1}  &  0.109       & \mrsnineeight &
{\tt mrs98z5}  &  0.1225      & \mrsnineeight \\   
{\tt mrs96r1}  &  0.113       & \mrsninesix &    
{\tt mrs96r2}  &  0.120       & \mrsninesix \\  
{\tt mrs96r3}  &  0.113       & \mrsninesix &   
{\tt mrs96r4}  &  0.120       & \mrsninesix \\   
{\tt mrs95ap}  &  0.1127      & \mrsninefive &
{\tt mrs95\_g} &  0.1148      & \mrsninefive \\
{\tt hmrs90e}  &  0.09838     & \hmrs & 
{\tt hmrs90b}  &  0.10796     & \hmrs \\
\hline
\end{tabular}
\end{center}
\caption{Historical MRS-type pdf sets, their corresponding values of
$\alpha_S(M_Z)$ and a reference to the paper or preprint that
describes their origin.
\label{pdlabelmrs}}
\end{table}
\begin{table}[h]
\begin{center}
\begin{tabular}{|c|c|c||c|c|c|}
\hline
               &              & &
{\tt cteq66m}  &  0.118       & \cteqsixsixm \\
{\tt cteq61m}  &  0.118       & \cteqsixonem &
{\tt cteq6\_m} &  0.118       & \cteqsix \\
{\tt cteq6\_d} &  0.118       & \cteqsix &
{\tt cteq6\_l} &  0.118       & \cteqsix \\
{\tt cteq6l1}  &  0.130       & \cteqsix &
{\tt cteq5hq}  &  0.118       & \cteqfive \\
{\tt cteq5f3}  &  0.106       & \cteqfive &
{\tt cteq5f4}  &  0.112       & \cteqfive \\
{\tt cteq5\_m} &  0.118       & \cteqfive &
{\tt cteq5\_d} &  0.118       & \cteqfive \\
{\tt cteq5\_l} &  0.127       & \cteqfive & 
{\tt cteq5l1}  &  0.127       & \cteqfive \\
{\tt cteq5hj}  &  0.118       & \cteqfive &
{\tt cteq5m1}  &  0.118       & \cteqfive \\
{\tt ctq5hq1}  &  0.118       & \cteqfive &
{\tt cteq4a5}  &  0.122       & \cteqfour \\
{\tt cteq4hj}  &  0.116       & \cteqfour &
{\tt cteq4lq}  &  0.114       & \cteqfour \\
{\tt cteq4\_m} &  0.116       & \cteqfour &
{\tt cteq4\_d} &  0.116       & \cteqfour \\
{\tt cteq4\_l} &  0.132       & \cteqfour &
{\tt cteq4a1}  &  0.110       & \cteqfour \\
{\tt cteq4a2}  &  0.113       & \cteqfour &
{\tt cteq4a3}  &  0.116       & \cteqfour \\
{\tt cteq4a4}  &  0.119       & \cteqfour &
{\tt cteq3\_m} &  0.112       & \cteqthree \\
{\tt cteq3\_l} &  0.112       & \cteqthree &
{\tt cteq3\_d} &  0.112       & \cteqthree \\
\hline
\end{tabular}
\end{center}
\caption{Historical CTEQ-type pdf sets, their corresponding values of
$\alpha_S(M_Z)$ and a reference to the paper or preprint that
describes their origin.
\label{pdlabelcteq}}
\end{table}


\clearpage

\section{Version 8.0 changelog}
\label{changelog8.0}
\begin{itemize}
\item Introduced NNLO capability for color-singlet processes.
\item Overall improvement in speed.
\item Added native support for additional pdf sets, NNPDF, CT14, MMHT.
\item Fixed small asymmetry in calculation of $W+1$~jet process at NLO.
\item Fixed per-mille level bug in NLO calculations of $W,Z,H+2$~jet processes.
\end{itemize}

\section{Version 7.0 changelog}
\label{changelog7.0}
\begin{itemize}
\item Implementation of OpenMP (Open Multi-processing)
\item Inclusion of four-photon process at NLO, {\tt nproc=289}.
\item Inclusion of vector boson fusion/vector boson scattering processes at LO,
 {\tt nproc=220,2201,222,2221,224,2241,226,228,2281,229,2291,223,2231,225,2251}.
\item After Higgs discovery, added s-channel Higgs diagrams to the $gg \to VV$ process,
in $VV$ production processes, {\tt nproc=61,62,64,81-84,86-90}
\end{itemize}

\section{Version 6.7 changelog}
\label{changelog6.7}
\begin{itemize}
\item Fixed errors reported in histograms
\item Changed implementation of PDF uncertainty output for cross sections and histograms; code now differentiates
between appropriate calculations for CTEQ and MSTW, NNPDF, Alekhin et al
\item Added possibility of (unweighted) LHE output for select LO processes
\item Added a check to VEGAS to enable graceful exit if integral is zero
\item Included changes to allow production of grids for use with APPLGRID
\item Moved $H+b$ processes from {\tt nproc=131,132,133} to {\tt nproc=136,137,138}
\item Added effect of massive loops in $gg\to ZZ$ box process and interference with Higgs diagrams,
{\tt nproc=128,129,130,131,132,133}
\item Changed implementation of $gg \to WW$ interference-related processeses {\tt nproc=126,127} to
the same style as $gg\to ZZ$, {\tt nproc=123,124,125,126}
\item Added 1-loop $HH$ processes,  {\tt nproc=601,602}
\item Added triphoton process at NLO, {\tt nproc=287}
\item Added $W\gamma\gamma$ process at LO,  {\tt nproc=370,371}
\item New, faster implementation of H+5 parton amplitudes 
\item Added two new cuts to the input file, {\tt ptmin\_photon(3rd)} and {\tt R(photon,jet)\_min}
\item Added a flag to the input file to control creation of APPLGRID output, {\tt creategrid}
\item Changed previous {\tt evtgen} logical flag in input file to an integer, {\tt nevtrequested}
\item Added a line in the input file to allow for an anomalous Higgs width for {\tt nproc=128,129,130,131,133}
\end{itemize}

\section{Versions 6.6 and 6.5 changelog}
\label{changelog6.5}
\begin{itemize}
\item Moved the ttH processes to the 640's
\item Added $tH$ and $\bar{t}H$  processes.
\item Added $tZ$ and $\bar{t}Z$  processes.
\item Added $\gamma+b$ and $\gamma+c$ process at leading order.
\item Added $W Z $ + 2 parton processes.
\item Shuffled 144 and 145.
\item Added new Dark Matter mono-jet and mono-photon processes (with process numbers in the 800's) 
\end{itemize}

\section{Version 6.4 changelog}
\label{changelog6.4}

\subsection{Code changes}
\begin{itemize}
\item Corrected bug in implementation of $H\bar q q gg$ virtual amplitudes.
\item Enabled effect of {\tt removebr} for process 307.
\item Fixed the implementation of a dynamic scale for single top + b processes.
\item Added a new scale choice for top production ($m(345)^2+p_T(345)^2$).
\item Improved numerical stability in calculation of virtual contribution to process
201 and in calculation of real corrections to processes 180--187.  
\end{itemize}

\section{Version 6.3 changelog}
\label{changelog6.3}

\begin{itemize}
\item Implementation of new processes:
\begin{itemize}
\item
Added processes {\tt 63,65} giving the radiation in the hadronic decay of the $W^+W^-$ process. 
\item 
Added processes {\tt 114} and {\tt 115} giving the $gg\to H\to WW$ process with radiation in the hadronic decay of the $W$.
\item 
Added processes {\tt 501-516} giving the production and decay of 
$t \bar{t} W^\pm$ including radiation in the semi-leptonic decay of the top and the anti-top.
\item 
Added processes {\tt 120} and {\tt 121} giving the $H\to Z/\gamma^*(\to l \bar{l})+\gamma$ 
in the gluon fusion production process
\item 
Added processes in the range {\tt 300-309} describing production of $Z+\gamma+\gamma$ 
and $Z+\gamma$+jet at NLO.
\item
Process numbers from version 6.2 were moved to make space 
(e.g. old $64 \to 69$ and old $63\to 64$, old $192-198 \to 500-531$.). 
\end{itemize}
\subsection{Code changes}
\item Lengthened the number of characters allowed for the input parameter LHAPDF {\tt group}, 
that is read from the input file.
\item Removed the tacit assumption of unitarity in processes, {\tt 13} and {\tt 18}, 
subprocess $gg \to \bar{q}$, 
allowing arbitrary values of $V_{cs}$ and $V_{cd}$.  
\item 
Corrected closed fermion loop virtual corrections to the $gg \to Z+2$-jet subprocess which were assigned to the wrong 
helicities in error. This bug was of minor numerical importance. 
\item
Fixed number of jets expected in nplotter\_generic.f for ZH and WH processes with removebr=T.
\item
Added additional parameters to the input files, 
especially for powheg style output and photon cuts, as specified in section \ref{Input_parameters}.
\item 
changed mbook so that zero bins still appear in the Topdrawer output,
(for consistency, especially for combining runs).
\item
Removed a correction used to debug hadronic $W$ decays, that was erroneously left in the code.
The hadronic branching ratio of the $W$ now changes at NLO as it should.

\item Corrected bug that led to the $W^-t$ process (187) being calculated incorrectly.

\item Removed a bug that caused the real part Higgs+1 jet process (203) 
to crash (in version 6.2 only).

\item
Coupling of Higgs boson to top and bottom quarks expressed in terms of the running mass.

\item
Corrected normalization of pseudoscalar Higgs cross section (processes 206 and 207)
and set active quark in the loop to be the top rather than the bottom.

\end{itemize}

\section{Version 6.2 changelog}
\label{changelog6.2}

A number of changes to the code and input files have been implemented
between v6.1 and v6.2. These are listed below.

\subsection{Input file changes}
\begin{itemize}
\item Implementation of new processes:
\begin{itemize}
\item 
Added process {\tt 233-235} and {\tt 238-240} that give results in the four-flavour scheme
      with top decay.
\item 
Added process {\tt 270} (Higgs + 2 jets, with Higgs decaying to $\gamma \gamma$).
\item 
Added LO processes {\tt 278} and {\tt 279} (Higgs + 3 jets with Higgs decaying to $WW$ or $ZZ$).
\item 
Adding process {\tt 90} for identical fermions in $ZZ$ decay.
\item
Added processes {\tt 12} and {\tt 17}, $Wb$ production from charm quarks
\end{itemize}
\end{itemize}


\subsection{Code changes}
\begin{itemize}
\item
Included new plotting routines for s-channel single top, 4F t-channel single top and
top pair production,
\begin{verbatim} nplotter_tbbar.f, nplotter_4ftwdk.f,  nplotter_ttbar.f
\end{verbatim}
\item
Corrected all nplotter files to ensure ntuple output.
\item
Added CT10 parton distributions.
\item
Upgraded treatment of $t \bar{t}$ one-loop matrix elements for processes {\tt 141-150} to use results of 
ref.~\cite{Badger:2011yu} rather than ref.~\cite{Korner:2002hy} with consequent improvement in speed.
\item
Updated check on whether non-zero $m_{56}$ cut is applied to exclude $Z \to$~neutrino processes (for which is is not needed). 
\item
Removed overcounting of neutrinos for non-resonant diagrams in the $ZZ$ process.
\item
Removed effect of $m_{34}$ for dirgam process {\tt 280}. 
Therefore for this process $m_{34}$ cut is not operative.
\item
Corrected $ZZ$ processes {\tt 80-90} where the Z's decayed differently
to different final states. 
There was a bug in the gluon-gluon contribution since \begin{verbatim} gg_ZZ.f \end{verbatim} did not handle the 
different couplings for Z(3+4) and Z(5+6) correctly.
\item
Removed dependence on path in \begin{verbatim} ffinit_mine.f \end{verbatim} and modified Install script 
to create symbolic links.
\item
Updated handling of anomalous couplings to allow for no form-factors if tevscale $< 0$, see section \ref{sec:anomalous}.
\item
Moved processes ({\tt 151-156} to {\tt 141-151}), ({\tt 131,132} to {\tt 251,252}) and ({\tt 141-143} to {\tt 131-133}).
\item
Improved implementation of $p_T$ cut in 
\begin{verbatim} gg_WW.f, gg_ZZ.f \end{verbatim}
\item
Added improvements to mbook.f for Root histogram errors
\item
Corrected masscuts.f to avoid problems with $m_{56}$ cut.
\end{itemize}


\section{Version 6.1 changelog}
\label{changelog}

A number of changes to the code and input files have been implemented
between v6.0 and v6.1. These are listed below.

\subsection{Input file changes}
\begin{itemize}
\item Implementation of new processes:
\begin{itemize}
\item Higgs decay modes for various
production channels ({\tt 93, 94, 98, 99, 104, 105, 109, 210, 214, 215}).
\item Calculation of $gg \to H \to WW$ including exact top and bottom loops ({\tt 121})
and interference with S.M. box diagrams ({\tt 122}).
\item Production of top pairs in association with a $W$ ({\tt 198, 199}).
\item Direct photon production including fragmentation ({\tt 280}).
\item Reorganization of $Wb$ calculations in
4FNS ({\tt 401}-{\tt 408}), 5FNS ({\tt 411, 416}) and when combined
({\tt 421, 426}).
\end{itemize}
\item Addition of flags {\tt writetop}, {\tt writedat}, {\tt writegnu}
and {\tt writeroot} to input file, to control writing of output into Topdrawer,
plain ascii, gnuplot and root script files.
\item New flag to trigger exclusion of gluon-gluon initiated  sub-processes
({\tt omitgg}).
\item New lines in input file to specify anomalous $ZZ\gamma$
and $Z\gamma\gamma$ couplings.
\item Changed implementation of transverse mass cut ({\tt m34transcut})
to be process-specific (see manual).
\item Added fragmentation functions of Gehrmann-de Ridder and Glover
(see description of {\tt fragset} in manual).
\item Added ability to over-ride switching from a scaling to a fixed isolation
cut according to value of {\tt epsilon\_h} (see manual).
\item Changed the role of {\tt dynamicscale} from a boolean flag to a string
variable specifying the type of dynamic scale to be applied. See manual
for description of new scale choices available.
\end{itemize}

\subsection{Code changes}
\begin{itemize}
\item Re-enabled n-tuple output for processes with specific {\tt nplotter} routines.
\item Corrected phase-space generation for diphoton and direct photon production
when cut on $p_T({\rm photon})$ is less than cut on $p_T({\rm jet})$.
\item Corrected implementation of errors in histogram output.
\item Added code to allow histogram output in {\tt gnuplot} and {\tt root} format.
\item Corrected virtual amplitudes for $WZ$ production with $Z \to $~neutrinos
({\tt 72, 77}).
\item Corrected implementation of processes {\tt 52, 53, 54}.
\item Implemented anomalous couplings for $W\gamma$ and $Z\gamma$ processes.
\item Corrected calculation of $W+2$~jet and $Z+2$~jet processes with
{\tt Gflag = Qflag = .true.} and a dynamic scale.
\item Added non-perturbative contributions to fragmentation functions.
\item Fixed implementation of finite $m_t$ correction factor that is applied to
$gg \to H$ cross-sections. Differences with v6.0 for $m_H > 2 m_t$ only.
\item Added ability to calculate direct photon production including fragmentation
contributions.
\item Corrected definition of array containing particle momenta in $H+1$~jet virtual
routines (v6.0 results for these processes may be compiler-dependent).
\item Corrected implementation of invariant mass cuts for cases when invariant
masses do not correspond to electroweak bosons.
\item Allowed off-shell $W$ bosons in top decays for processes {\tt 36},
 {\tt 151}, {\tt 152} and {\tt 153}.
\end{itemize}


\begin{thebibliography}{99}
%\cite{Ellis:2007qk}
\bibitem{Ellis:2007qk}
  R.~K.~Ellis and G.~Zanderighi,
  %``Scalar one-loop integrals for QCD''
  JHEP {\bf 0802}, 002 (2008)
  [arXiv:0712.1851 [hep-ph]].
  %%CITATION = JHEPA,0802,002;%%

%\cite{Georgi:1991ci}
\bibitem{Georgi:1991ci}
H.~Georgi,
%``Effective field theory and electroweak radiative corrections,''
Nucl.\ Phys.\ B {\bf 363}, 301 (1991).
%%CITATION = NUPHA,B363,301;%%

%\cite{Maltoni:2002qb}
\bibitem{Maltoni:2002qb}
F.~Maltoni and T.~Stelzer,
%``MadEvent: Automatic event generation with MadGraph,''
JHEP {\bf 0302}, 027 (2003)
[arXiv:hep-ph/0208156].
%%CITATION = HEP-PH 0208156;%%

\bibitem{Alpgen}
Michelangelo L. Mangano, Mauro Moretti, Fulvio Piccinini, Roberto Pittau and Antonello Polosa, {\it http://mlm.home.cern.ch/mlm/alpgen/}

\bibitem{Lusifer}
Stefan Dittmaier and Markus Roth,\\
{\it http://wwwth.mppmu.mpg.de/members/roth/Lusifer/lusifer.html}

\bibitem{Cacciari:2008}
   M.~Cacciari, G.~P.~Salam and G.~Soyez,
  JHEP {\bf 0804}, 063 (2008)
  [arXiv:0802.1189 [hep-ph]].

\bibitem{Bourhis:1997yu}
  L.~Bourhis, M.~Fontannaz and J.~P.~Guillet,
  %``Quarks and gluon fragmentation functions into photons,''
  Eur.\ Phys.\ J.\  C {\bf 2}, 529 (1998)
  [arXiv:hep-ph/9704447].
  %%CITATION = EPHJA,C2,529;%%

%\cite{GehrmannDeRidder:1998ba}
\bibitem{GehrmannDeRidder:1998ba}
  A.~Gehrmann-De Ridder, E.~W.~N.~Glover,
  %``Final state photon production at LEP,''
  Eur.\ Phys.\ J.\  {\bf C7}, 29-48 (1999).
  [hep-ph/9806316].

%\cite{Nagy:2003tz}
\bibitem{Nagy:2003tz}
  Z.~Nagy,
  %``Next-to-leading order calculation of three jet observables in hadron hadron
  %collision,''
  Phys.\ Rev.\  D {\bf 68}, 094002 (2003)
  [arXiv:hep-ph/0307268].
  %%CITATION = PHRVA,D68,094002;%%

%\cite{Eskola:1998df}
\bibitem{Eskola:1998df}
  K.~J.~Eskola, V.~J.~Kolhinen and C.~A.~Salgado,
  %``The scale dependent nuclear effects in parton distributions for  practical
  %applications,''
  Eur.\ Phys.\ J.\ C {\bf 9}, 61 (1999)
  [arXiv:hep-ph/9807297].
  %%CITATION = HEP-PH 9807297;%%
  
%\cite{Campbell:2006wx}
\bibitem{Campbell:2006wx}
  J.~M.~Campbell, J.~W.~Huston and W.~J.~Stirling,
  %``Hard interactions of quarks and gluons: A primer for LHC physics,''
  Rept.\ Prog.\ Phys.\  {\bf 70}, 89 (2007)
  [arXiv:hep-ph/0611148].
  %%CITATION = RPPHA,70,89;%%

%\cite{Ball:2008by}
\bibitem{Ball:2008by}
  R.~D.~Ball {\it et al.}  [NNPDF Collaboration],
  %``A determination of parton distributions with faithful uncertainty
  %estimation,''
  Nucl.\ Phys.\  B {\bf 809}, 1 (2009)
  [arXiv:0808.1231 [hep-ph]].
  %%CITATION = NUPHA,B809,1;%%

%\cite{Badger:2010mg}
\bibitem{Badger:2010mg}
  S.~Badger, J.~M.~Campbell and R.~K.~Ellis,
  %``QCD corrections to the hadronic production of a heavy quark pair and a
  %W-boson including decay correlations,''
  arXiv:1011.6647 [hep-ph].
  %%CITATION = ARXIV:1011.6647;%%

%\cite{Bern:1997sc}
\bibitem{Bern:1997sc}
  Z.~Bern, L.~J.~Dixon and D.~A.~Kosower,
  %``One-loop amplitudes for $e^+ e^-$ to four partons,''
  Nucl.\ Phys.\  B {\bf 513}, 3 (1998)
  [arXiv:hep-ph/9708239].
  %%CITATION = NUPHA,B513,3;%%

%\cite{Dixon:1998py}
\bibitem{Dixon:1998py}
  L.~J.~Dixon, Z.~Kunszt and A.~Signer,
  %``Helicity amplitudes for O(alpha-s) production of $W^{+} W^{-}$, $W^\pm Z$,
  %$Z Z$, $W^\pm \gamma$, or $Z \gamma$ pairs at hadron colliders,''
  Nucl.\ Phys.\  B {\bf 531}, 3 (1998)
  [arXiv:hep-ph/9803250].
  %%CITATION = NUPHA,B531,3;%%

%\cite{Dixon:1999di}
\bibitem{Dixon:1999di}
  L.~J.~Dixon, Z.~Kunszt, A.~Signer,
  %``Vector boson pair production in hadronic collisions at order $\alpha_s$ : Lepton correlations and anomalous couplings,''
  Phys.\ Rev.\  {\bf D60}, 114037 (1999).
  [hep-ph/9907305].

%\cite{Ellis:1991qj}
\bibitem{Ellis:1991qj}
  R.~K.~Ellis, W.~J.~Stirling and B.~R.~Webber,
  %``QCD and collider physics,''
  Camb.\ Monogr.\ Part.\ Phys.\ Nucl.\ Phys.\ Cosmol.\  {\bf 8}, 1 (1996).
  %%CITATION = CMPCE,8,1;%%

%\cite{Campbell:2011cu}
\bibitem{Campbell:2011cu}
  J.~M.~Campbell, R.~K.~Ellis and C.~Williams,
  %``Gluon-gluon contributions to W+ W- production and Higgs interference
  %effects,''
  arXiv:1107.5569 [hep-ph].
  %%CITATION = ARXIV:1107.5569;%%

%\cite{Melia:2010bm}
\bibitem{Melia:2010bm}
  T.~Melia, K.~Melnikov, R.~Rontsch and G.~Zanderighi,
  %``Next-to-leading order QCD predictions for W+W+jj production at the LHC,''
  JHEP {\bf 1012}, 053 (2010)
  [arXiv:1007.5313 [hep-ph]].
  %%CITATION = JHEPA,1012,053;%%

%\cite{Amsler:2008zzb}
\bibitem{Amsler:2008zzb}
  C.~Amsler {\it et al.}  [Particle Data Group],
  %``Review of particle physics,''
  Phys.\ Lett.\  B {\bf 667}, 1 (2008)
 and 2009 partial update for the 2010 edition.
  %%CITATION = PHLTA,B667,1;%%

%\cite{Korner:2002hy}
\bibitem{Korner:2002hy}
  J.~G.~Korner and Z.~Merebashvili,
  %``One-loop corrections to four-point functions with two external massive
  %fermions and two external massless partons,''
  Phys.\ Rev.\  D {\bf 66}, 054023 (2002)
  [arXiv:hep-ph/0207054].
  %%CITATION = PHRVA,D66,054023;%%

%\cite{Nason:1987xz}
\bibitem{Nason:1987xz}
P.~Nason, S.~Dawson and R.~K.~Ellis,
%``The Total Cross-Section For The Production Of Heavy Quarks In Hadronic
%Collisions,''
Nucl.\ Phys.\ B {\bf 303}, 607 (1988).
%%CITATION = NUPHA,B303,607;%%

\bibitem{Ravindran:2002dc}
V.~Ravindran, J.~Smith and W.~L.~Van Neerven,
%``Next-to-leading order QCD corrections to differential distributions of  Higgs boson production in hadron hadron collisions,''
Nucl.\ Phys.\ B {\bf 634}, 247 (2002)
[arXiv:hep-ph/0201114].
%%CITATION = HEP-PH 0201114;%%

%\cite{Campbell:2012uf}
\bibitem{Campbell:2012uf} 
  J.~M.~Campbell and R.~K.~Ellis,
  %``Top-quark processes at NLO in production and decay,''
  arXiv:1204.1513 [hep-ph].
  %%CITATION = ARXIV:1204.1513;%%
  %32 citations counted in INSPIRE as of 04 Jun 2013

%\cite{Schmidt:1997wr}
\bibitem{Schmidt:1997wr}
C.~R.~Schmidt,
%``H $\to$ g g g (g q anti-q) at two loops in the large-M(t) limit,''
Phys.\ Lett.\ B {\bf 413}, 391 (1997)
[arXiv:hep-ph/9707448].
%%CITATION = HEP-PH 9707448;%%

%\cite{deFlorian:1999zd}
\bibitem{deFlorian:1999zd}
D.~de Florian, M.~Grazzini and Z.~Kunszt,
%``Higgs production with large transverse momentum in hadronic collisions  at next-to-leading order,''
Phys.\ Rev.\ Lett.\  {\bf 82}, 5209 (1999)
[arXiv:hep-ph/9902483].
%%CITATION = HEP-PH 9902483;%%

%\cite{Glosser:2002gm}
\bibitem{Glosser:2002gm}
C.~J.~Glosser and C.~R.~Schmidt,
%``Next-to-leading corrections to the Higgs boson transverse momentum  spectrum in gluon fusion,''
JHEP {\bf 0212}, 016 (2002)
[arXiv:hep-ph/0209248].
%%CITATION = HEP-PH 0209248;%%

%\cite{Kleiss:1988xr}
\bibitem{Kleiss:1988xr}
R.~Kleiss and W.~J.~Stirling,
%``Top Quark Production At Hadron Colliders: Some Useful Formulae,''
Z.\ Phys.\ C {\bf 40}, 419 (1988).
%%CITATION = ZEPYA,C40,419;%%

%\cite{Frixione:1998jh}
\bibitem{Frixione:1998jh}
  S.~Frixione,
  %``Isolated photons in perturbative {QCD},''
  Phys.\ Lett.\  B {\bf 429}, 369 (1998)
  [arXiv:hep-ph/9801442].
  %%CITATION = PHLTA,B429,369;%%

%\cite{Bern:2002jx}
\bibitem{Bern:2002jx}
  Z.~Bern, L.~J.~Dixon and C.~Schmidt,
  %``Isolating a light Higgs boson from the di-photon background at the LHC,''
  Phys.\ Rev.\  D {\bf 66}, 074018 (2002)
  [arXiv:hep-ph/0206194].
  %%CITATION = PHRVA,D66,074018;%%

%\cite{DeFlorian:2000sg}
\bibitem{DeFlorian:2000sg}
  D.~De Florian and A.~Signer,
  %``W gamma and Z gamma production at hadron colliders,''
  Eur.\ Phys.\ J.\  C {\bf 16}, 105 (2000)
  [arXiv:hep-ph/0002138].
  %%CITATION = EPHJA,C16,105;%%

%\cite{Campbell:2008hh}
\bibitem{Campbell:2008hh}
  J.~M.~Campbell, R.~K.~Ellis, F.~Febres Cordero, F.~Maltoni, L.~Reina, D.~Wackeroth and S.~Willenbrock,
  %``Associated Production of a $W$ Boson and One $b$ Jet,''
  Phys.\ Rev.\  D {\bf 79}, 034023 (2009)
  [arXiv:0809.3003 [hep-ph]].
  %%CITATION = PHRVA,D79,034023;%%
  
%\cite{Caola:2011pz}
\bibitem{Caola:2011pz}
  F.~Caola, J.~M.~Campbell, F.~Febres Cordero, L.~Reina and D.~Wackeroth,
  %``NLO QCD predictions for W+1 jet and W+2 jet production with at least one b
  %jet at the 7 TeV LHC,''
  arXiv:1107.3714 [hep-ph].
  %%CITATION = ARXIV:1107.3714;%%

%\cite{Badger:2011yu}
\bibitem{Badger:2011yu}
  S.~Badger, R.~Sattler and V.~Yundin,
  %``One-Loop Helicity Amplitudes for $t\bar{t}$ Production at Hadron
  %Colliders,''
  Phys.\ Rev.\  D {\bf 83}, 074020 (2011)
  [arXiv:1101.5947 [hep-ph]].
  %%CITATION = PHRVA,D83,074020;%%

%\cite{Djouadi:1996yq}
\bibitem{Djouadi:1996yq} 
  A.~Djouadi, V.~Driesen, W.~Hollik and A.~Kraft,
  %``The Higgs photon - Z boson coupling revisited,''
  Eur.\ Phys.\ J.\ C {\bf 1}, 163 (1998)
  [hep-ph/9701342].
  %%CITATION = HEP-PH/9701342;%%

%\cite{Glover:1987nx}
\bibitem{Glover:1987nx} 
  E.~W.~N.~Glover and J.~J.~van der Bij,
  %``Higgs Boson Pair Production Via Gluon Fusion,''
  Nucl.\ Phys.\ B {\bf 309}, 282 (1988).
  %%CITATION = NUPHA,B309,282;%%
  %103 citations counted in INSPIRE as of 20 Aug 2013

%\cite{Martin:2009iq}
\bibitem{Martin:2009iq} 
  A.~D.~Martin, W.~J.~Stirling, R.~S.~Thorne and G.~Watt,
  %``Parton distributions for the LHC,''
  Eur.\ Phys.\ J.\ C {\bf 63}, 189 (2009)
  doi:10.1140/epjc/s10052-009-1072-5
  [arXiv:0901.0002 [hep-ph]].
  %%CITATION = doi:10.1140/epjc/s10052-009-1072-5;%%
  %3270 citations counted in INSPIRE as of 07 Apr 2016

%\cite{Lai:2010vv}
\bibitem{Lai:2010vv} 
  H.~L.~Lai, M.~Guzzi, J.~Huston, Z.~Li, P.~M.~Nadolsky, J.~Pumplin and C.-P.~Yuan,
  %``New parton distributions for collider physics,''
  Phys.\ Rev.\ D {\bf 82}, 074024 (2010)
  doi:10.1103/PhysRevD.82.074024
  [arXiv:1007.2241 [hep-ph]].
  %%CITATION = doi:10.1103/PhysRevD.82.074024;%%
  %1716 citations counted in INSPIRE as of 07 Apr 2016  

%\cite{Dulat:2015mca}
\bibitem{Dulat:2015mca} 
  S.~Dulat {\it et al.},
  %``New parton distribution functions from a global analysis of quantum chromodynamics,''
  Phys.\ Rev.\ D {\bf 93}, no. 3, 033006 (2016)
  doi:10.1103/PhysRevD.93.033006
  [arXiv:1506.07443 [hep-ph]].
  %%CITATION = doi:10.1103/PhysRevD.93.033006;%%
  %96 citations counted in INSPIRE as of 07 Apr 2016

%\cite{Ball:2012cx}
\bibitem{Ball:2012cx} 
  R.~D.~Ball {\it et al.},
  %``Parton distributions with LHC data,''
  Nucl.\ Phys.\ B {\bf 867}, 244 (2013)
  doi:10.1016/j.nuclphysb.2012.10.003
  [arXiv:1207.1303 [hep-ph]].
  %%CITATION = doi:10.1016/j.nuclphysb.2012.10.003;%%
  %539 citations counted in INSPIRE as of 07 Apr 2016
  
%\cite{Ball:2014uwa}
\bibitem{Ball:2014uwa} 
  R.~D.~Ball {\it et al.} [NNPDF Collaboration],
  %``Parton distributions for the LHC Run II,''
  JHEP {\bf 1504}, 040 (2015)
  doi:10.1007/JHEP04(2015)040
  [arXiv:1410.8849 [hep-ph]].
  %%CITATION = doi:10.1007/JHEP04(2015)040;%%
  %225 citations counted in INSPIRE as of 07 Apr 2016
    
%\cite{Harland-Lang:2014zoa}
\bibitem{Harland-Lang:2014zoa} 
  L.~A.~Harland-Lang, A.~D.~Martin, P.~Motylinski and R.~S.~Thorne,
  %``Parton distributions in the LHC era: MMHT 2014 PDFs,''
  Eur.\ Phys.\ J.\ C {\bf 75}, no. 5, 204 (2015)
  doi:10.1140/epjc/s10052-015-3397-6
  [arXiv:1412.3989 [hep-ph]].
  %%CITATION = doi:10.1140/epjc/s10052-015-3397-6;%%
  %142 citations counted in INSPIRE as of 12 May 2016

\end{thebibliography}


\end{document}





